\subsection{Degenererede basisløsninger}
En basisløsning skal have $n$ aktive lineært uafhængige bibetingelser i følge Definition \ref{def:basislosning} (b), det udelukker ikke at en basisløsning kan have flere aktive krav, hvis nogle af kravende er lineært afhængige. 
\begin{defn}[Degenereret basisløsning]
En basisløsning $\vec{x}\in \mathds{R}^n$ siges at være en \textbf{degenereret basisløsning}, hvis basisløsningen har mere en $n$ aktive betingelser.
\end{defn}
Det vil sige, at for $\vec{x}\in \mathds{R}^2$, vil $\vec{x}$ være degenereret, hvis $\vec{x}$ ligger på et skæringspunkt af $3$ betingelser.
Af Sætning \ref{stn:PQ} følger det, at antagelsen om, at alle bibetingelser er lineært uafhængige ikke ændre løsningsmængden, derfor vil denne rapport kun meget overfladisk nævne degenererede løsninger. 
Men da degenerede løsninger kan give nogle problematikker senere i Simplex-metoden, så den terminerer for tidligt, defineres degenereret basisløsninger, for at vise en forståelse for emnet eksistens.
Helt generelt gælder der for basisløsninger til en polyede på standard form med ligheder, at der altid vil $m$ lighedsbetingelser  aktive. Derfor må, det at have mere end $n$ aktive betingelser, være at mere end $n-m$ variabler er $0$
\begin{defn}
En polyede på standard form $P =\{ \vec{x} \in \mathds{R}^n | A \vec{x} = \vec{b}, \vec{b}\in \mathds{R}^m\}$ og lad $\vec{x}$ være en basisløsning. Lad $m$ være mængden af rækker i $A$. Så er $\vec{x}$ en degenereret basisløsning, hvis mere end $n-m$ komponenter af $\vec{x}$ er $0$.
\end{defn}


%I følge af definition \ref{def:basislosning} (b) skal en basis løsning, bare have $n$ lineære uafhængige aktive betingelser i $\mathds{R}^n$, dette giver muligheden for, at der er mere end $n$ aktive betingelser (Der kan selvfølge i $\mathds{R}^n$ ikke være mere end $n$ lineære uafhængige aktive betingelser). Sådan en basisløsning kaldes for en Degenereret basisløsning. Der vil i denne rapport, ikke blive arbejdet med degenereret basisløsninger, da der menes at rapports indhold i forvejen er tilstrækkeligt, dette kan komme til at give nogle problematikker senere i Simplex-metoden, der for den til at terminere for tidligt, så derfor defineres degenereret basisløsninger, for at vise en forståelse for emnet eksistens.
%\begin{defn}[Degenereret basisløsning]
%En basisløsning $\vec{x}\in \mathds{R}^n$ siges at være en \textbf{Degenereret basisløsning}, hvis basisløsningen har mere en $n$ aktive betingelser
%\end{defn}

%Det vil sige, at for $\vec{x}\in \mathds{R}^2$, vil $\vec{x}$ være degenereret, hvis $\vec{x}$ ligger på et skæringspunkt af $3$ betingelser.
%\textit{måske et eksempel}\\
%\subsubsection{Degenereret basisløsninger i standard form}
%En basisløsning til en polyede på standard form, vil $m$ lighedsbetingelser altid være aktive. Derfor må, at have mere end $n$ aktive betingelser, være at mere end $n-m$ variabler er $0$
%\begin{defn}
%En polyede på standard form $P =\{ \vec{x} \in \mathds{R}^n | A \vec{x} = \vec{b}, \vec{b}\in \mathds{R}^m\}$ og lad $\vec{x}$ være en basisløsning. Lad $m$ være mængden af rækker i $A$. Så er $\vec{x}$ en degenereret basisløsning, hvis mere end $n-m$ komponenter af $\vec{x}$ er $0$
%\end{defn}
%Dette vil dog næsten ikke ske, hvis komponenterne af $A$ og $\vec{b}$ er valgt tilfældigt.
%Det der sker er, at når der bliver valgt $m$ rækker af $A$ som danner basis og $m$ variabler bliver udregnet til en basisløsning, så opfylder denne løsning også et andet krav, og derfor vil der under rækkeoperationerne resultere i, at en af de valgte variabler bliver $0$