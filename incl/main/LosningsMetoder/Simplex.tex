\section{Simplex og Basisløsninger}
Den anden tager udgangspunkt i resultatet fra Sætning \ref{stn:eksistens}, der siger at en \textbf{optimal løsning ?!?} altid er en basisløsning, og viser hvordan simplexer kan bruges til at vælge en basis, som vil føre til en basisløsning med optimal værdi.
Der vil derfor med udgangspunkt i %indsæt kilde
blive gennemgået definitionen for en simplex og hvordan det kan bruges til at udtrykke en basisløsning.
\subsection{Simplex}
For at kunne definere en simplex, er det nødvendigt først at introducere begreberne konvekskombination og konvekshyslser.
\begin{defn}[Konveks kombination]
Lad $\vec{x}^1, ...,\vec{x}^k \in \mathds{R}^n$, og $\lambda_1,..., \lambda_k \geq 0 $ være skalare, som opfylder $\sum_{i=1}^k \lambda_i =1$ da er $\sum_{i=1}^k \lambda_i \vec{x}^1$ en \textbf{konveks kombination}.
\label{def:KonveksKombination}
\end{defn}
En konveks kombination er dermed et sær tilfælde af en lineær kombination, hvor skalrene summer til $1$.
\begin{defn}[Konveks hylster]
Lad $\vec{x}_1, ...,\vec{x}_k \in \mathds{R}^n$, da er $C_{x} = \{\sum_{i=1}^k \lambda_i \vec{x}_i| \vec{x}_1, ...,\vec{x}_k \in \mathds{R}^n, \sum_{i=1}^k \lambda_i =1\}$ et \textbf{konveks hylster} for vektorene $\vec{x}_1, ...,\vec{x}_k$. 
\label{def:Konvekshuld}
\end{defn}
Mens, at mængden af alle konvekse kombinationer af en given mængde vektorer, kaldes et konvekshylster, og minder derfor om spandet af en mængde vektorer.
Et særtilfælde af konveksehylstre er en simplex.
\begin{defn}[Simplex]
Lad $C_x$ være et konveks huld, af $k+1$ affint lineært uafhængige vektorer, da er $C_x$ en $k$-dimentionel \textbf{Simplex}.
\end{defn}
Eftersom det konveksehylster, kun består af konvekse kombinationer, vil det give mening at det udgør en konveks mængde.
\begin{stn}
Konveks huldet $C_x = \{\sum_{i=1}^k \lambda_i \vec{x}_1| \vec{x}_1, ...,\vec{x}_k \in \mathds{R}^n, \sum_{i=1}^k \lambda_i =1\}$ over en endelig mængde vektorer, er en konveks mængde
\end{stn}
\begin{proof}
Lad $\vec{z}, \vec{y}\in C_x = \{\sum_{i=1}^k \lambda_i \vec{x}_i| \vec{x}_1, ...,\vec{x}_k \in \mathds{R}^n, \sum_{i=1}^k \lambda_i =1\}$ være vilkårlige vektorer da må $\vec{z}= \sum_{i=1}^k \gamma_i \vec{x}_i, \vec{y}= \sum_{i=1}^k \eta_i \vec{x}_i$ for $\sum_{i=1}^k \gamma_i = 1$ og  $\sum_{i=1}^k \eta_i = 1$. 
Derfor må
\begin{align*}
	\lambda \vec{z} + (1- \lambda) \vec{y} &= \lambda\sum_{i=1}^k \gamma_i \vec{x}_i + (1-\lambda)\sum_{i=1}^k \eta_i \vec{x}_i
	\\ &=\sum_{i=1}^k (\lambda \gamma_i+(1-\lambda)\eta_i )\vec{x}_i,
\end{align*}
For $\lambda \in [0,1]$.
Betragt nu konstanterne 
\begin{align*}
	\sum_{i=1}^k (\lambda \gamma_i+(1-\lambda)\eta_i ) &= \lambda \sum_{i=1}^k \gamma_i + (1 - \lambda) \sum_{i=1}^k \eta_i 
	\\ &= \lambda \cdot 1 + (1 - \lambda) \cdot 1 = 1
\end{align*}
Hvorfor at $\lambda \vec{z} + (1- \lambda) \vec{y} $ er en konveks kombination af vektorene $\vec{x}_1, ...,\vec{x}_k $, ifølge Definition \ref{def:KonveksKombination}. 
Derfor må $ \lambda \vec{z} + (1- \lambda) \vec{y} \in C_x$, hvorfor at $C_x$ er konveks ifølge Definition \ref{def:Konveks}.
Og sætningen er bevist.
\end{proof}
Dermed er en simplex en konveks mængde udspændt af affint lineære vektorer.
Dette kan bruges til geometrisk at repræsenterer basisløsninger og deres basis matrix.
