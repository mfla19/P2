\section{Konveks hylster}
Ud over at mængder kan være konvekse så kan lineare kombinationer også være konvekse.
\begin{defn}[Konveks kombination]
Lad $\vec{x}^1, ...,\vec{x}^k \in \mathds{R}^n$, og $\lambda_1,..., \lambda_k \geq 0 $ være skalare, som opfylder $\sum_{i=1}^k \lambda_i =1$ da er $\sum_{i=1}^k \lambda_i \vec{x}^1$ en \textbf{konveks kombination}.
\label{def:KonveksKombination}
\end{defn}
Dermed er en konveks kombination et særtilfælde af lineære kombinationer, hvor skalerne summer til $1$.
På samme måde er der et særtilfælde af et span, kaldet et konveks hylster, som er alle konvekse kombinationer udspændt af en mængde vektorer.
\begin{defn}[Konveks hylster]
Lad $\vec{x}_1, ...,\vec{x}_k \in \mathds{R}^n$, da er $C_{x} = \{\sum_{i=1}^k \lambda_i \vec{x}_i| \vec{x}_1, ...,\vec{x}_k \in \mathds{R}^n, \sum_{i=1}^k \lambda_i =1\}$ et \textbf{konveks hylster} for vektorene $\vec{x}_1, ...,\vec{x}_k$. 
\label{def:Konvekshuld}
\end{defn}
Efter som at enhver konveks kombination af elementer fra en konveks mængde er et element i mængden, vil det give mening at konvekshuldet også var en konveks mængde.
\begin{stn}[Konvekse mængder]
Konveks huldet $C_x = \{\sum_{i=1}^k \lambda_i \vec{x}_1| \vec{x}_1, ...,\vec{x}_k \in \mathds{R}^n, \sum_{i=1}^k \lambda_i =1\}$ over en endelig mængde vektorer, er en konveks mængde
\end{stn}
\begin{proof}
Lad $\vec{z}, \vec{y}\in C_x = \{\sum_{i=1}^k \lambda_i \vec{x}_i| \vec{x}_1, ...,\vec{x}_k \in \mathds{R}^n, \sum_{i=1}^k \lambda_i =1\}$ være vilkårlige vektorer da må $\vec{z}= \sum_{i=1}^k \gamma_i \vec{x}_i, \vec{y}= \sum_{i=1}^k \eta_i \vec{x}_i$ for $\sum_{i=1}^k \gamma_i = 1$ og  $\sum_{i=1}^k \eta_i = 1$. 
Derfor må
\begin{align*}
	\lambda \vec{z} + (1- \lambda) \vec{y} &= \lambda\sum_{i=1}^k \gamma_i \vec{x}_i + (1-\lambda)\sum_{i=1}^k \eta_i \vec{x}_i
	\\ &=\sum_{i=1}^k (\lambda \gamma_i+(1-\lambda)\eta_i )\vec{x}_i,
\end{align*}
For $\lambda \in [0,1]$.
Betragt nu konstanterne 
\begin{align*}
	\sum_{i=1}^k (\lambda \gamma_i+(1-\lambda)\eta_i ) &= \lambda \sum_{i=1}^k \gamma_i + (1 - \lambda) \sum_{i=1}^k \eta_i 
	\\ &= \lambda \cdot 1 + (1 - \lambda) \cdot 1 = 1
\end{align*}
Hvorfor at $\lambda \vec{z} + (1- \lambda) \vec{y} $ er en konveks kombination af vektorene $\vec{x}_1, ...,\vec{x}_k $, ifølge Definition \ref{def:KonveksKombination}. 
Derfor må $ \lambda \vec{z} + (1- \lambda) \vec{y} \in C_x$, hvorfor at $C_x$ er konveks ifølge Definition \ref{def:Konveks}.
Og sætningen er bevist.
\end{proof}
Et særtilfælde af konvekshuld er kaldet en simplex.
\begin{defn}[Simplex]
Lad $C_x$ være et konveks huld, af $k+1$ affint lineært uafhængige vektorer, da er $C_x$ en $k$-dimentionel \textbf{Simplex}.
\end{defn}
Simplex metoden bygger på at betragte forskellige simplexer, og så finde den simplex, hvis skæring med $\vec{b}$ er lig den optimale løsning, det kan lade sig gøre da søjlerne i en basismatrix til en basisløsning udspænder en simplex.