\chapter{lineær programmering}

Lineær programmering er en anvendelse af lineær algebra til at finde den optimale resultat af et optimeringsproblem. Lineære programmeringsproblemer tager udgangspunkt i maksimering eller minimering af en lineær funktion. For variablene er der fastsat en række af betingelser, som begrænser de mulige løsninger til problemet.



Funktionen, som ønskes optimeret, kaldes \textbf{kriteriefunktionen}, og findes på formen:
\begin{align}
c_1x_1 + c_2x_2 + \cdots + c_nx_n.
\end{align}

Dertil tilføjes en række af betingelser for variablene. Betingelser, som definerer at en variabel er positiv, kaldes en positivitetsbetingelse. En 


\begin{eks}
\begin{center}
\begin{tabular}{@{}	l	>{$}r<{$}	>{$}r<{$}	>{$}l<{$}	@{}}
Maksimer 				& 4x_1&		+3 x_2& \\
med hensyn til 			& -x_1& 	+4 x_2& \leq 12\\
						&  x_1& 	+2 x_2& \leq 18\\
						& 2x_1& 	-\ x_2& \leq 26\\
						&  x_1&			  & \geq 0\\
						&  	  &		   x_2&	\geq 0
\end{tabular}
\end{center}
\end{eks}


Hvad skal med i dette afsnit?
- hvad betyder lineær programmering.
	en måde at anvende lineær algebra til at løse nogle forskellige problemer, hvor det gælder om at optimere et resultat.
	I et lineært programmeringsproblem er der en funktion, kaldet den objektive funktion, som forsøges optimeret.
	for et sådan problem er der fastsat en række begrænsninger. 
	fælles for den objektive funktion og begrænsningerne er at de alle er lineære funktioner. 
	
Objektfunktionen findes derved på formen:
$ F: \mathds{R}^n \rightarrow \mathds{R} 
$ hvor n er antallet af variable, som indgår i problemet.

Kriteriefunktionen er en lineær funktion af indgangene i vektoren $\vec{x}$.

Bibetingelserne findes på formen:
$\vec{a}$



%\end{comment}