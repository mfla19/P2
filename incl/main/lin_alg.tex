\chapter{Lineær algebra}
Dette kapitel er skrevet med udgangspunkt i 

Lineær algebra bruges til at løse lineære ligningssystemer. 
En af de mest nyttige algoritmer, til at løse sådanne ligningssystemer, er Gaussisk elimination, som vil blive beskrevet senere i dette kapitel.

\section{Matricer}
Lineære ligningssystemer kan opskrives i matricer. 
En matrix er defineret i Definition \ref{def:matricer}.

\begin{defn} [Matrix]\label{def:matricer}
En matrix er en rektangulær tabel over skalarer. 
Størrelsen på en matrix er $m \times n$, hvor $m$ er antal rækker og $n$ er antal søjler. 
En matrix kaldes kvadratisk hvis $m=n$. 
Skalaren i den $i$'te række og $j$'te søjle kaldes $(i,j)$-indgangen.
\end{defn}

Herunder ses en matrix $\underset{m \times n}{A}$ og dens indgange, $a_{i,j}$. Den $i$'te række i A betenges $\vec{a}_i$, mens den $j$'te søjle angives $\vec{A}_j$.

\begin{align*}
\underset{m \times n}{A} = \begin{bmatrix}
	a_{1,1} & a_{1,2} & \dots & a_{1,j} & \dots & a_{1,n} \\
	a_{2,1} & \ddots  &       &         &       & \vdots \\
	\vdots  &         & \ddots &        &       & \vdots \\
	a_{i,1} &         &       & a_{i,j} &       & \vdots \\
	\vdots  &         &       &         & \ddots& \vdots \\
	a_{m,1} & \dots   & \dots & \dots   & \dots & a_{m,n} 
\end{bmatrix}
\end{align*}

Matricer kan bruges til at vise mange forskellige ting fra den virkelige verden, det kan være antal af varer solgt fra forskellige butikker, eller prisen på forskellige produkter. 

\begin{eks}
Herunder ses en $4 \times 3$-matrix, der viser antallet af solgte varer fra tre forskellige butikker. Hver række i matricen repræsenterer en bestemt vare, mens hver søjle repræsenterer en butik. Den første række viser altså hvor mange trøjer hver butik har solgt.
\begin{align*}
\begin{matrix}
	Trøjer \\
	Kjoler \\
	Bukser \\
	Jakker
\end{matrix}
\begin{bmatrix}
	24 & 35 & 43 \\
	10 & 47 & 24 \\
	33 & 25 & 32 \\
	28 & 51 & 37
\end{bmatrix}
\end{align*}
I denne matrice kan man se, at der i dens $(2,3)$-indgang står $24$, hvilket betyder, at den tredje butik har solgt $24$ kjoler. Der står i dens $(3,1)$-indgang $33$, hvilket betyder, at den første butik har solgt $33$ par bukser.
\end{eks}

\begin{defn}[Transponeret matrix]
Lad $A$ være en $m \times n$ matrix. Så er den transponerede matrix, $A^T$, en $n \times n$ matrix, hvor hver indgang i $A^T$, $(i,j)$ er den $(j,i)$'te indgang i $A$
\label{def:(transmatrix)} 
\end{defn}
Det vil sige at rækkerne i $A$, bliver til søjlerne i $A^T$. Og søjlerne i $A$ bliver til rækkerne i $A^T$.

\begin{eks}
Givet en matrix $A$ 	bestemmes nu den transponerede
\begin{align*}
A = \begin{bmatrix}
	5 & 3 & 3 \\
	1 & 2 & 5
\end{bmatrix}
\end{align*}

\begin{align*}
A^T = \begin{bmatrix}
	5 & 1  \\
	3 & 2  \\
	3 & 5
\end{bmatrix}
\end{align*}
Det ses at $A$ er en $2 \times 3$ matrix og $A^T$ er en $3 \times 2$ matrix. 
\end{eks}


En bestemt type af matrix er en identitetsmatrix, der betegnes $I_n$. 

\begin{defn} [Identitetsmatrix]\label{def:imatrix}
For hvert positivt heltal $n$, er $n \times n$ identitetsmatricen, $I_n$, den $n \times n$ matrix, hvor hver søjle er standardvektorerne $\vec{e_1}$, $\vec{e_2}$, $\dots$, $\vec{e_n}$ i $\mathbb{R}^n$.
\end{defn}

En identitetsmatrix med $3$ rækker og $3$ søjler, ser således ud:
\begin{align*}
I_3 = \begin{bmatrix}
	1 & 0 & 0 \\
	0 & 1 & 0 \\
	0 & 0 & 1 
\end{bmatrix}
\end{align*}
Identitesmatricen bruges til at definere den inverse matrix til en matrix. 

\begin{defn}[Invers matrix]
Lad $A$ og $B$ være kvadratiske $m \times m$ matricer. Lad $AB=BA=I_m$, hvor $I_m$ er identitetsmatricen. Så er A invertibel og B er den inverse matrix til A. 
\label{def(inversmatrix)}
\end{defn}

\begin{defn} [Delmatrix]\label{delmatrix}
En matrix $A'$ er en delmatrix af $A$, hvis $A'$ kan dannes ved at fjerne hele søjler og/eller hele rækker fra $A$.
\end{defn}

Givet en matrix $A$ kan der dannes delmatricen $A'$.

\begin{align*}
A = \begin{bmatrix}
	1 & 2 & 3 \\
	4 & 5 & 6 \\
	7 & 8 & 9 
\end{bmatrix}
\end{align*}

\begin{align*}
A' = \begin{bmatrix}
	5 & 6 \\
	8 & 9
\end{bmatrix}
\end{align*}

\section{Regneoperationer med matricer}
Givet to $m \times n$ matricer $A$ og $B$ kan der udføres forskellige regneoperationer. 
Summen af to matricer findes ved at addere en indgang i $A$ med tilsvarende indgang i $B$, så $A+B$ er en $m \times n$ matrix, hvor indgang $(i,j)$ er $a_{i,j}+b_{i,j}$. 
Det samme gør sig gældende ved substraktion. \\
Givet en $m \times n$ matrix $A$ og en skalar $c$, er produktet af skalaren og matricen, $cA$, en $m \times n$ matrix, hvor indgangene er $c$ gange den tilsvarende indgang i $A$. \\

\begin{stn}
Lad $A$, $B$ og $C$ være $m \times n$ matricer, og lad $s$ og $t$ være tilfældige skalarer. 
Så gælder følgende
\begin{enumerate}[label=(\alph*)]
\item $A + B = B + A$
\item $(A + B) + C = A + (B + C)$
\item $A + O = A$
\item $A + (-A) = O$
\item $(st) A = s (tA)$
\item $s(A + B) = sA + sB$
\item $(s+t)A = sA + tA$
\end{enumerate}
\label{stn_regn}
\end{stn}

\begin{proof} 
Lad $A$, $B$ og $C$ være $m \times n$ matricer. Lad matricen $O$ være en nulmatrix. \\
(a) Det vises, at enhver indgang i $A + B$ er tilsvarende i $B + A$. Betragt indgangene $a_{i,j}$ og $b_{i,j}$. Summen af $a_{i,j} + b_{i,j}$ er det samme som $b_{i,j} + a_{i,j}$. \\
(b) Det vises, at enhver indgang i $(A + B) + C$ er den samme som den tilsvarende indgang i $A + (B + C)$. Ligesom i (a) tages der udgangspunkt i indgang $(i,j)$. Summen af $(a_{i,j} + b_{i,j}) + c_{i,j}$ er det samme som $a_{i,j} + (b_{i,j} + c_{i,j})$. Ifølge den associative lov, er det uden betydning hvor paranteserne er placeret i et additionsudtryk. Derfor må indgangen $(i,j)$ i $(A + B) + C$ være lig indgang $(i,j)$ i $A + (B + C)$. \\
(c) For enhver indgang i $A$, $a_{i,j}$ skal denne adderes med $0$. Derfor må $A + O = A$. \\
(d) For enhver indgang i $A$, $a_{i,j}$ skal denne fratrækkes samme indgang i $A$, $a_{i,j}$. Derfor må $A + (-A) = O$. \\
(e) Her tages der udgangspunkt i den (i,j)-indgang af A, $a_{i,j}$. Det ses så, at udsagnet bliver til $(st)a_{i,j} = s(ta_{i,j})$ når flere tal multipliceres, er det lige meget i hvilken rækkefølge, dermed er (e) bevist. \\
(f) Hver indgang i A, $a_{i,j}$ lægges sammen med den tilsvarende indgang i B, $b_{i,j}$, dette giver venstresiden $s(a_{i,j}+b_{i,j})$. Da det er tilladt at multiplicere ind i en parantes, kan udtrykket skrives sådan, $s(a_{i,j}+b_{i,j})=sa_{i,j}+sb_{i,j}$, dermed er (f) bevist. \\
(g) På samme måde som før tages der udgangspunkt i den (i,j)-indgang af A, $a_{i,j}$. Udtrykket på venstresiden er så $(s+t)a_{i,j}$. Igen kan der multipliceres ind i parantesen og dermed fås følgende, $(s+t)a_{i,j}=sa_{i,j}+ta_{i,j}$, hvilket beviser (g).
\end{proof}

\begin{eks}
For at vise eksempler på nogle af de ovenstående regneoperationer, tages der udgangspunkt i matricerne A og B, samt skalaren s.
\begin{align*}
A= \begin{bmatrix}
	2 & 3 & 4 \\
	5 & -2 & 1 	
\end{bmatrix}  
B= \begin{bmatrix}
	1 & 2 & -1 \\
	-3 & 4 & 0
\end{bmatrix}  
s=2
\end{align*}
Først udføres regneoperation (a) fra Sætning \ref{stn_regn},
\begin{align*}
A+B= \begin{bmatrix}
	2 & 3 & 4 \\
	5 & -2 & 1 	
\end{bmatrix}  
+ \begin{bmatrix}
	1 & 2 & -1 \\
	-3 & 4 & 0
\end{bmatrix}
= \begin{bmatrix}
	3 & 5 & 3 \\
	2 & 2 & 1
\end{bmatrix}.
\end{align*}
Herfter udføres (f),
\begin{align*}
s(A+B)= 5 \times \left( \begin{bmatrix}
	2 & 3 & 4 \\
	5 & -2 & 1 	
\end{bmatrix}  
+ \begin{bmatrix}
	1 & 2 & -1 \\
	-3 & 4 & 0
\end{bmatrix} \right)
= 5 \times \begin{bmatrix}
	3 & 5 & 3 \\
	2 & 2 & 1
\end{bmatrix}
= \begin{bmatrix}
	15 & 25 & 15 \\
	10 & 10 & 5
\end{bmatrix}.
\end{align*}
\end{eks}


\subsection{Matrix produkt}
Multiplikation af to matricer gøres ikke ved af multiplicere på tilsvarende indegange i to matricer. Det at gange to matricer sammen defineres herunder. 
\begin{defn} [Matrix produkt]
Lad $A$ være en $m \times n$ matrix og $B$ være en $n \times p$ matrix. Da er en produktet $A \dot B$ en $m \times p$ matrix, hvor indgangene er givet ved: 

$$ad_{i,j} = \vec{a}_i \cdot \vec{B}_j$$

hvor $\vec{a}_i$ er den $i$'te række i $A$, og $\vec{B}_j$ er den $j$'te søjle i $B$
\label{def:(matrixprodukt)}
\end{defn}
Det er vigtigt at antallet af søjler i $A$ er det samme som antallet af rækker i $B$. Er det ikke tilfældet, så er matrix produktet ikke defineret. Matrix produktet er oftest ikke kommutativt. Ligningen $AB=AB$ er derfor ikke sjældent gældende. 
\begin{eks}
Nu tages udgangspunkt i to matricer $A$ og $B$. 
\begin{align*}
\underset{2 \times 3}{A}= \begin{bmatrix}
	\bf{1} & \bf{3} & \bf{-2} \\
	5 & 4 & 0 	
\end{bmatrix}  
\underset{3 \times 4}{B}= \begin{bmatrix}
	\bf{2} & 3 & -1 & 3 \\
	\bf{1} & 4 & 5 & 5\\
	\bf{1} & 0 & 4 & 2
\end{bmatrix}  
\end{align*}
For at beregne indgang $(1,1)$ benyttes række et i $A$ og søjle et i $B$. 
$$ab_{1,1}=1\cdot 2+3\cdot 1-2 \cdot 1 = 3$$ 
De resterende indgange beregens på samme måde. 
Matrix produkt af $A$ og $B$ er dermed givet ved:
\begin{align*}
\underset{2 \times 4}{AB}= \begin{bmatrix}
	\bf{3} & 15 & 6 & 14 \\
	14 & 31 & 15 & 35
\end{bmatrix}  
\end{align*}
\end{eks}




\section{Lineære ligningssystemer}
Lineære ligninger, der indeholder ukendte variabler, kan skrives på formen

\begin{align*}
a_1x_1+a_2x_2+ \dots +a_nx_n = b,
\end{align*}

hvor $a_1, a_2, \dots , a_n$ og $b$ er reelle tal. 
Her kaldes $a_1,a_2, \dots , a_n$ koefficienter og $b$ er en konstant. En lineær ligning kunne for eksempel se sådan ud:

\begin{align*}
7x_1+3x_2-5x_3 = 10.
\end{align*}

Lineære ligninger må ikke indeholde to variable multipliceret, kvadratroden af en variabel, eller andet der gør den ikke-lineær. \\
Et sæt af $m$ lineære ligninger, der indeholder de samme $n$ variable, hvor både $n$ og $m$ er positive heltal, kaldes et lineært ligningssystem. Lineære ligningssystemer skrives på formen

\begin{align*}
a_{1,1}x_1+a_{1,2}x_2+ &\dots +a_{1,n}x_n = b_1\\
a_{2,1}x_1+a_{2,2}x_2+ &\dots +a_{2,n}x_n = b_2\\
&\vdots \\
a_{m,1}x_1+a_{m,2}x_2+ &\dots +a_{m,n}x_n = b_m
\end{align*}

Koefficienterne i et lineært ligningssystem kan skrives i en matrix, mens variable og konstanterne skrives som vektorer. Et lineært ligningsystem kan så skrives op som en matrix ligning på formen $A \vec{x} = \vec{b}$, hvor

\begin{align*}
A= \begin{bmatrix}
a_{1,1} & a_{1,2} & \dots & a_{1,n} \\
a_{2,1} & a_{2,2} & \dots & a_{2,n} \\
\vdots  &         &       & \vdots  \\
a_{m,1} & a_{m,2} & \dots & a_{m,n}
\end{bmatrix}, \ 
x= \begin{bmatrix}
x_1 \\
x_2 \\
\vdots \\
x_n
\end{bmatrix} og \ 
b= \begin{bmatrix}
b_1 \\
b_2 \\
\vdots \\
b_n
\end{bmatrix}
\end{align*}

Søjlerne i $A$ indeholder koefficienterne $x_1$, $x_2$, $\dots $, $x_n$ og kaldes derfor koefficientmatricen til det lineære ligningssystem. 
Den information, der er nødvendig for at løse et lineært ligningssystem, kan samles i en totalmatrix på formen:
\[
\left[
\begin{array}{cccc|c}
a_{1,1} & a_{1,2} & \dots & a_{1,n} & b_1 \\
a_{2,1} & a_{2,2} & \dots & a_{2,n} & b_2 \\
\vdots  &         &       &         & \vdots \\
a_{m,1} & a_{m,2} & \dots & a_{m,n} & b_n
\end{array}
\right]
\]

Totalmatricen laves ved at tilføje vektor $\vec{b}$ til koefficientmatricen $A$. Totalmatricen noteres således som $[A \ \vec{b}]$. \\

Løsningen til et lineært ligningssystem er en vektor i $\mathds{R}^n$, som ser således ud:

\begin{align*}
\begin{bmatrix}
s_1 \\
s_2 \\
\vdots \\
s_n
\end{bmatrix}
\end{align*}

Hvis $A$ er en $m \times n$ matrix, er en vektor $\vec{u}$  i $\mathds{R}^n$ en løsning til $A \vec{x}= \vec{b}$ hvis og kun hvis $A \vec{u}= \vec{b}$.

\begin{eks}
Følgende lineære ligningssystem er givet

\begin{align*}
8x_1+6x_2+4x_3 = 50 \\
2x_1+4x_2+6x_3 = 30.
\end{align*}

Dette kan så skrives i en matrix,

\begin{align*}
\begin{bmatrix}
8 & 6 & 4 & 50 \\
2 & 4 & 6 & 30
\end{bmatrix}.
\end{align*}

Løsningen til dette ligningssystem bliver følgende vektor

\begin{align*}
\begin{bmatrix}
2 \\
5 \\
1
\end{bmatrix}.
\end{align*}

Sætter man løsningen ind som variable fås følgende resultat,

\begin{align*}
8 \cdot 2 + 6 \cdot 5 + 4 \cdot 1 = 50 \\
2 \cdot 2 + 4 \cdot 5 + 6 \cdot 1 = 30.
\end{align*}

Det ses, at vektoren giver en rigtig løsning til ligningssystemet, fordi alle ligningerne går op når vektoren indsættes. 

\end{eks}

