\subsection{Underrum}
Spannet for en mængde af vektorer $S$ udspænder et underrum, som er en delmængde at vektorer som indeholder nulvektoren, og er lukket under vektor addition og skalar multiplikation.
\begin{defn}[Underrum]
Lad $W$ være en mængde af vektorer $\vec{v_1},...,\vec{v_k} \in \mathds{R}^n$, da er $W$  \textbf{underrum} til $\mathds{R}^n$, hvis:
\begin{enumerate}[label=\alph*]
\item $\vec{0} \in W$
\item $\vec{u}+\vec{v} \in W \quad \forall \vec{u}, \vec{v} \in W$
\item $c \cdot \vec{v} \in W \quad \forall \vec{v} \in W, \forall c \in \mathds{R}$
\end{enumerate}
\label{def:underrum}
\end{defn}
For at vise at spannet er mængden af vektorer $S$ er et underrum, skal det vises at nul vektoren kan skrives som en linear kombination af vektorene i $S$, det samme gælder for summen af to vektorer og skalaproduktet, af to vektorer.
\begin{stn}[Span er et underrum]
Lad $S=\{\vec{v_1},...,\vec{v_k}\} \subseteq \mathds{R}^n$, da er $span(S)$ et underrum til $\mathds{R}^n$
\label{stn:spanunderrum}
\end{stn}
\begin{proof}
For at vise at $span(S)$ er et underrum til $\mathds{R}^n$ skal det vises at $span(S)$ overholder alle betingelserne i Definition \ref{def:underrum}.
Først vises betingelse (b), lad  derfor $\vec{u}, \vec{v} \in span(S)$, da vil 
\begin{align*}
\vec{u}+\vec{v}= \sum_{i=1}^k c_i \vec{v_i} + \sum_{i=1}^k c'_i \vec{v_i} = \sum_{i=1} c_i\cdot c_i' \vec{v_i},
\end{align*}
hvor $c_i, c_i'$ er skalare.
Der med er $\vec{u}+\vec{v}$ en linear kombination af $\vec{v_1},...,\vec{v_k}$, hvorfor $\vec{u}+\vec{v} \in span(S)$, og $span(S)$ er lukket under vektor addition.
\\ Så vises at $span(S)$ er lukket under skalar multiplikation, lad derfor $c, c_i$ være skalare, da vil
\begin{align*}
c\vec{v}= c\sum_{i=1}^k c_i \vec{v_i}  = \sum_{i=1} c \cdot c_i \vec{v_i}.
\end{align*}
Hvorfor at $c\vec{v} \in span(S)$, og betingelse (c), er opfyldt.
\\Tilsidst vises det at $\vec{0} \in span(S)$.
Da nul vektoren kan skrives som den linear kombination af vektorene i $S$, $\vec{0} = \sum_{i=1}^k 0 \vec{v_i}$, medfører det at $\vec{0} \in span(S)$, hvorfor at $span(S)$ overholder betingelse (a), og dermed er $span(S)$ et underrum til $\mathds{R}^n$.
\end{proof}
Betragtes en $n\times m$ matrix igen, så vil den udspænde $\mathds{R}^m$ hvis den rang var $m$, er dens rang mindre end $m$, da udspænder dens søjler et underrum til $\mathds{R}^m$ kaldet søjlerummet.
\begin{defn}[Søjlerum]
Lad $A$ være en $n\times m$ matrix, da er \textbf{søjlerummet} af $A$ $Col A = span(\{A_j | j =1,...,n\}$
\label{def:sojlerum}
\end{defn} 

