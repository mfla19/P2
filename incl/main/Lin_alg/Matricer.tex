\section{Matricer}
Lineære ligningssystemer kan opskrives i matricer. 
En matrix er defineret i Definition \ref{def:matricer}.

\begin{defn} [Matrix]
En $m \times n$ matrix er en rektangulær tabel over skalarer med $m$ rækker og $n$ søjler. 
Hvis $m=n$ er matricen kvadratisk. 
Skalaren i $i$'te række og $j$'te søjle kaldes $(i,j)$-indgangen.
\label{def:matricer}
\end{defn}

Herunder ses en $m \times n$ matrix $A$ og dens indgange, $a_{i,j}$. Den $i$'te række i A betenges $\vec{a}_i$, mens den $j$'te søjle angives $\vec{A}_j$.

\begin{align*}
A = \begin{bmatrix}
	a_{1,1} & a_{1,2} & \dots & a_{1,j} & \dots & a_{1,n} \\
	a_{2,1} & \ddots  &       &         &       & \vdots \\
	\vdots  &         & \ddots &        &       & \vdots \\
	a_{i,1} &         &       & a_{i,j} &       & \vdots \\
	\vdots  &         &       &         & \ddots& \vdots \\
	a_{m,1} & \dots   & \dots & \dots   & \dots & a_{m,n} 
\end{bmatrix}
\end{align*}

Matricer kan bruges til at illustrere forskellige ting fra den virkelige verden. Det kan være antal varer solgt fra forskellige butikker, eller prisen på forskellige produkter. 

\begin{eks}
Herunder ses en $4 \times 3$ matrix, der viser antallet af solgte varer fra tre forskellige butikker. Hver række i matricen repræsenterer en bestemt vare, mens hver søjle repræsenterer en butik. Den første række viser altså hvor mange trøjer hver butik har solgt.
\begin{align*}
\begin{matrix}
	Trøjer \\
	Kjoler \\
	Bukser \\
	Jakker
\end{matrix}
\begin{bmatrix}
	24 & 35 & 43 \\
	10 & 47 & 24 \\
	33 & 25 & 32 \\
	28 & 51 & 37
\end{bmatrix}
\end{align*}
I matricens $(2,3)$-indgang står der $24$, hvilket betyder, at den tredje butik har solgt $24$ kjoler. I $(3,1)$-indgangen står $33$, så den første butik har solgt $33$ par bukser. 
\end{eks}

\begin{defn}[Transponeret matrix]
Lad $A$ være en $m \times n$ matrix. Så er den transponerede matrix, $A^T$, en $n \times m$ matrix, hvor hver indgang $(j,i)$ i $A^T$, er den $(i,j)$'te indgang i $A$
\label{def:(transmatrix)} 
\end{defn}
Det vil sige, at rækkerne i $A$, bliver til søjlerne i $A^T$, og søjlerne i $A$ bliver til rækkerne i $A^T$.

\begin{eks}
Givet en matrix $A$ bestemmes nu den transponerede $A^T$
\begin{align*}
A = \begin{bmatrix}
	5 & 3 & 3 \\
	1 & 2 & 5
\end{bmatrix}
\end{align*}

\begin{align*}
A^T = \begin{bmatrix}
	5 & 1  \\
	3 & 2  \\
	3 & 5
\end{bmatrix}
\end{align*}
Det ses at $A$ er en $2 \times 3$ matrix og $A^T$ er en $3 \times 2$ matrix. 
\end{eks}


En bestemt type matrix er en identitetsmatrix, der betegnes $I_n$. 

\begin{defn} [Identitetsmatrix]
For hvert positivt heltal $n$, er $n \times n$ identitetsmatricen, $I_n$, defineret som\\ $I_n = \rvect{\vec{e_1} & \vec{e_2} & \dots &  \vec{e_n}}$, hvor $\vec{e_1}$, $\vec{e_2}$, $\dots$, $\vec{e_n}$ er standardvektorerne i $\mathds{R}^n$
\label{def:imatrix}
\end{defn}

En identitetsmatrix med $3$ rækker og $3$ søjler, ser således ud:
\begin{align*}
I_3 = \begin{bmatrix}
	1 & 0 & 0 \\
	0 & 1 & 0 \\
	0 & 0 & 1 
\end{bmatrix}
\end{align*}

En anden bestemt type matrix er en delmatrix, som er en form for undermatrix. 
\begin{defn} [Delmatrix]
En matrix $A'$ er en delmatrix af $A$, hvis $A'$ kan dannes ved at fjerne hele søjler og/eller hele rækker fra $A$.
\label{delmatrix}
\end{defn}

Givet en matrix $A$ kan der dannes delmatricen $A'$.

\begin{align*}
A = \begin{bmatrix}
	1 & 2 & 3 \\
	4 & 5 & 6 \\
	7 & 8 & 9 
\end{bmatrix}, \quad \quad
A' = \begin{bmatrix}
	5 & 6 \\
	8 & 9
\end{bmatrix}
\end{align*}

\section{Regneoperationer med matricer}
Givet to $m \times n$ matricer, $A$ og $B$, kan der udføres forskellige regneoperationer. \\

\begin{stn}
Lad $A$, $B$ og $C$ være $m \times n$ matricer, og lad $s$ og $t$ være tilfældige skalarer. Lad matricen $O$ være en nulmatrix.
Så gælder følgende
\begin{enumerate}[label=(\alph*)]
\item $A + B = B + A$
\item $(A + B) + C = A + (B + C)$
\item $A + O = A$
\item $A + (-A) = O$
\item $(st) A = s (tA)$
\item $s(A + B) = sA + sB$
\item $(s+t)A = sA + tA$
\end{enumerate}
\label{stn_regn}
\end{stn}

\begin{proof} 
(a) Det vises, at enhver indgang i $A + B$ er tilsvarende i $B + A$. Betragt indgangene $a_{i,j}$ og $b_{i,j}$. Summen af $a_{i,j} + b_{i,j}$ er det samme som $b_{i,j} + a_{i,j}$, jf. den kommutative lov. \\
(b) Det vises, at enhver indgang i $(A + B) + C$ er den samme som den tilsvarende indgang i $A + (B + C)$. Ligesom i (a) tages der udgangspunkt i indgang $(i,j)$. Summen af $(a_{i,j} + b_{i,j}) + c_{i,j}$ er det samme som $a_{i,j} + (b_{i,j} + c_{i,j})$. Ifølge den associative lov, er det uden betydning hvor paranteserne er placeret i et additionsudtryk. Derfor må indgangen $(i,j)$ i $(A + B) + C$ være lig indgang $(i,j)$ i $A + (B + C)$. \\
(c) For enhver indgang i $A$, $a_{i,j}$ skal denne adderes med $0$. $a_{i,j}+0=a_{i,j}$. Derfor må $A + O = A$. \\
(d) For enhver indgang i $A$, $a_{i,j}$ skal denne fratrækkes samme indgang i $A$, $a_{i,j}$. Da \\ $a_{i,j} - a_{i,j} = 0$, må $A + (-A) = O$. \\
(e) Her tages der udgangspunkt i (i,j)-indgangen i A, $a_{i,j}$. Det ses så, at udsagnet bliver til $(st)a_{i,j} = s(ta_{i,j})$ når flere tal multipliceres, er det lige meget i hvilken rækkefølge, jf. den associative lov. \\
(f) Hver indgang i A, $a_{i,j}$ lægges sammen med den tilsvarende indgang i B, $b_{i,j}$, dette giver venstresiden $s(a_{i,j}+b_{i,j})$. Da det er tilladt at multiplicere ind i en parantes, kan udtrykket også skrives $s(a_{i,j}+b_{i,j})=sa_{i,j}+sb_{i,j}$, i følge den distributive lov. \\
(g) På samme måde som før tages der udgangspunkt i (i,j)-indgangen i A, $a_{i,j}$. Udtrykket på venstresiden er så $(s+t)a_{i,j}$. Ligesom før, givet den distributive lov, kan der multipliceres ind i parantesen og dermed fås følgende, $(s+t)a_{i,j}=sa_{i,j}+ta_{i,j}$.
\end{proof}

\begin{eks}
For at vise eksempler på nogle af ovenstående regneoperationer, tages der udgangspunkt i matricerne $A$ og $B$, samt skalaren $s$, hvor $s(A+B)$ skal findes.
\begin{align*}
A= \begin{bmatrix}
	2 & 3 & 4 \\
	5 & -2 & 1 	
\end{bmatrix}, \quad
B= \begin{bmatrix}
	1 & 2 & -1 \\
	-3 & 4 & 0
\end{bmatrix}, \quad
s=5
\end{align*}
Først udføres regneoperation (a) fra Sætning \ref{stn_regn},
\begin{align*}
A+B= \begin{bmatrix}
	2 & 3 & 4 \\
	5 & -2 & 1 	
\end{bmatrix}  
+ \begin{bmatrix}
	1 & 2 & -1 \\
	-3 & 4 & 0
\end{bmatrix}
= \begin{bmatrix}
	3 & 5 & 3 \\
	2 & 2 & 1
\end{bmatrix}.
\end{align*}
Herfter udføres (f),
\begin{align*}
s(A+B)= 5 \times \begin{bmatrix}
	3 & 5 & 3 \\
	2 & 2 & 1
\end{bmatrix}
= \begin{bmatrix}
	15 & 25 & 15 \\
	10 & 10 & 5
\end{bmatrix}.
\end{align*}
\end{eks}


\subsection{Matrix produkt}
Multiplikation af to matricer gøres ikke ved at multiplicere de tilsvarende indgange, som er tilfældet ved multiplikation af vektorer. Et produkt af to matricer defineres herunder. 
\begin{defn} [Matrix produkt]
Lad $A$ være en $m \times n$ matrix og $B$ være en $n \times p$ matrix. Da er produktet $A \cdot B$ en $m \times p$ matrix, hvor indgangene er givet ved: 

$$(AB)_{i,j} = \vec{a}_i \cdot \vec{B}_j$$

hvor $\vec{a}_i$ er den $i$'te række i $A$, og $\vec{B}_j$ er den $j$'te søjle i $B$
\label{def:(matrixprodukt)}
\end{defn}
Det er vigtigt, at antallet af søjler i $A$ er det samme som antallet af rækker i $B$. Er det ikke tilfældet, så er matrix produktet ikke defineret. Matrix produktet er oftest ikke kommutativt. Ligningen $AB=BA$ er derfor ikke altid gældende. 
\begin{eks}
Nu tages udgangspunkt i to matricer $A$ og $B$. 
\begin{align*}
\underset{2 \times 3}{A}= \begin{bmatrix}
	\bf{1} & \bf{3} & \bf{-2} \\
	5 & 4 & 0 	
\end{bmatrix},
\underset{3 \times 4}{B}= \begin{bmatrix}
	\bf{2} & 3 & -1 & 3 \\
	\bf{1} & 4 & 5 & 5\\
	\bf{1} & 0 & 4 & 2
\end{bmatrix}  
\end{align*}
For at beregne indgang $(1,1)$ benyttes række $1$ i $A$ og søjle $1$ i $B$. 
$$ab_{1,1}=1\cdot 2+3\cdot 1-2 \cdot 1 = 3$$ 
De resterende indgange beregnes på samme måde. 
Matrix produktet af $A$ og $B$ er altså givet ved:
\begin{align*}
\underset{2 \times 4}{AB}= \begin{bmatrix}
	\bf{3} & 15 & 6 & 14 \\
	14 & 31 & 15 & 35
\end{bmatrix}  
\end{align*}
\end{eks}