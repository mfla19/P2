\section{Vektorer og matricer}

Lineære ligningssystemer kan opskrives i matricer. 
En matrix er defineret i Definition \ref{def:matricer}.

\begin{defn}[Matrix]
En $m \times n$ matrix er en rektangulær tabel over skalarer med $m$ rækker og $n$ søjler. 
Hvis $m=n$ er matricen kvadratisk. 
Skalaren i $i$'te række og $j$'te søjle kaldes $(i,j)$-indgangen.
\label{def:matricer}
\end{defn}

En vektor er en matrix, der enten kun har en søjle eller en række. 
Vektorer kan også repræsenteres geometrisk, hvor en vektor er en pil med en retning og en længde.
Det vil sige, at en vektor er et særtilfælde af en matrix, og alle regneregler for matricer, gør sig derfor også gældende for vektorer, men ikke alle regneregler for vektorer kan bruges på matricer.
Vektorer anvendes blandt andet til at repræsentere rækker og søjler i matricer. 
Indgangene i en vektor kaldes vektorens vektorkomponenter, og svarer til vektorens koordinater i den geometriske repræsentation.

\begin{defn}[Vektor]
En \textbf{vektor} er en matrix med enten kun en række eller en søjle. \\
En \textbf{rækkevektor} $\vec{v}$ er en matrix med kun en række og med indgange $v_1,v_2,...,v_n$.\\
En \textbf{søjlevektor} $\vec{u}$ er en matrix med kun en søjle og med indgange $u_1,u_2,...,u_m$.\\
\end{defn}

Herunder ses en $m \times n$ matrix $A$ og dens indgange, $a_{i,j}$. Den $i$'te række i A betenges $\vec{a}_i$ og er en rækkevektor, mens den $j$'te søjle angives $\vec{A}_j$ og er en søjlevektor.

\begin{align*}
A = \begin{bmatrix}
	a_{1,1} & a_{1,2} & \dots & a_{1,j} & \dots & a_{1,n} \\
	a_{2,1} & \ddots  &       &         &       & \vdots \\
	\vdots  &         & \ddots &        &       & \vdots \\
	a_{i,1} &         &       & a_{i,j} &       & \vdots \\
	\vdots  &         &       &         & \ddots& \vdots \\
	a_{m,1} & \dots   & \dots & \dots   & \dots & a_{m,n} 
\end{bmatrix}
=
\begin{bmatrix}
\vec{a}_1\\
\vec{a}_2\\
\vdots\\
\vec{a}_i\\
\vdots\\
\vec{a}_m
\end{bmatrix}
=
\begin{bmatrix}
\vec{A}_1	& \vec{A}_2	& \dots	& \vec{A}_j & \dots	& \vec{A}_n
\end{bmatrix}
\end{align*}



Matricer kan bruges til at illustrere forskellige ting fra den virkelige verden. Det kan være antal varer solgt fra forskellige butikker eller prisen på forskellige produkter. 

\begin{eks}
Herunder ses en $4 \times 3$ matrix, der viser antallet af solgte varer fra tre forskellige butikker. Hver række i matricen repræsenterer en bestemt vare, mens hver søjle repræsenterer en butik. Den første række viser altså hvor mange trøjer hver butik har solgt.
\begin{align*}
\begin{matrix}
	Trøjer \\
	Kjoler \\
	Bukser \\
	Jakker
\end{matrix}
\begin{bmatrix}
	24 & 35 & 43 \\
	10 & 47 & 24 \\
	33 & 25 & 32 \\
	28 & 51 & 37
\end{bmatrix}
\end{align*}
I matricens $(2,3)$-indgang står der $24$, hvilket betyder, at den tredje butik har solgt $24$ kjoler. I $(3,1)$-indgangen står $33$, så den første butik har solgt $33$ par bukser. 
\end{eks}

\begin{defn}[Transponeret matrix]
Lad $A$ være en $m \times n$ matrix. Så er den transponerede matrix, $A^T$, en $n \times m$ matrix, hvor hver indgang $(j,i)$ i $A^T$, er den $(i,j)$'te indgang i $A$
\label{def:(transmatrix)} 
\end{defn}
Det vil sige, at rækkerne i $A$, bliver til søjlerne i $A^T$, og søjlerne i $A$ bliver til rækkerne i $A^T$.

\begin{eks}
Givet en matrix $A$ bestemmes nu den transponerede $A^T$
\begin{align*}
A = \begin{bmatrix}
	5 & 3 & 3 \\
	1 & 2 & 5
\end{bmatrix}
\end{align*}

\begin{align*}
A^T = \begin{bmatrix}
	5 & 1  \\
	3 & 2  \\
	3 & 5
\end{bmatrix}
\end{align*}
Det ses at $A$ er en $2 \times 3$ matrix og $A^T$ er en $3 \times 2$ matrix. 
\end{eks}


Standard basisvektorer er vektorer, hvis indgange alle er $0$ pånær én indgang, der har værdien $1$.

\begin{defn}[Standard basisvektorer]
Lad $\vec{e}_{i}\in\mathds{R}^n$ være en \textbf{standard basisvektor}. Da vil den $j$'te indgang være givet ved
\begin{align*}
e_{i_{j}}=\begin{cases} 0, \quad j \neq i
\\ 1 , \quad j = i \end{cases}.
\end{align*}

\end{defn}

En bestemt type matrix er en identitetsmatrix, der betegnes $I_n$ og dannes ud fra standard basisvektorer. 

\begin{defn} [Identitetsmatrix]
For hvert positivt heltal $n$, er textbf{identitetsmatricen} $I_n$ en $n \times n$ matrix, defineret som\\ $I_n = \rvect{\vec{e_1} & \vec{e_2} & \dots &  \vec{e_n}}$, hvor $\vec{e_1}$, $\vec{e_2}$, $\dots$, $\vec{e_n}$ er standardvektorerne i $\mathds{R}^n$
\label{def:imatrix}
\end{defn}

En identitetsmatrix med $3$ rækker og $3$ søjler, ser således ud:
\begin{align*}
I_3 = \begin{bmatrix}
	1 & 0 & 0 \\
	0 & 1 & 0 \\
	0 & 0 & 1 
\end{bmatrix}
\end{align*}

%En anden bestemt type matrix er en delmatrix, som består af en del af en anden matrix.
%\begin{defn} [Delmatrix]
%En matrix $A'$ er en delmatrix af $A$, hvis $A'$ kan dannes ved at fjerne hele søjler og/eller hele rækker fra $A$.
%\label{delmatrix}
%\end{defn}
%
%Givet en matrix $A$ kan der dannes delmatricen $A'$.
%
%\begin{align*}
%A = \begin{bmatrix}
%	1 & 2 & 3 \\
%	4 & 5 & 6 \\
%	7 & 8 & 9 
%\end{bmatrix}, \quad \quad
%A' = \begin{bmatrix}
%	5 & 6 \\
%	8 & 9
%\end{bmatrix}
%\end{align*}
%
