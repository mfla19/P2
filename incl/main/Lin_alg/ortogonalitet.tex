\subsection{Ortogonal og Ortonormal basis}
To sær tilfælde af basis er ortogonal basis og en ortonormal basis.
\begin{defn}[Ortogonalog Ortonormal basis]
En basis $B = \{\vec{v}_i,..., \vec{v}_k\}$ for  et underrum $W$ til $\mathds{R}^n$ kaldes en \textbf{ortogonal basis}, hvis 
$\vec{v}_i \bot \vec{v}_j \, \forall \vec{v}_i, \vec{v}_j \in B, \, i \neq j$. En ortogonal basis er \textbf{ortonormal} hvis $\Vert\vec{v}_i\Vert = 1 \, \forall \vec{v}_i \in B$.
\end{defn}
For at vise, at en basis er ortogonal, er det nødvendigt at vise, at alle vektorer er indbyrdes ortogonale, her kan gøres brug af deres indbyrdes vektorprodukt.
\begin{lma}
Lad $\vec{v}, \vec{u} \in \mathds{R}^n$, og $\vec{v}\neq \vec{0}$, $\vec{u} \neq \vec{0}$ da er $\vec{v} \bot \vec{u}$ hvis og kun vis $\vec{v}^T\vec{u} = 0$.
\label{lma:vinkelret}
\end{lma}
\begin{proof}
For at vise Lemma \ref{lma:vinkelret}, vises først, at $\vec{v}^T\vec{u} = \Vert\vec{v}\Vert\Vert\vec{u}\Vert\cos{\theta}$, hvor $\theta$ er vinklen mellem de to vektorer. 
Betragt derfor trekanten med siderne $\Vert\vec{v}\Vert , \Vert\vec{u}\Vert$ og $\Vert\vec{v-u}\Vert$, da følger det af cosinusrelationerne, at 
\begin{align*}
\Vert\vec{v}\Vert^2 +  \Vert\vec{u}\Vert^2 - 2\Vert\vec{v}\Vert\Vert\vec{u}\Vert\cos{\theta} &= \Vert\vec{v-u}\Vert^2 
\\ &= (\vec{v}-\vec{u})^T(\vec{v}-\vec{u})
\\&= \sum_{i=1}^n (v_i- u_i)^2 
\\&= \sum_{i=1}^n v_i^2 + u_i^2 - 2 v_iu_i 
\\&= \Vert\vec{v}\Vert^2 +  \Vert\vec{u}\Vert^2 - 2\vec{v}^T\vec{u}.
\end{align*}
Dermed kan det konkluderes,  at $\vec{v}^T\vec{u} = \Vert\vec{v}\Vert\Vert\vec{u}\Vert\cos{\theta}$.
\\Antag nu, at $\vec{v}$ og $\vec{u}$ er ortogonale, da vil $\theta= \pi/2$, hvormed
\begin{align*}
\vec{v}^T\vec{u} = \Vert\vec{v}\Vert\Vert\vec{u}\Vert\cos{\pi/2}= \Vert\vec{v}\Vert\Vert\vec{u}\Vert\cdot0 = 0.
\end{align*}
Dermed er vektorproduktet nul når vektorene er ortogonale.
Antag til sidst at $\vec{v}^T\vec{u}=0$, da vil 
\begin{align*}
0 = \vec{v}^T\vec{u} = \Vert\vec{v}\Vert\Vert\vec{u}\Vert\cos{\theta}
\end{align*}
Da hverken $\vec{v}$ eller $\vec{u}$ er lig $\vec{0}$ må $\cos{\theta} = 0 $, hvorfor det kan konkluderes, at de må være ortogonale.
\end{proof}
Her af følger, at to ortogonale vektorer er lineært uafhængige.
\begin{lma}
Lad $\vec{u}, \vec{v} \in \mathds{R}^n$, så $\vec{v} \neq \vec{0}$ og $\vec{u} \neq \vec{0}$, da er $\vec{v}$ og $\vec{u}$ lineært uafhængige hvis $\vec{v} \bot \vec{u}$.
\label{lma:ortolinuaf}
\end{lma}
\begin{proof}
Lad $\vec{v}, \vec{u}$ være ortogonale, og antag for modstrid, at de også er lineært afhængige.
Da gælder at $\vec{v} = \lambda \vec{u}$, for en skalar $\lambda$.
Da $\vec{v}$ og $\vec{u}$ er ortogonale, følger det af Lemma \ref{lma:vinkelret}, at 
\begin{align*}
\vec{v}^T\vec{u} = \lambda\vec{u}^T\vec{u} = \lambda \Vert \vec{u} \Vert^2 = 0.
\end{align*}
Hvis $\lambda = 0$ vil $\vec{v}= \vec{0}$, og hvis $\Vert \vec{u} \Vert = 0$ vil $\vec{u} = \vec{0}$, hvorfor der er opstået modstrid, derfor er $\vec{v}$ og $\vec{u}$ lineært uafhængige.
\end{proof}
Det medfører, at enhver mængde af ortogonale vektorer, som udspænder et underrum, i følge Definition \ref{def:basis} er
en basis for underrummet.
Er der derimod tale om en ikke ortogonal basis, kan den omdannes til sådanne ved brug af Gram-Schmidt processen.
\begin{stn}[Gram-Schmidt processen]
Lad $\{\vec{u}_1, ..., \vec{u}_k\}$ være en basis for underrummet $W$ til $\mathds{R}^n$, og lad 
$\vec{v}_1 = \vec{u}_1$, og $\vec{v}_i= \vec{u}_i - \sum_{j=1}^{i-1} \frac{\vec{u}_i^T \vec{v}_j}{\Vert\vec{v}_j\Vert}\vec{v}_j$ for $i \geq 2$.
Da udgør $B = \{\vec{v}_1,..., \vec{v}_k\}$ en ortogonal basis for $W$.
\label{stn:gram}
\end{stn}
\begin{proof}
For at  vise, at $B$ er en ortogonal basis for $W$, laves et induktionsbevis, lad derfor først $k= 1$, da $\vec{u}_1\neq \vec{0}$ og udgør en basis for $W_1$, så må $\vec{v}_1 = \vec{u}_1$ nødvendigvis også gøre det.
Da $B$ kun består af en vektor, må den være ortogonal.
\\Antag, at $B$ er en ortogonal basis for $W_n$ for $k=n$, da vises det, at det medføre, at $B$ også er en ortogonal basis for  $W_{n+1}$ når $k = n+1$.
\\Betragt derfor elementet 
\begin{align*}
\vec{v}_{n+1} = \vec{u}_{n+1} - \sum_{i=1}^n \frac{\vec{u}_{n+1}^T\vec{v}_i}{\Vert\vec{v}_i\Vert^2}\vec{v}_i.
\end{align*}
Da $\vec{v}_1,...,\vec{v}_n$ er antaget til at være ortogonale, medfører det at
\begin{align*}
\vec{v}_j^T\vec{v}_{n+1} &= \vec{v}_j^T\vec{u}_{n+1} - \sum_{i=1}^n \frac{\vec{u}_{n+1}^T\vec{v}_i}{\Vert\vec{v}_i\Vert^2}\vec{v}_j^T\vec{v}_i
\\ &= \vec{v}_j^T\vec{u}_{n+1} - \frac{\vec{u}_{n+1}^T\vec{v}_j}{\Vert\vec{v}_j\Vert^2}\vec{v}_j^T\vec{v}_j
\\ & = \vec{v}_j^T\vec{u}_{n+1} -  \frac{\vec{u}_{n+1}^T\vec{v}_j}{\Vert\vec{v}_j\Vert^2}\Vert\vec{v}_j\Vert^2
\\ & =\vec{v}_j^T\vec{u}_{n+1} - \vec{v}_j^T\vec{u}_{n+1} = 0 , 
\end{align*}
for $j = 1,...,n$. 
Derfor kan det konkluderes at $\vec{v}_{n+1}$ er ortogonal med $\vec{v}_1,...,\vec{v}_n$, hvormed det følger af Lemma \ref{lma:ortolinuaf}, at $\vec{v}_1,...,\vec{v}_n, \vec{v}_{n+1}$ er lineær uafhængige.
Da $\vec{v}_1,...,\vec{v}_n$ er en basis for $W_n$, må $\vec{v}_1,...,\vec{v}_n, \vec{u}_{n+1}$ være en basis for $W_{n+1}$, fordi $span(\vec{v}_1,...,\vec{v}_n) = span(\vec{u}_1,...,\vec{u}_n) = W_n$ pr. induktionsantagelsen. 
Da $\vec{v}_{n+1}$ er lineær afhængig af $\vec{u}_{n+1}$ må $span(\vec{v}_1,...,\vec{v}_n, \vec{v}_{n+1}) = span(\vec{v}_1,...,\vec{v}_n, \vec{u}_{n+1})$, og $\vec{v}_{n+1} \neq \vec{0}$ da $\vec{u}_{n+1} \neq \vec{0}$, hvorfor $\vec{v}_1,...,\vec{v}_n, \vec{v}_{n+1}$ udgør en ortogonal basis for $W$.
\end{proof}
Derfor følger det, at et hvert underrum har en ortogonal basis.
\begin{kor}
Et hvert underrum $W$ til $\mathds{R}^n$ har en ortogonal og ortonormal basis.
\end{kor}
\begin{proof}
Af Sætning \ref{stn:gram} fremgår det, hvordan enhver basis kan konstrueres til en ortogonal basis. 
For at konstruere en ortonomal basis, betragt et vilkårligt element $\vec{v}_i$ fra en ortogonal basis, og definer vektoren $\vec{u}_i = \frac{1}{\Vert\vec{v}_i\Vert}\vec{v}_i$, da vil $\{\vec{u}_1,...,\vec{u}_n\}$ udgøre en ortonormal basis. 
Da $\Vert\vec{u}_i\Vert^2 = \frac{1}{\Vert\vec{v}_i\Vert^2}\Vert\vec{v}_i\Vert^2 = 1$.
\end{proof}

\subsubsection{Ortogonal komplement}
Ortogonalitet kan også bruges til at generere en ny delmængde af $\mathds{R}^n$ ud fra en given delmængde.
\begin{defn}[Ortogonal komplement]
En delmængde vektorer $S^{\bot} \subseteq \mathds{R}^n$ kaldes det \textbf{ortogonale komplement} til delmængden $S \subseteq \mathds{R}^n$, hvis 
\begin{align*}
	S^{\bot} = \{\vec{v} \in \mathds{R}^n \mid \vec{v}^T\vec{u} = 0, \, \forall \vec{u} \in S\}
\end{align*}
\label{def:ortokom}
\end{defn}
Det ortogonale komplement indeholder dermed alle de elementer, der er ortogonale til mængden.
\begin{prop}
Hvis $\vec{v} \in S$ og $\vec{v} \in S^{\bot}$, så er  $\vec{v}=\vec{0}$.
\label{prop:nulortokomp}
\end{prop}
\begin{proof}
Hvis $\vec{v} \in S$ og $\vec{v} \in S^{\bot}$ så følger det af Definition \ref{def:ortokom} at $\vec{v}^T\vec{v} = 0$.
Dermed så $\vec{v}^T\vec{v} = \sum_{i=1}^n v_i^2 =0$, hvilket kun er muligt, hvis $v_i = 0$ for ethvert $i = 1,..., n$, hvorfor $\vec{v}=\vec{0}$.
\end{proof}
Dermed indeholder både mængden og dets ortogonale komplement nulvektoren, da nulvektoren er ortogonal til enhver vektor i følge Lemma \ref{lma:vinkelret}. 
Det resultere i, at det ortogonale komplement aldrig kan være lig den tommemængde.
En mængde og dens ortogonale komplement har den egenskab at de tilsammen udspænder hele $\mathds{R}^n$.
\begin{stn}[$\mathds{R}^n$ udspændt af $W$ og $W^{\bot}$]
Lad $W$ være et underrum til $\mathds{R}^n$, da eksistere der entydige vektorer  $\vec{w} \in W$ og $\vec{z} \in W^{\bot}$ så $\vec{u}= \vec{w}+\vec{z}$ for en hver $\vec{u} \in \mathds{R}^n$.
\label{stn:Rnorto}
\end{stn}
\begin{proof}
Lad $\{\vec{v}_1,...,\vec{v}_k\}$ betegne en ortonormal basis for $W$, og betragt en vilkårlig $\vec{u} \in \mathds{R}^n$. 
Lad $w \in W$ være det element, der er en linear kombination af $\{\vec{v}_1,...,\vec{v}_k\}$ med skalare $\lambda_i = \vec{u}^T\vec{v}_i$, da følger det, at
\begin{align*}
\vec{v}_j^T\vec{w} &= \vec{v}_j^T\sum_{i=0}^k(\vec{u}^T\vec{v}_i)\vec{v}_i
\\ & = (\vec{u}^T\vec{v}_j)\vec{v}_j^T\vec{v}_j = \vec{u}^T\vec{v}_j \Vert \vec{v}_j \Vert^2 = \vec{u}^T\vec{v}_j,
\end{align*}
for $j=1,...,k$, da $\{\vec{v}_1,...,\vec{v}_k\}$ er en ortonormal basis.
\\ Lad nu $\vec{z} = \vec{u}- \vec{w}$, da $\vec{v}_j^T\vec{w}= \vec{v}_j^T\vec{u}$ for et hvert $j=1,...,k$ gælder at 
\begin{align*}
\vec{v}_j^T\vec{z} = \vec{v}_j^T\vec{u}- \vec{v}_j^T\vec{w} = 0,
\end{align*}
derfor må $\vec{z} \in W^{\bot}$. 
Då $\vec{u}$ er valgt vilkårligt, medfører det, at der eksisterer vektorer $\vec{w} \in W$ og $\vec{z} \in W^{\bot}$ så ethvert $\vec{u} \in \mathds{R}^n$ så $\vec{u}= \vec{w}+\vec{z}$.
\\Så vises, at $\vec{w}$ og $\vec{z}$ er entydige.
Lad derfor $\vec{w}' \in W$ og $\vec{z}' \in W^{\bot}$ så $\vec{u}= \vec{w}' + \vec{z}'$.
Da må $\vec{w} + \vec{z} = \vec{w}' + \vec{z}'$, hvorfor $\vec{w}-\vec{w}' = \vec{z}-\vec{z}'$.
Det følger derfor, at $\vec{w}-\vec{w}', \vec{z}-\vec{z}' \in W$ og $\vec{w}-\vec{w}', \vec{z}-\vec{z}' \in W^{\bot}$, hvilket, ifølge Proportion \ref{prop:nulortokomp}, betyder, at $\vec{w}-\vec{w}' = \vec{z}-\vec{z}' = \vec{0}$, hvormed $\vec{w}= \vec{w}'$ og $\vec{z}=\vec{z}'$ og entydigheden er bevist.
\end{proof}
Derfor vil, foreningsmængden af en delmængde af $\mathds{R}^n$ og dets ortogonale komplements baser udgøre en basis for $\mathds{R}^n$.
\begin{kor}
Lad $B$  og $B^{\bot}$ udgøre en basis for henholdsvis underrummet $W$ til $\mathds{R}^n$, og dets ortogonale komplement $W^{\bot}$, da vil $\{B, B^{\bot}\}$ være en basis for $\mathds{R}^n$.
\label{kor:basisRnorto}
\end{kor}
\begin{proof}
Det følger af Sætning \ref{stn:Rnorto}, at et hvert element i $\mathds{R}^n$ kan skrives som en linear kombination af af et element fra $W$ og et fra $W^{\bot}$, derfor må $span\{B, B^{\bot}\} = \mathds{R}^n$. 
Da $B$ og $B^{\bot}$ er lineært uafhængige i følge Lemma \ref{lma:ortolinuaf} følger det af Korollar \ref{kor:serbase}, at $\{B, B^{\bot}\}$ udgør en basis for $\mathds{R}^n$.
\end{proof}
\begin{bem}
Det følger derfor af \ref{kor:basisRnorto}, at
\begin{align}
\dim{\mathds{R}^n} = \dim{span\{B, B^{\bot}\}} = \dim{W} + \dim{W^{\bot}} = n.
\end{align}
\end{bem}
 

 