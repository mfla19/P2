\subsection{Ortogonal og Ortonormal basis}
Et speciel tilfælde af en basis er en ortogonal basis og en ortonormal basis.
\begin{defn}
En basis $B = \{\vec{v}_i,..., \vec{v}_k\}$ for  et underrum $W$ til $\mathds{R}^n$ kaldes en \textbf{ortogonal basis}, hvis 
$\vec{v}_i \bot \vec{v}_j \, \forall \vec{v}_i, \vec{v}_j \in B, \, i \neq j$.
\\En ortogonal basis er \textbf{ortonormal} hvis $\Vert\vec{v}_i\Vert = 1 \, \forall \vec{v}_i \in B$.
\end{defn}
En måde at tjekke om to vektore er ortogonale, er ved beregne deres vektor produkt.
\begin{lma}
Lad $\vec{v}, \vec{u} \in \mathds{R}^n$, og $\vec{v}\neq \vec{0}$, $\vec{u} \neq \vec{0}$ da er $\vec{v} \bot \vec{u}$ hvis og kun vis $\vec{v}^T\vec{u} = 0$.
\label{lma:vinkelret}
\end{lma}
\begin{proof}
For at vise Lemma \ref{lma:vinkelret}, vises først at $\vec{v}^T\vec{u} = \Vert\vec{v}\Vert\Vert\vec{u}\Vert\cos{\theta}$, hvor $\theta$ er vinklen mellem de to vektorer. 
Betragt derfor trekanten med siderne $\Vert\vec{v}\Vert , \Vert\vec{u}\Vert$ og $\Vert\vec{v-u}\Vert$, da følger det af cosinusrelationerne, at 
\begin{align*}
\Vert\vec{v}\Vert^2 +  \Vert\vec{u}\Vert^2 - 2\Vert\vec{v}\Vert\Vert\vec{u}\Vert\cos{\theta} &= \Vert\vec{v-u}\Vert^2 
\\ &= \vec{v-u}^T\vec{v-u} 
\\&= \sum_{i=1}^n (v_i- u_i)^2 
\\&= \sum_{i=1}^n v_i^2 + u_i^2 - 2 v_iu_i 
\\&= \Vert\vec{v}\Vert^2 +  \Vert\vec{u}\Vert^2 - 2\vec{v}^T\vec{u}.
\end{align*}
Dermed kan det konkluderes,  at $\vec{v}^T\vec{u} = \Vert\vec{v}\Vert\Vert\vec{u}\Vert\cos{\theta}$.
\\Antag nu at $\vec{v}$ og $\vec{u}$ er ortogonale, da vil $\theta= \pi/2$ hvorfor at 
\begin{align*}
\vec{v}^T\vec{u} = \Vert\vec{v}\Vert\Vert\vec{u}\Vert\cos{\pi/2}= \Vert\vec{v}\Vert\Vert\vec{u}\Vert\cdot0 = 0.
\end{align*}
Dermed er vektorproduktet nul når vektorene er ortogonale.
Antag til sidst at $\vec{v}^T\vec{u}=0$, da vil 
\begin{align*}
0 = \vec{v}^T\vec{u} = \Vert\vec{v}\Vert\Vert\vec{u}\Vert\cos{\theta}
\end{align*}
Da hverken $\vec{v}$ eller $\vec{u}$ er lig $\vec{0}$ må $\cos{\theta} = 0 $, hvorfor det kan konkluderes at de må være ortogonale.
\end{proof}
Det kan benyttes til at vise at to vektorer der er ortogonale også er lineært uafhængige.
\begin{lma}
Lad $\vec{u}, \vec{v} \in \mathds{R}^n$, så $\vec{v} \neq \vec{0}$ og $\vec{u} \neq \vec{0}$, da er $\vec{v}$ og $\vec{u}$ lineært uafhængige hvis $\vec{v} \bot \vec{u}$.
\label{lma:ortolinuaf}
\end{lma}
\begin{proof}
Lad $\vec{v}, \vec{u}$ være ortogonale, og antag for modstrid at de også er lineært afhængige.
Da gælder at $\vec{v} = \lambda \vec{u}$, for en skalar $\lambda$.
Da $\vec{v}$ og $\vec{u}$ er ortogonale følger det af Lemma \ref{lma:vinkelret}, at 
\begin{align*}
\vec{v}^T\vec{u} = \lambda\vec{u}^T\vec{u} = \lambda \Vert \vec{u} \Vert^2 = 0.
\end{align*}
Hvis $\lambda = 0$ vil $\vec{v}= \vec{0}$, og hvis $\Vert \vec{u} \Vert = 0$ vil $\vec{u} = \vec{0}$, hvorfor der er opstået modstrid, hvorfor at $\vec{v}$ og $\vec{u}$ er lineær uafhængige.
\end{proof}
Det betyder også at en hver mængde af ortogonale vektorer som udspænder et underrum, er i følge Sætning 
en basis for det underrum.
Det viser sig at en hver basis kan omskrives til en ortogonal basis ved brug af Gram-Schmidt processen.
\begin{stn}[Gram-Schmidt processen]
Lad $\{\vec{u}_1, ..., \vec{u}_k\}$ være en basis for underrummet $W$ til $\mathds{R}^n$, og lad 
$\vec{v}_1 = \vec{u}_1$, og $\vec{v}_i= \vec{u}_i - \sum_{j=1}^{i-1} \frac{\vec{u}_i^T \vec{v}_j}{\Vert\vec{v}_j\Vert}\vec{v}_j$ for $i \geq 2$.
Da udgør $B = \{\vec{v}_1,..., \vec{v}_k\}$ en ortogonal basis for $W$.
\label{stn:gram}
\end{stn}
\begin{proof}
For at $B$ er en ortogonal basis for $W$ laves et induktionsbevis, lad derfor først $k= 1$, da $\vec{u}_1\neq \vec{0}$ og udgør en basis for $W$, så må $\vec{v}_1 = \vec{u}_1$ nødvendigvis også gøre det.
Da $B$ kun består af en vektor, må den være ortogonal.
\\Antag, at $B$ er en ortogonal basis for $W$ for $k=n$, da vises det at det medføre at $B$ også er en ortogonal basis for  $W$ når $k = n+1$.
\\Betragt derfor elementet 
\begin{align*}
\vec{v}_{n+1} = \vec{u}_{n+1} - \sum_{i=1}^n \frac{\vec{u}_{n+1}^T\vec{v}_i}{\Vert\vec{v}_i\Vert^2}\vec{v}_i.
\end{align*}
Da $\vec{v}_1,...,\vec{v}_n$ er antaget til at være ortogonale, medfører det at
\begin{align*}
\vec{v}_j^T\vec{v}_{n+1} &= \vec{v}_j^T\vec{u}_{n+1} - \sum_{i=1}^n \frac{\vec{u}_{n+1}^T\vec{v}_i}{\Vert\vec{v}_i\Vert^2}\vec{v}_j^T\vec{v}_i
\\ &= \vec{v}_j^T\vec{u}_{n+1} - \frac{\vec{u}_{n+1}^T\vec{v}_j}{\Vert\vec{v}_j\Vert^2}\vec{v}_j^T\vec{v}_j
\\ & = \vec{v}_j^T\vec{u}_{n+1} -  \frac{\vec{u}_{n+1}^T\vec{v}_j}{\Vert\vec{v}_j\Vert^2}\Vert\vec{v}_j\Vert^2
\\ & =\vec{v}_j^T\vec{u}_{n+1} - \vec{v}_j^T\vec{u}_{n+1} = 0 , 
\end{align*}
for $j = 1,...,n$. 
Derfor kan det konkluderes at $\vec{v}_{n+1}$ er ortogonal med $\vec{v}_1,...,\vec{v}_n$, derfor følger det af Lemma \ref{lma:ortolinuaf}, at $\vec{v}_1,...,\vec{v}_n, \vec{v}_{n+1}$ er lineær uafhængige.
Da $\vec{v}_{n+1}$ er lineær afhængig af $\vec{u}_{n+1}$ må $span\{\vec{v}_1,...,\vec{v}_n, \vec{v}_{n+1}\} = span\{\vec{v}_1,...,\vec{v}_n, \vec{u}_{n+1}\}$.
Derfor må $\vec{v}_{n+1} \neq \vec{0}$ da $\vec{u}_{n+1} \neq \vec{0}$, og $\vec{v}_1,...,\vec{v}_n, \vec{v}_{n+1}$ må udgøre en ortogonal basis for $W$.
\end{proof}
Derfor må det nødvendigvis følge at et hvert underrum har en ortogonal basis.
\begin{kor}
Et hvert underrum $W$ til $\mathds{R}^n$ har en ortogonal og ortonormal basis.
\end{kor}
\begin{proof}
Af Sætning \ref{stn:gram} fremgår det hvordan enhver basis kan konstrueres til en ortogonal basis. 
For at konstruere en ortonomal basis, betragt et vilkårligt element $\vec{v}_i$ fra en ortogonal basis, og definer vektoren $\vec{u}_i = \frac{1}{\Vert\vec{v}_i\Vert}\vec{v}_i$, da vil $\{\vec{u}_1,...,\vec{u}_n\}$ udgøre en ortonormal basis. 
\end{proof}
Til en hver ortogonal mængde er der en tilhørende ortogonal komplementær mængde.
\begin{defn}
En delmængde vektorer $S^{\bot} \subseteq \mathds{R}^n$ kaldes det \textbf{ortogonale komplement} til delmængden $S \subseteq \mathds{R}^n$, hvis 
\begin{align*}
	S^{\bot} = \{\vec{v} \in \mathds{R}^n \mid \vec{v}^T\vec{u} = 0, \, \forall \vec{u} \in S\}
\end{align*}
\label{def:ortokom}
\end{defn}
Den ortogonale komplementær mængde indeholder alle de elementer som der er ortogonale til den originale mængde
\begin{prop}
Hvis $\vec{v} \in S$ og $\vec{v} \in S^{\bot}$ så er  $\vec{v}=\vec{0}$.
\label{prop:nulortokomp}
\end{prop}
\begin{proof}
Hvis $\vec{v} \in S$ og $\vec{v} \in S^{\bot}$ så følger det af Definition \ref{def:ortokom} at $\vec{v}^T\vec{v} = 0$.
Dermed så $\vec{v}^T\vec{v} = \sum_{i=1}^n v_i^2 =0$, hvilket kun er muligt, hvis $v_i = 0$ for ethvert $i = 1,..., n$, hvorfor at $\vec{v}=\vec{0}$.
\end{proof}
Dermed indeholder både den originale mængde og den ortogonale komplementær mængde nulvektoren, og den ortogonale komplementær mængde 
adskiller sig derfor fra den klassiske komplementær mængde, hvis fællesmængde mængde med original mængden er den tomme mængde.
Men de mængden og dens ortogonale komplementær mængde kan stadig udspænde hele $\mathds{R}^n$.
\begin{stn}
Lad $W$ være et underrum til $\mathds{R}^n$, da for et hvert $\vec{u} \in \mathds{R}^n$ eksistere der unikke vektorer $\vec{w} \in W$ og $\vec{z} \in W^{\bot}$ så $\vec{u}= \vec{w}+\vec{z}$.
\label{stn:Rnorto}
\end{stn}
\begin{proof}
Lad $\{\vec{v}_1,...,\vec{v}_k\}$ betegne en ortonormal basis for $W$, og lad $w \in W$ være det element, der er en linear kombination af $\{\vec{v}_1,...,\vec{v}_k\}$ med skalare $\lambda_i = \vec{u}^T\vec{v}_i$ for et vilkårligt $\vec{u} \in \mathds{R}^n$, da følger det, at
\begin{align*}
\vec{v}_j^T\vec{w} &= \vec{v}_j^T\sum_{i=0}^k(\vec{u}^T\vec{v}_i)\vec{v}_i
\\ & = (\vec{u}^T\vec{v}_j)\vec{v}_j^T\vec{v}_j = \vec{u}^T\vec{v}_j \Vert \vec{v}_j \Vert^2 = \vec{u}^T\vec{v}_j,
\end{align*}
for $j=1,...,k$, da $\{\vec{v}_1,...,\vec{v}_k\}$ er en ortonormal basis.
\\ Lad nu $\vec{z} = \vec{u}- \vec{w}$, da $\vec{v}_j\vec{w}= \vec{v}_j\vec{u}$ for et hvert $j=1,...,k$ gælder at 
\begin{align*}
\vec{v}_j\vec{z} = \vec{v}_j\vec{u}- \vec{v}_j\vec{w} = 0
\end{align*}
hvorfor at $\vec{z} \in W^{\bot}$. 
Då $\vec{u}$ er valgt vilkårligt, må der eksistre vektorer $\vec{w} \in W$ og $\vec{z} \in W^{\bot}$ så ethvert $\vec{u} \in \mathds{R}^n$ så $\vec{u}= \vec{w}+\vec{z}$.
\\Så vises at $\vec{w}$ og $\vec{z}$ er entydige.
Lad derfor $\vec{w}' \in W$ og $\vec{z}' \in W^{\bot}$ så $\vec{u}= \vec{w}' + \vec{z}'$.
Da må $\vec{w} + \vec{z} = \vec{w}' + \vec{z}'$, hvorfor at $\vec{w}-\vec{w}' = \vec{z}-\vec{z}'$.
Det følger derfor at $\vec{w}-\vec{w}', \vec{z}-\vec{z}' \in W$ og $\vec{w}-\vec{w}', \vec{z}-\vec{z}' \in W^{\bot}$, hvorfor det følger af Proportion \ref{prop:nulortokomp} at $\vec{w}-\vec{w}' = \vec{z}-\vec{z}' = \vec{0}$, hvorfor at $\vec{w}= \vec{w}'$ og $\vec{z}=\vec{z}'$ og entydigheden er bevist.
\end{proof}
Det må derfor følge at forenigsmængden af basen for mængden og den ortogonale komplment er en basis for $\mathds{R}^n$
\begin{kor}
Lad $B$  og $B^{\bot}$ udgøre en basis for henholdsvis underrummet $W$ til $\mathds{R}^n$, og dets ortogonale komplement $W^{\bot}$, da vil $\{B, B^{\bot}\}$ være en basis for $\mathds{R}^n$.
\label{kor:basisRnorto}
\end{kor}
\begin{proof}
Det følger af Sætning \ref{stn:Rnorto} at et hvert element i $\mathds{R}^n$ kan skrives som en linear kombination af af et element fra $W$ og et fra $W^{\bot}$ derfor må $span\{B, B^{\bot}\} = \mathds{R}^n$. 
Da $B$ og $B^{bot}$ er lineært uafhængige i følge Lemma \ref{lma:ortolinuaf} følger det af Sætning 
at $\{B, B^{\bot}\}$ udgør en basis for $\mathds{R}^n$.
\end{proof}
Derfor må summen af dimentionen af de to underrum også nødvendigvis være lig $n$.
\begin{kor}
Lad $W$ være et underrum til $\mathds{R}^n$, da er $\dim {W} + \dim {W^{\bot}} = n$.
\end{kor}
\begin{proof}
Af Korolar \ref{kor:basisRnorto} følger at $\{B, B^{\bot}\}$ er en basis for $\mathds{R}^n$, hvorfor at
\begin{align}
\dim{\mathds{R}^n} = \dim{span\{B, B^{\bot}\}} = \dim{W} + \dim{W^{\bot}}.
\end{align}
\end{proof}
 

 