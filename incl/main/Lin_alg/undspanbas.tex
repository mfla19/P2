\section{Underrum, Span og Basis}
Rummet $\mathds{R}^n$ er mængden af vektorer af dimension $n$, og på samme måde som med mængder generelt, er det muligt at betragte delmængder af $\mathds{R}^n$.
Et special tilfælde af disse delmængder er et underrum.
En delmængden af vektorer  er et underrum, hvis den indeholder nul vektoren, og er lukket under vektor addition og skalar multiplikation. 
\begin{defn}[Underrum]
Lad $W$ være en mængde af vektorer $\vec{v_1},...,\vec{v_k} \in \mathds{R}^n$, da er $W$  \textbf{underrum} til $\mathds{R}^n$, hvis:
\begin{enumerate}[label=\alph*]
\item $\vec{0} \in W$
\item $\vec{u}+\vec{v} \in W \quad \forall \vec{u}, \vec{v} \in W$
\item $c \cdot \vec{v} \in W \quad \forall \vec{v} \in W, \forall c \in \mathds{R}$
\end{enumerate}
\label{def:underrum}
\end{defn}
En måde at beskrive delmængderne af vektorer på, er ved at finde en mængde vektorer $S$, hvor alle andre vektorer i mængden er en lineær kombination af disse vektorer. 
Mængden $S$ siges at generer/udspænde delmængden, mens delmængden kaldes spannet af $S$.
\begin{defn}[Span]
Lad $S=\{\vec{v_1},...,\vec{v_k}\}$ være en ikke tom mængde af vektorer, hvor $\vec{v_i} \in \mathds{R}^n$ for $i = 1,..,k$. 
Da er \textbf{spannet af $S$} mængden af vektorer
\begin{align*}
span(S) = \{\vec{u}| \vec{u}=\sum_{i=0}^k c_i \vec{v_i}, \vec{v_i} \in S, c_i \in \mathds{R}\}.
\end{align*} 
\label{def:span}
\end{defn}
Bemærk at $\sum_{i=0}^k c_i \vec{v_i}$ er det samme som at multiplicere en matrix og en vektor, det betyder at en vektor $\vec{u}$ tilhøre spannet af en mængde vektorer $S$, hvis og kun hvis der er en løsning til ligningen $A\vec{x} = \vec{v}$, hvor at vektorene fra $S$ udgør søjlerne i matricen $A$.
Løsningen vil være vektoren $\vec{x}=\vec{c}$, hvis $i$te indgang vil være skalaren $c_i$.
Det betyder at hvis $S$ kan udspænde $\mathds{R}^n$ skal der være en løsning til $A \vec{x} = \vec{v}$ for en hver vektor $\vec{v} \in \mathds{R}^n$. 
Tilføjes det endnu en søjle til $A$, som er lineært afhængig af de originale søjler, da vil der stadig være en løsning til $A \vec{x} = \vec{v}$ for en hver vektor $\vec{v} \in \mathds{R}^n$, det antyder at spannet af to delmængder $S$ og $S'$ er ens hvis deres snitmængde er lineært afhængig af deres fællesmængde.
\begin{stn}[Ækvivalente span]
Lad $S = \{v_1,...,v_k\}$ og $S_u = \{v_1,...,v_k, u\}$ være mængder af vektorer i $\mathds{R}^n$, da $span(S) = span(S_u)$, hvis og kun hvis $u \in span(S)$.
\end{stn}
\begin{proof}
Antag først at $\vec{u} \in span(S)$, da er $\vec{u}$ en linear kombination af $v_1,..., v_k$, hvorfor
\begin{align*}
span(S_u) &= \{ \vec{b} \in \mathds{R}^n| \exists \vec{x} \in \mathds{R}^n: \, \sum_{j=1}^k c_j v_{ij} + c_{k+1} u_i  = b_i\}
\\&= \{ \vec{b} \in \mathds{R}^n| \exists \vec{x} \in \mathds{R}^n: \, \sum_{j=1}^k c_j v_{ij} + c_{k+1} \sum_{j=1}^k C_j v_{ij}  = b_i\}
\\&= \{ \vec{b} \in \mathds{R}^n| \exists \vec{x} \in \mathds{R}^n: \, \sum_{j=1}^k K_j v_{ij}  = b_i\} = span(S)
\end{align*}
hvor $c_j, C_j, c_{k+1}, K_j$ er vilkårlige skalare.
\\ Antag $span(S) = span(S_u)$, dvs. at de to mængder udspænder den samme mængde af vektorer, hvorfor at de samme vektorer som er en linear kombination af $\vec{v_1},...,\vec{v_k}, \vec{u}$ også er en linear kombination $\vec{v_1},..., \vec{v_k}$, hvorfor $\vec{u}$ må være en linear kombination af  $\vec{v_1},..., \vec{v_k}$, derfor følger det af Definition \ref{def:span} at $\vec{u} \in span(S).$
\label{stn:akvivalentespan}
\end{proof}
Det betyder at det mindste antal af vektorer som $S$ skal indeholde, før spannet ændres er mængden af lineært uafhængige vektorer i spannet.
Hvis $S$ er lineært uafhængig og udspænder et underrum, da kaldes $S$ for en basis for underrummet.
\begin{defn}[Basis]
Lad $S =\{v_1,...,v_k\}$ hvor $\vec{v_1},...,\vec{v_k} \in \mathds{R}^n$ er lineært uafhængige, og lad $V$ være et underrum til $\mathds{R}^n$, da er $S$ en \textbf{basis} til $V$, hvis $V = span(S)$.
\label{def:basis}
\end{defn}
Det viser sig at spannet af en mængde vektorer altid er et underrum, og derfor er $S$ altid en basis til dets span, hvis $S$ er lineært uafhængig.
\begin{stn}[Span er et underrum]
Lad $S=\{\vec{v_1},...,\vec{v_k}\} \subseteq \mathds{R}^n$, da er $span(S)$ et underrum til $\mathds{R}^n$
\label{stn:spanunderrum}
\end{stn}
\begin{proof}
For at vise at $span(S)$ er et underrum til $\mathds{R}^n$ skal det vises at $span(S)$ overholder alle betingelserne i Definition \ref{def:underrum}.
Først vises betingelse (b), lad  derfor $\vec{u}, \vec{v} \in span(S)$, da vil 
\begin{align*}
\vec{u}+\vec{v}= \sum_{i=1}^k c_i \vec{v_i} + \sum_{i=1}^k c'_i \vec{v_i} = \sum_{i=1} c_i\cdot c_i' \vec{v_i},
\end{align*}
hvor $c_i, c_i'$ er skalare.
Der med er $\vec{u}+\vec{v}$ en linear kombination af $\vec{v_1},...,\vec{v_k}$, hvorfor $\vec{u}+\vec{v} \in span(S)$, og $span(S)$ er lukket under vektor addition.
\\ Så vises at $span(S)$ er lukket under skalar multiplikation, lad derfor $c, c_i$ være skalare, da vil
\begin{align*}
c\vec{v}= c\sum_{i=1}^k c_i \vec{v_i}  = \sum_{i=1} c \cdot c_i \vec{v_i}.
\end{align*}
Hvorfor at $c\vec{v} \in span(S)$, og betingelse (c), er opfyldt.
\\Tilsidst vises det at $\vec{0} \in span(S)$.
Da nul vektoren kan skrives som den linear kombination af vektorene i $S$, $\vec{0} = \sum_{i=1}^k 0 \vec{v_i}$, medfører det at $\vec{0} \in span(S)$, hvorfor at $span(S)$ overholder betingelse (a), og dermed er $span(S)$ et underrum til $\mathds{R}^n$.
\end{proof}
Det betyder at hvis en mængde af vektorer er lineært uafhængige så udgør de en basis for et underrum. 
Den mest kendte basis er standard vektorene, som udspænder $\mathds{R}^n$, da en hver vektor $\vec{v} \in \mathds{R}^n$ kan udtrykkes som en lineare kombination af standard vektorene $\vec{e_i}$, for $i = 1,..., n$; $\vec{v}= \sum_{i=1}^n v_i \vec{e_i}$ hvor $v_i$ er den $i$te indgang i $\vec{v}$.
En anden basis for samme rum er $B=\{2\vec{e_i}| i =1,...,n\}$, her vil $\vec{v} = \sum_{i=1}^n \frac{1}{2} v_i \cdot 2\vec{e_i}$, det betyder at $\vec{v}$ udtrykt i forhold til basisen $B$ vil være $\vec{v}_B = \frac{1}{2}\vec{v}$.
Helt generelt gælder der; hvis $A_B$ er en $n \times m$ matrix hvis søjler er udgjort af vektorerne fra $B$, da kan en vektor $\vec{v}$ udtrykkes i forhold til basisen $B$ som løsningen til ligningssystemet $A \vec{x} =\vec{v}$, det betyder at $\vec{v}_B =  \vec{x}$. 
Ligger vektor $\vec{v}$ ikke i det underrum som $B$ er basis for, vil der ikke være en løsning til $A \vec{x} =\vec{v}$, hvorfor at vektoren ikke kan udtrykkes i forhold til basisen, hvilket også følger af Definition \ref{def:span} og Sætning \ref{stn:spanunderrum}.
Det betyder, at der er forskellige baser til samme underrum, hvor nogle er mere hensigtsmæssige i forskellige tilfælde.
Men da en basis består af lineært uafhængige vektorer som udspænder underrummet, og da to mængder af vektorer ikke kan udspænde samme underrum, have forskellig kardinaliltet og være lineært uafhængige på samme tid, betyder det at to forskellige basisr for samme underrum, altid vil indeholde den samme mængde vektorer.
\begin{stn}[Kardinalteten af en basis]
Lad $V$ være et ikke tomt underrum til $\mathds{R}^n$, og lad $B$ og $B'$ udgøre en basis for $V$, da vil $|B|=|B'|$
\label{stn:basiskardinalitet}
\end{stn}
\begin{proof}
Lad $B$ bestå af $k$ vektorer og $B'$ af $p$ vektorer, antag for modstrid $k < p$, da vil der eksistere to matricer $A_{B}$ og $A_{B'}$, hvis søjler er vektorerne fra henholdsvis $B$ og $B'$.
Hvis $rank(A_{B}) = rank(A_{B'}) = k$ da vil vektorerne i $B'$ være lineært afhængig, hvorfor at $B'$ ikke er en basis.
Hvis $rank(A_{B}) < rank(A_{B'} )$ da vil der eksistere en vektor $\vec{b} \in \mathds{R}^n$ hvor der eksistere en løsning til $A_{B'}\vec{x} = \vec{b}$, men ikke til ligningssystemet $A_B \vec{x}=\vec{b}$, hvorfor at $span(B) \neq span(B')$ hvorfor at både $B$ og $B'$ ikke kan være basis til $V$ af Definition \ref{def:basis}.
Derfor følger det at $p=k$.
\end{proof}
Kardinaliten af basen for et underrum, kaldes dimensionen af underrummet.
\begin{defn}[Dimension]
Lad $B$ være en basis for et ikke tomt underrum $V$ til $\mathds{R}^n$, da er \textbf{dimensionen} af $V$ givet ved $\dim{V} = |B|$
\label{def:dim}
\end{defn}

\subsection{Fra span til base}
Da spannet af to mængder er det samme, hvis deres snitmængde er lineært afhængig af deres fælles mængde, betyder det at hvis deres fælles mængde er lineært uafhængig da er fællesmængden en basis for spannet.
Derfor kan enhver mængde af lineært afhængige vektorer blive til basis, ved at fjerne vektorer til de resterne er lineært uafhængige.
\begin{stn}
Lad $S=\{\vec{v_1}, ..., \vec{k}\}$, for $\vec{v_1}, ..., \vec{k} \in \mathds{R}^n$ og lad $span(S) = V$ være et underrum til $\mathds{R}^n$ da er kan $S$ gøres til en basis for $V$ ved at fjerne vektorer fra $S$.
\label{stn:reduceringbasis}
\end{stn}
\begin{proof}
Antag uden at miste generealitet at de først $l$ vektorer i $S$ er  lineært uafhængige, da vil $\vec{v_i}$, for $i = l+1,...,k$ være en linear kombination af $\{\vec{v_i} | i = 1,...,l\}$. 
Derfor følger det af Sætning \ref{stn:akvivalentespan} at $V = span(\{\vec{v_i}| j =1,...,k\}) =span(\{\vec{v_i}| j=1,...,l\})$.
Da alle vektorer i $\{\vec{v_i}| j=1,...,l\}$ er lineært uafhængige følger det af Definition \ref{def:basis} at $\{\vec{v_i}| j=1,...,l\}$ er en basis for $V$. 
Derfor bliver $S$ en basis ved at fjerne $\{\vec{v_i}| i = l+1,...,k\}$.
\end{proof}
Udspænder delmængden derimod et underrum til et andet underrum kan der tilføjes lineært uafhængige vektorer til delmængden, til den udspænder underrummet.
\begin{stn}
Lad $S=\{\vec{v_1}, ..., \vec{v_k}\}$ være en delmængde for underrummet $V$ til $\mathds{R}^n$, for $\vec{v_1}, ..., \vec{v_k} \in \mathds{R}^n$ være lineært uafhængige vektorer, da vil $S$ kunne udvides til en basis for $V$ ved at tilføje ekstra vektorer.
\end{stn}
\begin{proof}
Observer at $span(S) \subseteq V$, og at $S$ er en basis for $span(S)$.
Lad $\vec{v_{k+1}} \in V$ være lineært uafhængig med $\vec{v_1}, ..., \vec{v_k}$, så vil $span(S\cup\{\vec{v_{k+1}}\}) \subseteq V$ med basis $S\cup\{\vec{v_{k+1}}\}$. 
Gentag til der ikke er flere vektorer i $V$ som er lineært uafhængig med delmængden af vektorer, og kald mængden $S_V$.
Bemærk at $span(S_V) \subseteq V$.
Det vises nu at $V \subseteq span(S_V)$. 
Af Definiton \ref{def:span} indeholder $span(S_V)$ alle vektorer som er en linear kombination af vektorerne i $S_V$, og da der ikke er nogle vektorer i $V$ som er lineært uafhængig til $S_V$ pr konstruktion af $S_V$, må $V$ være indeholdt i $S_V$. 
Hvorfor $V \subseteq span(S_V)$, hvilket medfører at $V = span(S_V)$, med basis $S_V$. 
Da $S_V$ var konstrueret ved at tilføje vektorer til $S$ er sætningen hermed bevist.
\end{proof}
Det betyder at en basis til et underrum er det størst mulige antal lineært uafhængige vektorer i underrummet, en anden måde at betragte det på er også at en basis til underrummet er den mindste mulige mængde af vektorer som udspænder underrummet.
%\begin{stn}
%Lad $S=\{\vec{v_1},..., \vec{v_K}\} \subseteq \mathds{R}^n$ så $\dim(span(S)) = m$ da
%\begin{enumerate}[label=\alph*]
%\item $\exists S_m =\{\vec{v_{B(1)}},...,\vec{v_{B(m)}}\}\subseteq S$ så $span(S_m) = span(S)$, hvor $B(1),...,B(m)$ er indeks.
%\item Lad $S_k = \{\vec{v_1},..., \vec{v_k}\} \subseteq S$ og $k \leq m$, da $\exists S_{m-k} = \{v_{B(k+1)},...,v_{B(m)}\} \subseteq S\setminus S_k$ så $span(S_k \cup S_{m-k}) = span(S)$.
%\end{enumerate} 
%\end{stn}

\begin{stn}[Fra span til basis]
Lad $S=\{\vec{v_1},..., \vec{v_K}\} \subseteq \mathds{R}^n$ så $\dim(span(S)) = m$, og $B(1),...,B(m)$ betegne et indeks, da
\begin{enumerate}[label=\alph*]
\item $\exists S_m =\{\vec{v_{B(1)}},...,\vec{v_{B(m)}}\}\subseteq S$ så $S_m$ er en base til $span (S)$
\item $\exists S_{m-k} = \{v_{B(k+1)},...,v_{B(m)}\} \subseteq S\setminus S_k$, for $S_k = \{\vec{v_1},..., \vec{v_k}\} \subseteq S$ og $k \leq m$ så $S_k \cup S_{m-k}$ udgør en basis for $span(S)$.
\end{enumerate} 
\label{stn:spantilbasis}
\end{stn}
Bemærk at udsagn (a) er et særtilfælde af udsagn (b) med $k=0$, hvor med det kun er nødvendigt at bevise udsagn (b).
\begin{proof}
Antag at der eksistere en vektor $\vec{v_{B(k+1)}} \in S$, som opfylder at $\vec{v_{B(k+1)}} \notin span(S_k)$, dermed kan $S_k$ ikke udgøre en basis for $span(S)$, tilføj derfor $\vec{v_{B(k+1)}}$ til $S_k$, gentag til der ikke eksistere en $\vec{v_{B(k+1)}} \in S$ der er lineært uafhængig af $S_k$. 
\\Antag nu at der ikke eksistere en vektor $\vec{v_{B(k+1)}} \in S$, som opfylder at $\vec{v_{B(k+1)}} \notin span(S_k)$, hvorfor $span(S) = span(S_k)$, og $\dim{span(S)}=\dim{span(S_k)}= m$.
Da $S_k$ kun indeholder lineært uafhængige vektorer medfører det at $S_k$ er en basis for $S$.
\\Da $\dim(span(S)) = m$ og $\dim{span(S_k)}=k$ må det følge af Definition \ref{def:dim}, og Definition \ref{def:basis}, at der skal tilføjes $m-k$ vektorer til $S_k$ for at $S_k$ er en basis til $S$.
%
%
%
%Da $\dim(span(S)) = m$ må det følge af Definition \ref{def:dim}, og Definition \ref{def:basis}, at der må være mindst $m-k$ vektorer i $S_k$ i $S \setminus S_k$, der ikke tilhører $span(S_k)$. 
%Tilføj nu en vektor $\vec{v_{B(k+1)}} \in S$, som opfylder at $\vec{v_{B(k+1)}} \notin span(S_k)$ til $S_k$.
%Gentag til $m=k$, hvilket svare til at tilføje $m-k$ vektorer til $S_k$. 
%Er $k=m$ medføre det, at $span(S_k) = span (S)$, hvorfor at $S_k$ må udgøre en basis for $S$.
\end{proof} 
Dermed kan en mængde af vektorer altid reduceres eller udvides til at udgøre en basis for et underrum.



\section{Span og basis for en matrix}
På samme måde som at en matrix kan skabes ud fra en delmængde af vektorer, så kan en delmængde af vektorer generes udfra søjlerne eller rækkerne i en matrix. 
Søjlerne i en matrix med rang $m$ vil udspænde $\mathds{R}^m$.
\begin{stn}[Span af $\mathds{R}^n$]
Lad $A$ være en $m\times n$ matrix og $S_A= \{A_j| A_j \in A\}$, da er følgende udsagn ækvivalente:
\begin{enumerate}[label=\alph*]
\item $span(S_A) = \mathds{R}^m$
\item $\exists \vec{x} \in \mathds{R}^n \forall \vec{b} \in \mathds{R}^m: \quad A\vec{x}=\vec{b}$
\item $rank(A) = m$
\item Den rækkerederede form af $A$ har ingen nul rækker
\item En hver række i $A$ har en pivot indgang
\end{enumerate}
\end{stn}
\begin{proof}
Først vises (a) <=> (b):
Lad $span(S_A) = \mathds{R}^m$ så følger det af Definition \ref{def:span} at $A\vec{x}= \vec{b}$ har en løsning for ethvert $\vec{b} \in \mathds{R}^m$.
På samme måde hvis $A\vec{x}= \vec{b}$ har en løsning for ethvert $\vec{b} \in \mathds{R}^m$, medfører det at $span(S_A) = \mathds{R}^m$. 
\\(c) <=> (d): 
Lad $rank(A) = m$, da følger det af definitionen for rang at antallet af nulrækker i den rækkereducerede form af matricen $A$ er $m-m = 0$.
På samme måde må rangen af $A$ være lig antallet af rækker, hvis der ikke er nogle nulrækker i den rækkereduceret form af $A$.
\\(d) <=> (e):
Lad den rækkereducerede form af $A$ have ingen nul rækker, da må der være en pivot indgang i hver række. 
Da der skal være en indgang pr. række med værdi forskellig $0$. 
På samme måde gælder at hvis der er en pivot indgang i en række kan det ikke være en nul række, når matricen bringes på rækkereduceret form, derfor må (e) => (d).
\\(b) => (d):
Antag for modstrid at $A$ indeholder en nulrække på rækkereduceret form, lad denne række være $\vec{a_i}^T = \vec{0}$, da vil $A\vec{x} = \vec{b}$ kun have en løsning hvis $b_i=0$, hvor $b_i$ er den $i$te indgang i $\vec{b}$, da $b_i = \vec{a_i}^T \vec{x} = \vec{0} \vec{x} = 0$. 
Derfor er der ikke en løsning til ethvert $\vec{b} \in \mathds{R}^m$, hvorfor der er opstået modstrid.
\\(e) => (b): 
Betragt total matricen $[A \vec{b}]$, da der er en pivot indgang i hver række, medføre det at totalmatricen ikke kan indeholde en række på formen $[\vec{0}^T, b_i]$, hvor $b_i \neq 0$, derfor må der være en løsning til ligningsstemmet $A\vec{x} = \vec{b}$.
Da enhver indgang i $\vec{b}$ kan udtrykkes $b_i = \sum_{j=1}^n a_{ij} x_i = k \cdot x_i$, hvor $a_{ij}$ er den $j$te indgang i den $i$te række, og $k$ er en skalar.
Da et hvert tal kan udtrykkes som produktet af to realle tal, må der altid være en løsning til $A\vec{x}=\vec{b}$.
\end{proof}
Underrummet udspændt af søjlerne i $A$ kaldes for søjlerummet af $A$, mens rækkerummet af $A$ er udspændt af $A$s rækker.
\begin{defn}[Søjlerum og Rækkerum]
Lad $A$ være en $n\times m$ matrix, da er \textbf{søjlerummet} af $A$ $Col A = span(\{A_j | j =1,...,n\})$, mens \textbf{rækkerummet} af $A$ er $Row A = span(\{\vec{a_i}|i=1,...,m\})$.
\label{def:sojlerum}
\end{defn} 
Det betyder at de lineært uafhængige søjler og rækker vil udgøre en basis for henholdsvis søjlerummet og rækkerummet, hvilket er de rækker og søjler med pivot indgang. 
De næste to resultater vil kun blive vist for søjlerummet, da de er ækvivalente til rækkerummet og det samme gælder for deres beviser.
\begin{kor}
Lad $A$ være en $n \times m$ matrix da vil søjlerne med en pivot indgang udgøre en basis for $Col (A)$
\label{kor:pivotsojle}
\end{kor}
\begin{proof}
Antag uden at miste generealitet at de først $k$ søjler i $A$ har en pivot indgang, da er $A_j$, for $j=1,...,k$ lineært uafhængige, mens $A_j$, for $j = k+1,...,n$ vil være en linear kombination af $\{A_j | j = 1,...,k\}$. 
Det følger derfor af Sætning \ref{stn:reduceringbasis} at $Col A = span(\{A_j| j=1,...,k\})$.
\end{proof}
Det betyder at dimensionen af søjlerummet og rækkerummet er lig antallet af pivotindgange i matricen, hvilket pr defintion er rangen af matricen.
\begin{kor}
Lad $A$ være en $n\times m $ matrix da er $\dim{col A} = rank (A)$
\end{kor}
\begin{proof}
Af Korolar \ref{kor:pivotsojle} følger at søjlerne i $A$ med pivot indgang udgør en base for $Col A$, da antallet af søjler med pivot indgang er lig rangen af $A$ må $\dim{Col A} = rank (A)$ i følge Definition \ref{def:dim}.
\end{proof}


%\begin{kor}
%Lad $A$ være en $n\times m $ matrix, da udgør rækkerne med pivot indgang en base for rækkerummet for $A$
%\end{kor}
%\begin{proof}
%Antag uden at miste generealitet at de først $k$ rækker i $A$ har en pivot indgang, da er $\vec{a_i}^T$, for $i=1,...,k$ lineært uafhængige, mens $\vec{a_i}^T$, for $i = k+1,...,n$ vil være en linear kombination af $\{\vec{a_i}^T | i = 1,...,k\}$. 
%Det følger derfor af Sætning \ref{stn:reduceringbasis} at $\{\vec{a_i}^T | i = 1,...,k\}$ vil være en basis for rækkerummet af $A$.
%\end{proof}
\subsection{Dimensionen af underrum}
Dimensionen af et underrum, indeholder indformation om, hvorvidt en delmængde af vektorer fra underrummet er lineært uafhængige, om de udgør en basis for underrummet, men også om delmængden er lig underrummet.
Hvis delmængden af vektorer er indeholdt i et underrum, vil de kun kunne være lineært uafhængige, hvis delmængden maksimalt indeholder et antal vektorer svarende til underrummets dimension. 
Da delmængden ellers ville udspænde et underrum som vil indeholde vektorer, der ikke er indeholdt i det originale underrum.
\begin{kor}
Lad $V$ være et underrum til $\mathds{R}^n$ med $\dim{V}=k$, da vil enhver lineært uafhængig delmængde af $V$ maksimalt indeholde $k$ vektorer.
\label{kor:linearuafhangighedunderrum}
\end{kor}
\begin{proof}
Antag at der eksistere et delmængde $S \subseteq V$ med $p > k$ lineært uafhængige vektorer, hvis $span(S) = V$, da vil $S$ være en basis til $V$ af Definition \ref{def:basis}, men det strider mod Sætning \ref{stn:basiskardinalitet}. Det samme gør sig gældende for alle underrum af $V$.
Derfor må $S$ generer et underrum $W$, hvor $V \subset W$, hvilket strider mod Definition \ref{def:underrum}, da $V$ skal være lukket under vektor addition og skalar multiplikation.
\end{proof}
Indeholder delmængden præcis det antal vektorer som svare til dimensionen af underrummet, da er delmængden en basis til underrummet, hvis delmængden er lineært uafhængig eller udspænder underrummet.
\begin{kor}
Lad $V$ være et underrum til $\mathds{R}^n$ med $\dim{V}=k$, da vil $S \subseteq V$, hvor $|S|=k$ være en basis for $V$, hvis $S$ er lineært uafhængig eller $span(S) = V$.
\label{kor:serbase}
\end{kor}
\begin{proof}
Lad først $S$ være lineært uafhængig, da følger det af Definition \ref{def:basis}, at $S$ er en basis for et $k$ dimensionalt underrum til $V$.
Antag nu at der er en vektor $\vec{v}$ i $V$ som ikke er i $span(S)$, da vil $\vec{v}$ være lineært uafhængig med $S$, hvorfor at $\dim{V} > \dim{span(S)}$, hvilket strider mod defintionen af $S$. 
Derfor må $V = span(S)$, og $S$ må derfor udgøre en basis for $V$.
 som må være lig $V$, hvorfor at $S$ er en base til $V$.
\\Lad nu $span(S) = V$, da må $\dim{span(S)} = \dim{V} = k$, hvorfor det følger af Definition \ref{def:dim} og Definition\ref{def:basis} at $S$ består af $k$ lineært uafhængige vektorer, og da $|S|=k$ må $S$ derfor være en basis for $V$.
\end{proof}
Da en lineært uafhængig delmængde af vektorer med samme kardinaliltet som dimensionen af underrummet, må det udgøre en basis for at underrum som er lig underrummet selv.
\begin{stn}
Lad $V$ og $W$ være underrum til $\mathds{R}^n$, og lad $V \subseteq W$, da vil $\dim{V} \leq \dim{W}$, i tilfældet at $\dim{V}=\dim{W}$ da er $V=W$.
\label{stn:dimunderrum}
\end{stn}
\begin{proof}
Af Korolar \ref{kor:linearuafhangighedunderrum} følger det at en hver lineær uafhængig delmængde $S$ af $W$ vil indeholde mindre end eller lig med $\dim{W}$ vektorer, hvorfor at ethvert underrum udspændt af $S$ vil have $\dim{span(S)} \leq \dim{W}$.

Lad $B_V$ være en base for $V$, hvorfor at $|B_V| = \dim{V} = \dim{W}$, og da $B_V$ er lineært uafhængig følger det af Korolar \ref{kor:serbase} at $V = span(B_V) = W$.
\end{proof}


