Rummet $\mathds{R}^n$ er mængden af vektorer af dimension $n$, og på samme måde som med mængder generelt, er det muligt at betragte delmængder af $\mathds{R}^n$.
Et special tilfælde af disse delmængder er et underrum.
\begin{defn}[Underrum]
Lad $W$ være en mængde af vektorer $\vec{v_1},...,\vec{v_k} \in \mathds{R}^n$, da er $W$  \textbf{underrum} til $\mathds{R}^n$, hvis:
\begin{enumerate}[label=\alph*]
\item $\vec{0} \in W$
\item $\vec{u}+\vec{v} \in W \quad \forall \vec{u}, \vec{v} \in W$
\item $c \cdot \vec{v} \in W \quad \forall \vec{v} \in W, \forall c \in \mathds{R}$
\end{enumerate}
\label{def:underrum}
\end{defn}
En anden delmængde er et span.
\begin{defn}[Span]
Lad $S=\{\vec{v_1},...,\vec{v_k}\}$ være en ikke tom mængde af vektorer, hvor $\vec{v_i} \in \mathds{R}^n$ for $i = 1,..,k$. 
Da er \textbf{spannet af $S$} mængden af vektorer
\begin{align*}
span(S) = \{\vec{u}| \vec{u}=\sum_{i=0}^k c_i \vec{v_i}, \vec{v_i} \in S, c_i \in \mathds{R}\}.
\end{align*} 
og $S$ siges at udspænde dets span.
\label{def:span}
\end{defn}
Det vises nu, at spannet opfylder kravene for et underrum.
\begin{stn}[Span er et underrum]
Lad $S=\{\vec{v_1},...,\vec{v_k}\} \subseteq \mathds{R}^n$, da er $span(S)$ et underrum til $\mathds{R}^n$
\label{stn:spanunderrum}
\end{stn}
\begin{proof}
For at vise at $span(S)$ er et underrum til $\mathds{R}^n$ skal det vises at $span(S)$ overholder alle betingelserne i Definition \ref{def:underrum}.
Først vises betingelse (b), lad  derfor $\vec{u}, \vec{w} \in span(S)$, da vil 
\begin{align*}
\vec{u}+\vec{w}= \sum_{i=1}^k c_i \vec{v_i} + \sum_{i=1}^k c'_i \vec{v_i} = \sum_{i=1} c_i\cdot c_i' \vec{v_i},
\end{align*}
hvor $c_i, c_i'$ er skalare.
Der med er $\vec{u}+\vec{w}$ en linear kombination af $\vec{v_1},...,\vec{v_k}$, hvorfor $\vec{u}+\vec{v} \in span(S)$, og $span(S)$ er lukket under vektor addition.
\\ Så vises at $span(S)$ er lukket under skalar multiplikation, lad derfor $c, c_i$ være skalare, da vil
\begin{align*}
c\vec{w}= c\sum_{i=1}^k c_i \vec{v_i}  = \sum_{i=1} c \cdot c_i \vec{v_i}.
\end{align*}
Hvorfor at $c\vec{w} \in span(S)$, og betingelse (c), er opfyldt.
\\Tilsidst vises det at $\vec{0} \in span(S)$.
Da nul vektoren kan skrives som den linear kombination af vektorene i $S$; $\vec{0} = \sum_{i=1}^k 0 \vec{v_i}$, medfører det, at $\vec{0} \in span(S)$, hvorfor at $span(S)$ overholder betingelse (a), og dermed er $span(S)$ et underrum til $\mathds{R}^n$.
\end{proof}
Det betyder, at spannet er et underrum udspændt af alle de mulige lineære kombinationer af en mængde vektorer. 
Derfor må to mængder vektorer som er lineært afhængig af hinanden derfor udspænde samme underrum.
\begin{stn}[Ækvivalente span]
Lad $S = \{\vec{v_1},...,\vec{v_k}\}$ og $S_u = \{\vec{v_1},...,\vec{v_k}, \vec{u}\}$ være mængder af vektorer i $\mathds{R}^n$, da $span(S) = span(S_u)$, hvis og kun hvis $u \in span(S)$.
\label{stn:akvivalentespan}
\end{stn}
\begin{proof}
Antag først, at $\vec{u} \in span(S)$, da er $\vec{u}$ en linear kombination af $v_1,..., v_k$, hvorfor
\begin{align*}
span(S_u) &= \{ \vec{b} \in \mathds{R}^n\mid \exists \vec{x} \in \mathds{R}^n: \, \sum_{i=1}^k c_i \vec{v_i} + c_{k+1} \vec{u}  =\vec{b}\}
\\&= \{ \vec{b} \in \mathds{R}^n\mid \exists \vec{x} \in \mathds{R}^n: \, \sum_{i=1}^k c_i \vec{v_i} + c_{k+1} \sum_{i=1}^k C_i \vec{v_i} = \vec{b}\}
\\&= \{ \vec{b} \in \mathds{R}^n\mid \exists \vec{x} \in \mathds{R}^n: \, \sum_{i=1}^k K_j \vec{v_i} = \vec{b}\} = span(S)
\end{align*}
hvor $c_j, C_j, c_{k+1}, K_j$ er vilkårlige skalare.
\\ Antag $span(S) = span(S_u)$, dvs. at de to mængder udspænder den samme mængde af vektorer, hvorfor de samme vektorer som er en linear kombination af $\vec{v_1},...,\vec{v_k}, \vec{u}$ også er en linear kombination $\vec{v_1},..., \vec{v_k}$, dermed må $\vec{u}$ være en linear kombination af  $\vec{v_1},..., \vec{v_k}$, hvorefter det følger af Definition \ref{def:span}, at $\vec{u} \in span(S)$.
\end{proof}
Derfor må den mindste mængde vektorer, som udspænder et underrum, være lineært uafhængige.
Denne mængde af vektorer kaldes en basis.
\begin{defn}[Basis]
Lad $S =\{v_1,...,v_k\}$, hvor $\vec{v_1},...,\vec{v_k} \in \mathds{R}^n$ er lineært uafhængige, og lad $V$ være et underrum til $\mathds{R}^n$, da er $S$ en \textbf{basis} til $V$, hvis $V = span(S)$.
\label{def:basis}
\end{defn}
Antallet af vektorerne i basen vil være konstant.
\begin{stn}[Kardinalteten af en basis]
Lad $V$ være et ikke tomt underrum til $\mathds{R}^n$, og lad $B$ og $B'$ udgøre en basis for $V$, da vil $|B|=|B'|$.
\label{stn:basiskardinalitet}
\end{stn}
\begin{proof}
Lad $B$ bestå af $k$ vektorer og $B'$ af $p$ vektorer, antag for modstrid $k < p$, da vil der eksistere to matricer $A_{B}$ og $A_{B'}$, hvis søjler er vektorerne fra henholdsvis $B$ og $B'$.
Hvis $rank(A_{B}) = rank(A_{B'}) = k$ da vil vektorerne i $B'$ være lineært afhængig, hvorfor $B'$ ikke er en basis.
Hvis $rank(A_{B}) < rank(A_{B'} )$ da vil der eksistere en vektor $\vec{b} \in \mathds{R}^n$, hvor der eksistere en løsning til $A_{B'}\vec{x} = \vec{b}$, men ikke til ligningssystemet $A_B \vec{x}=\vec{b}$, hvorfor $span(B) \neq span(B')$, derfor kan både $B$ og $B'$ ikke være basis til $V$ af Definition \ref{def:basis}.
Derfor følger det at $p=k$.
\end{proof}
Hvilket fører til denne definition.
\begin{defn}[Dimension]
Lad $B$ være en basis for et ikke tomt underrum $V$ til $\mathds{R}^n$, da er \textbf{dimensionen} af $V$ givet ved $\dim{V} = |B|$
\label{def:dim}
\end{defn}
Selvom der ikke kan være forskellige antal vektorer i en basis, kan forskellige vektorer udgøre en basis for samme underrum.
Den mest kendte basis for $\mathds{R}^n$ er standardvektorene, og ved notationen $\rvect{a & b}^T$ forståes den lineære kombination
\begin{align*}
\begin{bmatrix} a \\ b \end{bmatrix} = a\begin{bmatrix} 1 \\0 \end{bmatrix} + b \begin{bmatrix} 0 \\ 1 \end{bmatrix}.
\end{align*}
Standard vektorerne udgør det, som kaldes en ortonormalbasis.




