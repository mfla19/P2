\subsection{Ortogonal komplement}
Ortogonalitet kan også bruges til at generere en ny delmængde af $\mathds{R}^n$ ud fra en given delmængde.
\begin{defn}[Ortogonal komplement]
En delmængde vektorer $S^{\bot} \subseteq \mathds{R}^n$ kaldes det \textbf{ortogonale komplement} til delmængden $S \subseteq \mathds{R}^n$, hvis 
\begin{align*}
	S^{\bot} = \{\vec{v} \in \mathds{R}^n \mid \vec{v}^T\vec{u} = 0, \, \forall \vec{u} \in S\}
\end{align*}
\label{def:ortokom}
\end{defn}
Det ortogonale komplement indeholder dermed alle de elementer, der er ortogonale til mængden.
\begin{prop}
Hvis $\vec{v} \in S$ og $\vec{v} \in S^{\bot}$, så er  $\vec{v}=\vec{0}$.
\label{prop:nulortokomp}
\end{prop}
\begin{proof}
Hvis $\vec{v} \in S$ og $\vec{v} \in S^{\bot}$ så følger det af Definition \ref{def:ortokom} at $\vec{v}^T\vec{v} = 0$.
Dermed så $\vec{v}^T\vec{v} = \sum_{i=1}^n v_i^2 =0$, hvilket kun er muligt, hvis $v_i = 0$ for ethvert $i = 1,..., n$, hvorfor $\vec{v}=\vec{0}$.
\end{proof}
Dermed indeholder både mængden og dets ortogonale komplement nulvektoren, da nulvektoren er ortogonal til enhver vektor i følge Lemma \ref{lma:vinkelret}. 
Det resultere i, at det ortogonale komplement aldrig kan være lig den tommemængde.
En mængde og dens ortogonale komplement har den egenskab at de tilsammen udspænder hele $\mathds{R}^n$.
\begin{stn}[$\mathds{R}^n$ udspændt af $W$ og $W^{\bot}$]
Lad $W$ være et underrum til $\mathds{R}^n$, da eksistere der entydige vektorer  $\vec{w} \in W$ og $\vec{z} \in W^{\bot}$ så $\vec{u}= \vec{w}+\vec{z}$ for en hver $\vec{u} \in \mathds{R}^n$.
\label{stn:Rnorto}
\end{stn}
\begin{proof}
Lad $\{\vec{v}_1,...,\vec{v}_k\}$ betegne en ortonormal basis for $W$, og betragt en vilkårlig $\vec{u} \in \mathds{R}^n$. 
Lad $w \in W$ være det element, der er en linear kombination af $\{\vec{v}_1,...,\vec{v}_k\}$ med skalare $\lambda_i = \vec{u}^T\vec{v}_i$, da følger det, at
\begin{align*}
\vec{v}_j^T\vec{w} &= \vec{v}_j^T\sum_{i=0}^k(\vec{u}^T\vec{v}_i)\vec{v}_i
\\ & = (\vec{u}^T\vec{v}_j)\vec{v}_j^T\vec{v}_j = \vec{u}^T\vec{v}_j \Vert \vec{v}_j \Vert^2 = \vec{u}^T\vec{v}_j,
\end{align*}
for $j=1,...,k$, da $\{\vec{v}_1,...,\vec{v}_k\}$ er en ortonormal basis.
\\ Lad nu $\vec{z} = \vec{u}- \vec{w}$, da $\vec{v}_j^T\vec{w}= \vec{v}_j^T\vec{u}$ for et hvert $j=1,...,k$ gælder at 
\begin{align*}
\vec{v}_j^T\vec{z} = \vec{v}_j^T\vec{u}- \vec{v}_j^T\vec{w} = 0,
\end{align*}
derfor må $\vec{z} \in W^{\bot}$. 
Då $\vec{u}$ er valgt vilkårligt, medfører det, at der eksisterer vektorer $\vec{w} \in W$ og $\vec{z} \in W^{\bot}$ så ethvert $\vec{u} \in \mathds{R}^n$ så $\vec{u}= \vec{w}+\vec{z}$.
\\Så vises, at $\vec{w}$ og $\vec{z}$ er entydige.
Lad derfor $\vec{w}' \in W$ og $\vec{z}' \in W^{\bot}$ så $\vec{u}= \vec{w}' + \vec{z}'$.
Da må $\vec{w} + \vec{z} = \vec{w}' + \vec{z}'$, hvorfor $\vec{w}-\vec{w}' = \vec{z}-\vec{z}'$.
Det følger derfor, at $\vec{w}-\vec{w}', \vec{z}-\vec{z}' \in W$ og $\vec{w}-\vec{w}', \vec{z}-\vec{z}' \in W^{\bot}$, hvilket, ifølge Proportion \ref{prop:nulortokomp}, betyder, at $\vec{w}-\vec{w}' = \vec{z}-\vec{z}' = \vec{0}$, hvormed $\vec{w}= \vec{w}'$ og $\vec{z}=\vec{z}'$ og entydigheden er bevist.
\end{proof}
Derfor vil, foreningsmængden af en delmængde af $\mathds{R}^n$ og dets ortogonale komplements baser udgøre en basis for $\mathds{R}^n$.
\begin{kor}
Lad $B$  og $B^{\bot}$ udgøre en basis for henholdsvis underrummet $W$ til $\mathds{R}^n$, og dets ortogonale komplement $W^{\bot}$, da vil $\{B, B^{\bot}\}$ være en basis for $\mathds{R}^n$.
\label{kor:basisRnorto}
\end{kor}
\begin{proof}
Det følger af Sætning \ref{stn:Rnorto}, at et hvert element i $\mathds{R}^n$ kan skrives som en linear kombination af af et element fra $W$ og et fra $W^{\bot}$, derfor må $span\{B, B^{\bot}\} = \mathds{R}^n$. 
Da $B$ og $B^{\bot}$ er lineært uafhængige i følge Lemma \ref{lma:ortolinuaf} følger det af Korollar \ref{kor:serbase}, at $\{B, B^{\bot}\}$ udgør en basis for $\mathds{R}^n$.
\end{proof}
\begin{bem}
Det følger derfor af \ref{kor:basisRnorto}, at
\begin{align}
\dim{\mathds{R}^n} = \dim{span\{B, B^{\bot}\}} = \dim{W} + \dim{W^{\bot}} = n.
\end{align}
\end{bem}
 