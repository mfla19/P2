\subsection{Invers matrix}
Hvis matrixproduktet af to matricer, $A$ og $B$, giver identitetsmatricen, siges $B$ at være den inverse matrix til $A$. 
\begin{defn}[Invers matrix]
Lad $A$ og $B$ være kvadratiske $n \times n$ matricer. Lad $AB=BA=I_n$, hvor $I_n$ er identitetsmatricen. Så er $A$ invertibel og $B$ er den inverse matrix til $A$. $B$ noteres da $A^{-1}$. 
\label{def(inversmatrix)}
\end{defn}
I forlængelse af dette er det værd at bemærke, at den inverse matrix til A er en entydig matrix.
\begin{stn}\label{stn:invers_unik}
Den inverse matrix til $A$ er en entydig/unik matrix. 
\end{stn}
\begin{proof}
Antag at der findes to inverse matricer til $A$, disse kaldes $B$ og $C$. De to matricer vil være den samme da: 
\begin{align*}
B=BI_n=B(AC)=(BA)C=I_nC=C
\end{align*}
\end{proof}
Den inverse matrix kan bruges til at løse specifikke ligningssystemer. Lad $A$ være en $n \times n$ matrix, og $\vec{b}$ være en vektor i $\mathds{R}^n$. Betragt ligningssystemet: 
\begin{align*}
A \vec{x} &= \vec{b}\\
A^{-1} A \vec{x} &= A^{-1} \vec{b}\\
I_n \vec{x} &= A^{-1} \vec{b}.
\end{align*} 
Ved at gange den inverse matrix på begge sider af lighedstegnet, opnåes et udtryk for den entydige løsning til ligningssystemet: 
\begin{align}
\vec{x} &= A^{-1} \vec{b}.
\end{align} 

\begin{stn}
Lad $A$ være en $n \times n$ matrix. Der gælder at: 
\begin{enumerate}[label=(\alph*)]
\item $A$ er invertibel hvis og kun hvis den reducerede trappeform af $A$ er identitesmatricen, $I_n$
\item Hvis $A$ er invertibel, så vil de samme elementære rækkeoperationer som reducerer $A$ til $I_n$ føre $I_n$ over i $A^{-1}$.  
\end{enumerate}
\begin{align*}
\begin{bmatrix}
A & I_n
\end{bmatrix} \sim \dots \sim
\begin{bmatrix}
I_n & A^{-1}
\end{bmatrix}
\end{align*}
\label{stn:inversmatrix}
\end{stn}

\begin{proof}
(a) Først lad $A$ være invertibel. Betragt nu en vektor $\vec{x}$ i $\mathds{R}^n$, som opfylder $A\vec{x}=\vec{0}$. Så fåes det at ligning $(4.1)$, at $\vec{x}=A^{-1} \vec{0}=\vec{0}$. Da løsningen af  $A\vec{x}=\vec{0}$ er $\vec{0}$ må $rang(A)=n$. Det betydet at antallet af pivot-søjler er det samme som antellet af søjler, samt antallet af rækker. Den reducerede trappeform af $A$ må derfor være identitetsmatricen. 

Lad nu den reducerede trappeform af $A$ være identitesmatricen. Så må der findes en invertibel $n \times n$ matrix $P$ således at: $PA=I_n$
\begin{align*}
A=I_nA=(P^{-1}P)A=P^{-1}(PA)=P^{-1}I_n=P^{-1}
\end{align*}
Da $P$ er invertibel må $P^{-1}$ også være invertibel. Derfor er $A$ invertibel. 

(b) Betragt matricen $[A \quad I_n]$. Enhver $n \times n$ matrix, $A$, kan skrives som en matrix, $R$, på reduceret trappeform, ved hjælp af elementære rækkeoperationer. Udføres de samme rækkeoperationer på $n \times 2n$ matricen, $[A \quad I_n]$, tranformeres denne til $[R \quad B]$. Betragt en invertibel matrix $P$, som opfyldet at: $P[A \quad I_n]=[R \quad B]$. Så følger det at:  
\begin{align*}
[R \quad B]=P[A \quad I_n]=[PA \quad PI_n]=[PA \quad P]. 
\end{align*}
Så er $PA=R$ og $P=B$. Hvis $R \neq I_n$ så er $A$ ikke invertibel. Derfor må $R=I_n$, og dermed er $PA=I_n\Leftrightarrow P=A^{-1}$. P er da lig med både $A^{-1}$ og $B$, og $A^{-1}=B$. 
$$[A \quad I_n]\sim[R \quad B]=[PA \quad P]=[I_n \quad A^{-1}]$$
\end{proof}

\begin{eks}
Givet en matrix A ønskes det nu at finde en inverse. 
\begin{align*}
A= \begin{bmatrix}
1 & 3 \\
1 & 2
\end{bmatrix}
\end{align*}
Først opstilles $[A \quad I_n]$ matricen: 
\begin{align*}
\begin{bmatrix}
1 & 3 & 1 & 0 \\
1 & 2 & 0 & 1
\end{bmatrix}.
\end{align*}
Nu udførers elementære rækkeoperationer således at $A$ transformeres til $I_n$. 
\begin{align*}
\begin{bmatrix}
1 & 3 & 1 & 0 \\
1 & 2 & 0 & 1
\end{bmatrix}.
\sim \begin{bmatrix}
1 & 3 & 1 & 0 \\
0 & -1 & -1 & 1
\end{bmatrix}.
\sim \begin{bmatrix}
1 & 3 & 1 & 0 \\
0 & 1 & 1 & -1
\end{bmatrix}.
\sim \begin{bmatrix}
1 & 0 & -2 & 3 \\
0 & 1 & 1 & -1
\end{bmatrix}.
\end{align*}
Det er nu muligt at aflæse $A^{-1}$:
\begin{align*}
A^{-1} =\begin{bmatrix}
-2 & 3\\
1 & -1 
\end{bmatrix}.
\end{align*}
\end{eks}

\subsection{Elementærmatricer og invertibilitet}

\begin{defn}
Enhver elementærmatrix er invertibel. Den inverse af en elementærmatrix er også en elementærmatrix. 
\end{defn}

Enhver rækkeoperation kan laves omvendt. For eksempel kan den matrix der fås, ved at lægge to gange den første række til anden række, ændres tilbage ved at tage $-2$ gange den første række og lægge til den anden række i den nye matrix. På den måde opnås den inverse af en elementærmatrix. \\
Ud fra et antal elementære rækkeoperationer kan der opnås en matrix på reduceret trappeform. Dette kan ligeledes opnås, ved at gange samme antal elementærmatricer, som elementære rækkeroperationer, på den givne matrix. 
Hvis $A$ er en $m \times n$ matrix på reduceret trappeform $R$, findes der elementærmatricer $E_1$, $E_2$, \dots, $E_k$, så 
\begin{align*}
E_k E_{k-1} \dotsm E_1 A = R.
\end{align*}

Hvis $P = E_k E_{k-1} \dotsm E_1$, så er $P$ et produkt af elementærmatricer og derfor invertibel. Derudover er $PA = R$. 

\begin{stn}\label{stn:mn_invertibel}
Lad $A$ være en $m \times n$ matrix på reduceret trappeform $R$. Da findes der en invertibel $m \times m$ matrix $P$ så $PA=R$. 
\end{stn}

\begin{comment}
Sætning \ref{stn:mn_invertibel} følger af følgende resultat. 

\begin{kor}
Matrix-ligningen $A\vec{x}=\vec{b}$ har samme løsninger som $R\vec{x}=\vec{c}$ hvor $\begin{bmatrix} R & \vec{c} \end{bmatrix}$ er matricen $\begin{bmatrix} A & \vec{b} \end{bmatrix}$ på reduceret trappeform.
\end{kor}

\begin{proof}
Der findes en invertibel matrix $P$ således at $P \begin{bmatrix} A & \vec{b} \end{bmatrix} = \begin{bmatrix} R & \vec{c} \end{bmatrix}$. Derfor
\begin{align*}
\begin{bmatrix} 
PA & P\vec{b} 
\end{bmatrix} =
P \begin{bmatrix}
A & \vec{b}
\end{bmatrix} =
\begin{bmatrix}
R \vec{c}
\end{bmatrix},
\end{align*}
og altså $PA = R$ og $P\vec{b}=\vec{c}$. Da $P$ er invertibel, følger det at $A = P^{-1}R$ og $\vec{b}=P^{-1}\vec{c}$.
Hvis $\vec{v}$ er en løsning til $A\vec{x}=\vec{b}$, så $A\vec{v}=\vec{b}$. Så
\begin{align*}
R\vec{v} = (PA)\vec{v} = P(A\vec{v}) = P\vec{b}=\vec{c}. 
\end{align*} 
Hvorfor $\vec{v}$ er en løsning til $R\vec{x}=\vec{c}$. Antag at $\vec{v}$ er en løsning til $R\vec{x} = \vec{c}$. Så er $R\vec{v} = \vec{c}$, og derfor
\begin{align*}
A\vec{v} = (P^{-1}R)\vec{v} = P^{-1}R\vec{v} = P^{-1}\vec{c} = \vec{b}. 
\end{align*}
Derfor er $\vec{v}$ en løsning til $A\vec{x} = \vec{b}$. Udtrykkene $A\vec{x} = \vec{b}$ og $R\vec{x} = \vec{c}$ har altså samme løsninger.
\end{proof}
\end{comment}