\section{Basis}

\begin{defn}
Lad $S =\{v_1,...,v_k\}$ hvor $\vec{v_1},...,\vec{v_k} \in \mathds{R}^n$ er lineært uafhængige, og lad $V$ være et underrum til $\mathds{R}^n$, da er $S$ en \textbf{basis} til $V$, hvis $V = span(S)$.
\label{def:basis}
\end{defn}


\begin{stn}
Lad $S=\{\vec{v_1}, ..., \vec{k}\}$, for $\vec{v_1}, ..., \vec{k} \in \mathds{R}^n$ og lad $span(S) = V$ være et underrum til $\mathds{R}^n$ da er kan $S$ gøres til en basis for $V$ ved at fjerne vektorer fra $S$.
\label{stn:reduceringbasis}
\end{stn}

\begin{proof}
Antag uden at miste generealitet at de først $l$ vektorer i $S$ er  lineært uafhængige, da vil $\vec{v_i}$, for $i = l+1,...,k$ være en linear kombination af $\{\vec{v_i} | i = 1,...,l\}$. 
Derfor følger det af Sætning \ref{stn:akvivalentespan} at $V = span(\{\vec{v_i}| j =1,...,k\}) =span(\{\vec{v_i}| j=1,...,l\})$.
Da alle vektorer i $\{\vec{v_i}| j=1,...,l\}$ er lineært uafhængige følger det af Definition \ref{def:basis} at $\{\vec{v_i}| j=1,...,l\}$ er en basis for $V$. 
Derfor bliver $S$ en basis ved at fjerne $\{\vec{v_i}| i = l+1,...,k\}$.
\end{proof}

\begin{kor}
Lad $A$ være en $n \times m$ matrix da vil søjlerne med en pivot indgang udgøre en basis for $Col (A)$
\end{kor}

\begin{proof}
Antag uden at miste generealitet at de først $k$ søjler i $A$ har en pivot indgang, da er $A_j$, for $j=1,...,k$ lineært uafhængige, mens $A_j$, for $j = k+1,...,n$ vil være en linear kombination af $\{A_j | j = 1,...,k\}$. 
Det følger derfor af Sætning \ref{stn:reduceringbasis} at $Col A = span(\{A_j| j=1,...,k\})$.
\end{proof}

\begin{stn}
Lad $S=\{\vec{v_1}, ..., \vec{k}\}$ være en delmængde for underrummet $V$ til $\mathds{R}^n$, for $\vec{v_1}, ..., \vec{k} \in \mathds{R}^n$ være lineært uafhængige vektorer, da vil $S$ kunne udvides til en basis for $V$ ved at tilføje ekstra vektorer.
\end{stn}
\begin{proof}
Observer at $span(S) \subseteq V$, og $S$ vil være en basis for $span(S)$.
Lad $\vec{v_{k+1}} \in V$ være lineært uafhængig med $\vec{v_1}, ..., \vec{k}$, så vil $span(S\cup\{\vec{v_{k+1}}\}) \subseteq V$ med basis $S\cup\{\vec{v_{k+1}}\}$. 
Gentag til der ikke er flere vektorer i $V$ som er lineært uafhængig med delmængden af vektorer, og kald mængden $S_V$.
Bemærk at $span(S_V) \subseteq V$.
Det vises nu at $V \subseteq span(S_V)$. 
Af Definiton \ref{def:span} indeholder $span(S_V)$ alle vektorer som er en linear kombination af vektorerne i $S_V$, og da der ikke er nogle vektorer i $V$ som er lineært uafhængig til $S_V$ pr konstruktion af $S_V$, må $V$ være indeholdt i $S_V$. 
Hvorfor $V \subseteq span(S_V)$, hvilket medfører at $V = span(S_V)$, med basis $S_V$. 
Da $S_V$ var konstrueret ved at tilføje vektorer til $S$ er sætningen hermed bevist.
\end{proof}



lad $span(S) = V$ være et underrum til $\mathds{R}^n$ da er kan $S$ gøres til en basis for $V$ ved at fjerne vektorer fra $S$.