\section{Span, Basis og Underrum}
Normalt vil der ved en vilkårlig vektor $\vec{v} \in \mathds{R}^n$ forstås den lineare kombination af standard vektorene $\vec{e_1}$, for $i = 1,..., n$; $\vec{v}= \sum_{i=1}^n v_i \vec{e_i}$ hvor $v_i$ er den $i$te indgang i $\vec{v}$. 
Og de $n$ første standard vektorer siges at udspænde eller generere $\mathds{R}^n$.
Dette kan generaliseres for en hvilken som helst mængde af vektorer af samme længde, det gøres ved introduktion af span, basis og underrum.
\subsection{Span}
En mængde af vektorer $S$ udspænder en anden mængde af vektorer $V$, hvis en hver vektor i $V$ kan skrives som en linear kombination af vektorerne i $S$, er det tilfældet kaldes $V$ for spannet af $S$.
\begin{defn}[Span]
Lad $S=\{\vec{v_1},...,\vec{v_k}\}$ være en ikke tom mængde af vektorer, hvor $\vec{v_i} \in \mathds{R}^n$ for $i = 1,..,k$. 
Da er \textbf{spannet af $S$} mængden af vektorer
\begin{align*}
span(S) = \{\vec{u}| \vec{u}=\sum_{i=0}^k c_i \vec{v_i}, \vec{v_i} \in S, c_i \in \mathds{R}\}.
\end{align*} 
\label{def:span}
\end{defn}
Bemærk at $\sum_{i=0}^k c_i \vec{v_i}$ er det samme som at multiplicere en matrix og en vektor, det betyder at en vektor $\vec{u}$ tilhøre spannet af en mængde vektorer $S$, hvis og kun hvis der er en løsning til ligningen $A\vec{x} = \vec{v}$, hvor at vektorene fra $S$ udgør søjlerne i matricen $A$.
Løsningen vil være vektoren $\vec{x}=\vec{c}$, hvis $i$te indgang vil være skalaren $c_i$.
Det betyder at hvis $S$ kan udspænde $\mathds{R}^n$ skal der være en løsning til $A \vec{x} = \vec{b}$ for en hver vektor $\vec{b} \in \mathds{R}^n$. 
Derfor indeholder matricen hvis søjler består vektorene fra $S$ indformation om hvilken mængde af vektorer som $S$ udspænder
\begin{stn}[Span af $\mathds{R}^n$]
Lad $A$ være en $m\times n$ matrix og $S_A= \{A_j| A_j \in A\}$, da er følgende udsagn ækvivalente:
\begin{enumerate}[label=\alph*]
\item $span(S_A) = \mathds{R}^m$
\item $\exists \vec{x} \in \mathds{R}^n \forall \vec{b} \in \mathds{R}^m: \quad A\vec{x}=\vec{b}$
\item $rank(A) = m$
\item Den rækkerederede form af $A$ har ingen nul rækker
\item En hver række i $A$ har en pivot indgang
\end{enumerate}
\end{stn}
\begin{proof}
Først vises (a) <=> (b):
Lad $span(S_A) = \mathds{R}^m$ så følger det af Definition \ref{def:span} at $A\vec{x}= \vec{b}$ har en løsning for ethvert $\vec{b} \in \mathds{R}^m$.
På samme måde hvis $A\vec{x}= \vec{b}$ har en løsning for ethvert $\vec{b} \in \mathds{R}^m$, medfører det at $span(S_A) = \mathds{R}^m$. 
\\(c) <=> (d): 
Lad $rank(A) = m$, da følger det af definitionen for rang at antallet af nulrækker i den rækkereducerede form af matricen $A$ er $m-m = 0$.
På samme måde må rangen af $A$ være lig antallet af rækker, hvis der ikke er nogle nulrækker i den rækkereduceret form af $A$.
\\(d) <=> (e):
Lad den rækkereducerede form af $A$ have ingen nul rækker, da må der være en pivot indgang i hver række. 
Da der skal være en indgang pr. række med værdi forskellig $0$. 
På samme måde gælder at hvis der er en pivot indgang i en række kan det ikke være en nul række, når matricen bringes på rækkereduceret form, derfor må (e) => (d).
\\(b) => (d):
Antag for modstrid at $A$ indeholder en nulrække på rækkereduceret form, lad denne række være $\vec{a_i}^T = \vec{0}$, da vil $A\vec{x} = \vec{b}$ kun have en løsning hvis $b_i=0$, hvor $b_i$ er den $i$te indgang i $\vec{b}$, da $b_i = \vec{a_i}^T \vec{x} = \vec{0} \vec{x} = 0$. 
Derfor er der ikke en løsning til ethvert $\vec{b} \in \mathds{R}^m$, hvorfor der er opstået modstrid.
\\(e) => (b): 
Betragt total matricen $[A \vec{b}]$, da der er en pivot indgang i hver række, medføre det at totalmatricen ikke kan indeholde en række på formen $[\vec{0}^T, b_i]$, hvor $b_i \neq 0$, derfor må der være en løsning til ligningsstemmet $A\vec{x} = \vec{b}$.
Da enhver indgang i $\vec{b}$ kan udtrykkes $b_i = \sum_{j=1}^n a_{ij} x_i = k \cdot x_i$, hvor $a_{ij}$ er den $j$te indgang i den $i$te række, og $k$ er en konstant forskellig fra nul.
Da et hvert tal kan udtrykkes som produktet af to realle tal, må der altid være en løsning til $A\vec{x}=\vec{b}$.
\end{proof}
Hvis $A$ er en $n \times m$ matrix med rang $m$, betyder det derfor at søjlerne udspænder $\mathds{R}^m$, men har tages der ikke højde for $n$ så længe $n \geq m$, dvs. at både $A$ og $A'$ hvor $A'$ er en $l \times m$ matrix, og $m \leq l \leq n$, udspænder $\mathds{R}^m$, hvis rangen af $A'$ også er $m$. 
Det antyder derfor at spannet af to mængder vektorer $S$ og $S'$ er det samme, hvis alle $\vec{v} \in S'\setminus S$ er en linarkombination af vektorene i $S$.
\begin{stn}[Ækvivalente span]
Lad $S = \{v_1,...,v_k\}$ og $S_u = \{v_1,...,v_k, u\}$ være mængder af vektorer i $\mathds{R}^n$, da $span(S) = span(S_u)$, hvis og kun hvis $u \in span(S)$.
\end{stn}
\begin{proof}
Antag først at $\vec{u} \in span(S)$, da er $\vec{u}$ en linear kombination af $v_1,..., v_k$, hvorfor
\begin{align*}
span(S_u) &= \{ \vec{b} \in \mathds{R}^n| \exists \vec{x} \in \mathds{R}^n: \, \sum_{j=1}^k c_j v_{ij} + c_{k+1} u_i  = b_i\}
\\&= \{ \vec{b} \in \mathds{R}^n| \exists \vec{x} \in \mathds{R}^n: \, \sum_{j=1}^k c_j v_{ij} + c_{k+1} \sum_{j=1}^k C_j v_{ij}  = b_i\}
\\&= \{ \vec{b} \in \mathds{R}^n| \exists \vec{x} \in \mathds{R}^n: \, \sum_{j=1}^k K_j v_{ij}  = b_i\} = span(S)
\end{align*}
hvor $c_j, C_j, c_{k+1}, K_j$ er vilkårlige skalare.
\\ Antag $span(S) = span(S_u)$, dvs. at de to mængder udspænder den samme mængde af vektorer, hvorfor at de samme vektore som er en linear kombination af $\vec{v_1},...,\vec{v_k}, \vec{u}$ også er en linear kombination $\vec{v_1},..., \vec{v_k}$, hvorfor $\vec{u}$ må være en linear kombination af  $\vec{v_1},..., \vec{v_k}$, derfor følger det af Definition \ref{def:span} at $\vec{u} \in span(S).$
\label{stn:akvivalentespan}
\end{proof}

\section{Underrum}
\begin{defn}
Lad $W$ være en mængde af vektorer $\vec{v_k} \in \mathds{R}^n$, da er $W$  \textbf{underrum} til $\mathds{R}^n$, hvis:
\begin{enumerate}[label=\alph*]
\item $\vec{0} \in W$
\item $\vec{u}+\vec{v} \in W \quad \forall \vec{u}, \vec{v} \in W$
\item $c \cdot \vec{v} \in W \quad \forall \vec{v} \in W, \forall c \in \mathds{R}$
\end{enumerate}
\label{def:underrum}
\end{defn}
(b) kaldes at være lukket under vektor addition (c) lukket under skalar multiplikation

\begin{stn}
Lad $S=\{\vec{v_1},...,\vec{v_k}\} \subseteq \mathds{R}^n$, da er $span(S)$ et underrum til $\mathds{R}^n$
\end{stn}
\begin{proof}
For at vise at $span(S)$ er et underrum til $\mathds{R}^n$ skal det vises at $span(S)$ overholder alle betingelserne i Definition \ref{def:underrum}.
Først vises betingelse (b), lad  derfor $\vec{u}, \vec{v} \in span(S)$, da vil 
\begin{align*}
\vec{u}+\vec{v}= \sum_{i=1}^k c_i \vec{v_i} + \sum_{i=1}^k c'_i \vec{v_i} = \sum_{i=1} c_i\cdot c_i' \vec{v_i},
\end{align*}
hvor $c_i, c_i'$ er skalare.
Der med er $\vec{u}+\vec{v}$ en linear kombination af $\vec{v_1},...,\vec{v_k}$, hvorfor $\vec{u}+\vec{v} \in span(S)$, og $span(S)$ er lukket under vektor addition.
\\ Så vises at $span(S)$ er lukket under skalar multiplikation, lad derfor $c, c_i$ være skalare, da vil
\begin{align*}
c\vec{v}= c\sum_{i=1}^k c_i \vec{v_i}  = \sum_{i=1} c \cdot c_i \vec{v_i}.
\end{align*}
Hvorfor at $c\vec{v} \in span(S)$, og betingelse (c), er opfyldt.
\\Tilsidst vises det at $\vec{0} \in span(S)$.
Da nul vektoren kan skrives som den linear kombination af vektorene i $S$, $\vec{0} = \sum_{i=1}^k 0 \vec{v_i}$, medfører det at $\vec{0} \in span(S)$, hvorfor at $span(S)$ overholder betingelse (a), og dermed er $span(S)$ et underrum til $\mathds{R}^n$.
\end{proof}

\begin{defn}
Lad $A$ være en $n\times m$ matrix, da er \textbf{søjlerummet} af $A$ $Col A = span(\{A_j | j =1,...,n\}$
\label{def:sojlerum}
\end{defn} 


%\section{Basis}

\begin{defn}
Lad $S =\{v_1,...,v_k\}$ hvor $\vec{v_1},...,\vec{v_k} \in \mathds{R}^n$ er lineært uafhængige, og lad $V$ være et underrum til $\mathds{R}^n$, da er $S$ en \textbf{basis} til $V$, hvis $V = span(S)$.
\label{def:basis}
\end{defn}

\begin{stn}
Lad $A$ være en $n \times m$ matrix da vil søjlerne med en pivot indgang udgøre en basis for $Col (A)$
\end{stn}

\begin{proof}
Antag uden at miste generealitet at de først $k$ søjler i $A$ har en pivot indgang, da er $A_j$, for $j=1,...,k$ lineært uafhængige, mens $A_j$, for $j = k+1,...,n$ vil være en linear kombination af $\{A_j | j = 1,...,k\}$. 
Derfor følger det af Sætning \ref{stn:akvivalentespan} at $span(\{A_j| j =1,...,n\}) =span(\{A_j| j=1,...,k\})$.
Hvorfor det følger af Definition \ref{def:sojlerum} at $Col A = span(\{A_j| j=1,...,k\})$, og da alle vektorer i $\{A_j | j = 1,...,k\}$ er lineært uafhængige følger det af Definition \ref{def:basis} at $\{A_j | j = 1,...,k\}$ er en basis for $Col A$.
\end{proof}
