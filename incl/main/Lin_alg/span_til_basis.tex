\subsection{Fra span til base}
Ligesom en basis kan udvides til at udgøre en basis for et underrum af højere dimension end basens kardinalitet, kan et span reduceres til en basis.
\begin{stn}[Fra span til basis]
Lad $S=\{\vec{v_1},..., \vec{v_K}\} \subseteq \mathds{R}^n$ så $\dim(span(S)) = m$, og lad $B(1),...,B(m)$ betegne et indeks, da gælder
\begin{enumerate}[label=\alph*]
\item $\exists S_m =\{\vec{v}_{B(1)},...,\vec{v}_{B(m)}\}\subseteq S$ så $S_m$ er en base til $span (S)$
\item $\exists S_{m-k} = \{\vec{v}_{B(k+1)},...,\vec{v}_{B(m)}\} \subseteq S\setminus S_k$, for $S_k = \{\vec{v}_{1},..., \vec{v}_k\} \subseteq S$ og $k \leq m$ så $S_k \cup S_{m-k}$ udgør en basis for $span(S)$.
\end{enumerate} 
\label{stn:spantilbasis}
\end{stn}
Bemærk, at udsagn (a) er et særtilfælde af udsagn (b) med $k=0$, hvormed det kun er nødvendigt at bevise udsagn (b).
\begin{proof}
Antag, at der eksisterer en vektor $\vec{v}_{B(k+1)} \in S$, som opfylder, at $\vec{v}_{B(k+1)} \notin span(S_k)$, dermed kan $S_k$ ikke udgøre en basis for $span(S)$. 
Tilføj derfor $\vec{v}_{B(k+1)}$ til $S_k$, gentag til der ikke eksisterer en $\vec{v}_{B(k+1)} \in S$ der er lineært uafhængig af $S_k$. 
\\Antag nu, at der ikke eksisterer en vektor $\vec{v}_{B(k+1)} \in S$, som opfylder at $\vec{v}_{B(k+1)} \notin span(S_k)$, hvorfor $span(S) = span(S_k)$, og $\dim{span(S)}=\dim{span(S_k)}= m$.
Da $S_k$ kun indeholder lineært uafhængige vektorer medfører det, at $S_k$ er en basis for $S$.
\\Da $\dim(span(S)) = m$ og $\dim{span(S_k)}=k$, må det følge af Definition \ref{def:dim} og Definition \ref{def:basis}, at der skal tilføjes $m-k$ vektorer til $S_k$, for at $S_k$ er en basis til $S$.
\end{proof} 
Dermed kan en mængde af vektorer altid reduceres eller udvides til at udgøre en basis for et underrum.


