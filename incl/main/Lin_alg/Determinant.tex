\subsection{Determinant}
Determinanten af en matrix er en scalar, der blandt andet kan bruges til at finde ud af om en matrix er invertibel eller ej.
Først vises, hvordan determinanten af en $2 \times 2$-matrix findes.

\begin{defn}
For en matrix $A$, 
\begin{align*}
A= \begin{bmatrix}
a & c \\
b & d
\end{bmatrix},
\end{align*}
er determinanten defineret ved $det(A)=a \cdot d - b \cdot c$.
\end{defn}

Det er altså relativt nemt at finde determinanten for en $2 \times 2$-matrix. 
Når determinanten så er fundet, kan denne bruges til at finde ud af, om matricen er invertibel. 
Samtidig kan determinanten også bruges til at finde den inverse matrix. 

\begin{stn}
$A$ er invertibel hvis og kun hvis $det(A) \neq 0$ og $A^{-1} = \frac{1}{det(A)} \cdot \begin{bmatrix}
d & -c \\
-b & a
\end{bmatrix}$.
\end{stn}

Determinanten kan findes for alle $N \times N$-matricer ved brug af undermatricer.

\begin{defn}
$A_{ij} =$ matricen $A$, men uden $i$'te række og $j$'te søjle. 
\end{defn}

\begin{eks}
Her ses et eksempel på en undermatrix
\begin{align}
A=\begin{bmatrix}
a & d & g \\
b & e & h \\
c & f & i
\end{bmatrix}
\: A_{12}=\begin{bmatrix}
d & g \\
f & i
\end{bmatrix}.
\end{align}
\end{eks}

\begin{defn}
For $\underset{N \times N}{A}=(a_{ij})$ defineres 
\begin{align*}
det(A)=a_{1,1} \cdot (-1)^{1+1} \cdot det(A_{1,1}) + a_{1,2} \cdot (-1)^{1+2} \cdot det(A_{1,2}) + \dots + a_{1,N} \cdot (-1)^{1+N} \cdot det(A_{1,N}).
\end{align*}
Dette kan også skrives, 
\begin{align*}
det(A)=\sum_{j=1}^{N} a_{1,j} \cdot (-1)^{1+j} \cdot det(A_{1,j}).
\end{align*}
\end{defn}

\textbf{mangler bevis}

For $N \times N$ matricer kan det være omfattende at finde determinanten. Det kræver $N!$ multiplikationer at berenge determinanten af en $N \times N$-matrix. I en $2 \times 2$-matrix kræver det to multiplikationer, i en $3 \times 3$-matrix kræver det 6, fordi alle indgang i den valgte række eller søjle ganges med den tilhørende undermatrix, dette giver $3 \cdot 2$ multiplikationer. Tilføjes endnu en række og søjle så,  ganges $4$ på de tidligere $6$ multiplikationer og det samlede antal bliver så $24$ og sådan fortsætter det.\\

\begin{stn}
Når determinanten i en $N \times N$-matrix beregnes, kan det frit vælges hvilken række eller søjle der tages udgangspunkt i. Tages der udgangspunkt i den $i$'te række ser formlen sådan ud;
\begin{align*}
det(A)=\sum_{j=1}^{N}a_{i,j} \cdot (-1)^{i+j} \cdot det(A_{i,j}),
\end{align*}
her er $i$ konstant. Tages der udgangspunkt i den $j$'te søjle ser formlen sådan ud;
\begin{align*}
det(A)=\sum_{i=1}^{N}a_{i,j} \cdot (-1)^{i+j} \cdot det(A_{i,j}),
\end{align*}
her er $j$ konstant.
\end{stn}

Da det kan være meget omfattende at beregne determinanten for en stor matrix, kan det være smart, at tage udgangspunkt i en række eller søjle med mange nuller.
Dette betyder at flere led i udregningen går ud og dermed formindskes antallet af multiplikationer. 

\begin{eks}
Givet en matrix A,
\begin{align*}
A=\begin{bmatrix}
2 & 3 & 0 & 1 \\
0 & 4 & 7 & 2 \\
1 & 0 & 0 & 3 \\
0 & 5 & 0 & -1
\end{bmatrix}
\end{align*}
findes determinanten ved hjælp af den række eller søjle med flest nuller, i dette tilfælde er det den tredje søjle.
\begin{align*}
det(A)&= 7 \cdot (-1)^{2+3} \cdot det \left( \begin{bmatrix}
2 & 3 & 1 \\
1 & 0 & 3 \\
0 & 5 & -1
\end{bmatrix} \right)+0+0+0 \\
&= -7 \cdot \left( 1 \cdot (-1)^{2+1} \cdot det \left(
\begin{bmatrix}
3 & 1 \\
5 & -1
\end{bmatrix} \right) + 0 + 3 \cdot (-1)^{2+3} \cdot det \left( 
\begin{bmatrix}
2 & 3 \\
0 & 5
\end{bmatrix} \right) \right) \\
&= -7 \cdot (-(3 \cdot (-1)-5 \cdot 1)-3 \cdot(2 \cdot 5 - 0 \cdot 3)) \\
&= -7 \cdot (8-30) \\
&= -7 \cdot (-22) \\
&= 154
\end{align*}
Determinanten af $A$ er altså $154$.
\end{eks}

Specielt for matricer på trappeform er det nemt at finde determinanten, da man her altid vil kunne vælge en søjle der kun har 1 indgang der er forskellig fra nul. 
Dette betyder at det er muligt, bare at gange alle indgange i diagonalen for at finde determinanten. 
Determinanten af en matrix kan dog godt ændre sig hvis der udføres rækkeoperationer, derfor er det ikke muligt bare at reducere en matrice til trappeform og så finde determinanten. 
De elementære rækkeoperationer påvirker determinanten på forskellige måder. 

\begin{stn}
\begin{itemize}
\item \textbf{Ombytning:} Ændrer fortegn på determinanten. 
\item \textbf{Scaling:} Ganger determinanten med k (det k der ganges på den række scalingen udføres på).
\item \textbf{Udskiftning:} Dette ændrer ikke determinanten.
\end{itemize}
\end{stn}

\begin{proof}
\textbf{ikke færdigt bevis}
Lad $A$ være en $N \times N$-matrix, med rækkerne $\vec{a_1}, \vec{a_2}, \dots , \vec{a_N}$. \\
Først bevises (a). 
Det vises først
Så bevises (b). 
Lad $B$ være der fås ved at gange hver indgang i en række $p$ i A med en skalar $k$. 
Da fås 
\begin{align*}
det(B)=\sum_{j=1}^{N}k \cdot a_{i,j} \cdot (-1)^{i+j} \cdot det(A_{i,j}),
\end{align*}
hvilket er det samme som 
\begin{align*}
det(B)=k \cdot \sum_{j=1}^{N}a_{i,j} \cdot (-1)^{i+j} \cdot det(A_{i,j}).
\end{align*}
determinanten for $A$ er
\begin{align*}
det(A)=\sum_{j=1}^{N}a_{i,j} \cdot (-1)^{i+j} \cdot det(A_{i,j})
\end{align*}
udfra det ses det, at determinanten af $B$ opnås ved at gange determinanten af $A$ med $k$, dermed gælder det, at
\begin{align*}
det(B)=k \cdot det(A)
\end{align*}
Til sidst bevises (c). 
Lad $B$ være den matrix der fås ved at bytte rækkerne $p$ og $q$, hvor $p<q$. 
Lad nu $q=p+1$, så er $a_{p,q}=b_{q,p}$ og $A_{p,q}=B_{q,p}$
\end{proof}

\begin{stn}
Lad $A$ og $B$ være to kvadratiske matricer af samme størrelse, så gælder:
\begin{enumerate}
\item A er invertibel $\Leftrightarrow$ $det(A) \neq 0$
\item $det(A \cdot B) = det(A) \cdot det(B)$
\item $det(A^T)=det(A)$
\item $det(A^{-1})=\frac{1}{det(A)}$
\end{enumerate}
\end{stn}

\begin{proof}
\textbf{ikke færdigt bevis}
Først bevises (1), (2) og (3) hvis $A$ er invertibel.\\
\end{proof}