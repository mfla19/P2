\section{Lineære ligningssystemer}\label{afsnit:lign_sys}
Lineære ligninger, der indeholder ukendte variabler, kan skrives på formen

\begin{align*}
a_1x_1+a_2x_2+ \dots +a_nx_n = b,
\end{align*}

hvor $a_1, a_2, \dots , a_n$ og $b$ er reelle tal. 
Her kaldes $a_1,a_2, \dots , a_n$ koefficienter og $b$ er en konstant. En lineær ligning kunne for eksempel se sådan ud:

\begin{align*}
7x_1+3x_2-5x_3 = 10.
\end{align*}

Lineære ligninger må ikke indeholde to variable multipliceret, kvadratroden af en variabel, eller andet der gør den ikke-lineær. \\
Et sæt af $m$ lineære ligninger, der indeholder de samme $n$ variable, hvor både $n$ og $m$ er positive heltal, kaldes et lineært ligningssystem. Lineære ligningssystemer skrives på formen

\begin{align*}
a_{1,1}x_1+a_{1,2}x_2+ &\dots +a_{1,n}x_n = b_1\\
a_{2,1}x_1+a_{2,2}x_2+ &\dots +a_{2,n}x_n = b_2\\
&\vdots \\
a_{m,1}x_1+a_{m,2}x_2+ &\dots +a_{m,n}x_n = b_m
\end{align*}

Koefficienterne i et lineært ligningssystem kan skrives i en matrix, mens variable og konstanterne skrives som vektorer. Et lineært ligningsystem kan så skrives op som en matrix ligning på formen $A \vec{x} = \vec{b}$, hvor

\begin{align*}
A= \begin{bmatrix}
a_{1,1} & a_{1,2} & \dots & a_{1,n} \\
a_{2,1} & a_{2,2} & \dots & a_{2,n} \\
\vdots  &         &       & \vdots  \\
a_{m,1} & a_{m,2} & \dots & a_{m,n}
\end{bmatrix}, \ 
x= \begin{bmatrix}
x_1 \\
x_2 \\
\vdots \\
x_n
\end{bmatrix} og \ 
b= \begin{bmatrix}
b_1 \\
b_2 \\
\vdots \\
b_n
\end{bmatrix}
\end{align*}

Søjlerne i $A$ indeholder koefficienterne $x_1$, $x_2$, $\dots $, $x_n$ og kaldes derfor koefficientmatricen til det lineære ligningssystem. 
Den information, der er nødvendig for at løse et lineært ligningssystem, kan samles i en totalmatrix på formen:
\[
\left[
\begin{array}{cccc|c}
a_{1,1} & a_{1,2} & \dots & a_{1,n} & b_1 \\
a_{2,1} & a_{2,2} & \dots & a_{2,n} & b_2 \\
\vdots  &         &       &         & \vdots \\
a_{m,1} & a_{m,2} & \dots & a_{m,n} & b_n
\end{array}
\right]
\]

Totalmatricen laves ved at tilføje vektor $\vec{b}$ til koefficientmatricen $A$. Totalmatricen noteres således som $[A \ \vec{b}]$. Den lodrette streg mellem koefficienterne og konstanterne er ikke nødvendig, men kan tilføjes for at vise at der er tale om en totalmatrix.\\

Løsningen til et lineært ligningssystem er en vektor i $\mathds{R}^n$, som ser således ud:

\begin{align*}
\begin{bmatrix}
s_1 \\
s_2 \\
\vdots \\
s_n
\end{bmatrix}
\end{align*}

Hvis $A$ er en $m \times n$ matrix, er en vektor $\vec{u}$  i $\mathds{R}^n$ en løsning til $A \vec{x}= \vec{b}$ hvis og kun hvis $A \vec{u}= \vec{b}$.

\begin{eks}
Følgende lineære ligningssystem er givet

\begin{align*}
8x_1+6x_2+4x_3 = 50 \\
2x_1+4x_2+6x_3 = 30.
\end{align*}

Dette kan så skrives i en matrix,

\begin{align*}
\begin{bmatrix}
8 & 6 & 4 & 50 \\
2 & 4 & 6 & 30
\end{bmatrix}.
\end{align*}

Løsningen til dette ligningssystem bliver følgende vektor

\begin{align*}
\begin{bmatrix}
2 \\
5 \\
1
\end{bmatrix}.
\end{align*}

Sætter man løsningen ind som variable fås følgende resultat,

\begin{align*}
8 \cdot 2 + 6 \cdot 5 + 4 \cdot 1 = 50 \\
2 \cdot 2 + 4 \cdot 5 + 6 \cdot 1 = 30.
\end{align*}

Det ses, at vektoren giver en rigtig løsning til ligningssystemet, fordi alle ligningerne går op når vektoren indsættes. 

\end{eks}

\subsection{Elemntære rækkeoperationer}
Når et lineært ligningssystem skal løses, skal flere skridt gennemføres. Først skrives ligningssystemet op som en totalmatrix, så findes matricen på trappeform og dernæst rækkereduceret trappeform ved hjælp af rækkeoperationer. De tre typer af elementære rækkeoperationer gennemgås herunder. \\

\begin{defn}[Elementære rækkeoperationer]
Der findes 3 elementære rækkeoperationer
\begin{enumerate}
\item Ombytning af to rækker. (Ombytning)
\item Multiplicering af en vektor med en scalar. (Scaling)
\item Lægge en række gange en scalar til en anden række. (Udskiftning)
\end{enumerate}
\label{defn_elemen_operation}
\end{defn}

\begin{eks}[Rækkeoperationer]
\textbf{Ombytning}\\
Herunder ses et eksempel på to rækker der ombyttes i en matrix.\\

\begin{align*}
A= \begin{bmatrix}
0 & 2 & 3 \\
1 & 4 & 7
\end{bmatrix}
\sim \begin{bmatrix}
1 & 4 & 7 \\
0 & 2 & 3
\end{bmatrix} = B.
\end{align*}

\textbf{Scaling} \\
Et eksempel på scaling ses nedenfor, hvor første række multipliceres med 2.\\

\begin{align*}
A= \begin{bmatrix}
2 & 1 & 3 \\
0 & 4 & 8 \\
1 & 7 & 1
\end{bmatrix}
\sim \begin{bmatrix}
4 & 2 & 6 \\
0 & 4 & 8 \\
1 & 7 & 1
\end{bmatrix} = B.
\end{align*}

\textbf{Udskiftning}\\
Et eksempel på udskiftning ses herunder, hvor to gange række 2 trækkes fra række 1.\\

\begin{align*}
A= \begin{bmatrix}
1 & 4 & 7 \\
0 & 2 & 3
\end{bmatrix}
\sim \begin{bmatrix}
1 & 0 & 1 \\
0 & 2 & 3
\end{bmatrix} = B.
\end{align*}
\end{eks}

Målet med elementære rækkeoperationer er, at opnå trappeform og derefter reduceret trappeform. 

\begin{defn}[Trappeform]
En matrix siges at være på trappeform hvis følgende tre betingelser er opfyldt. 
\begin{enumerate}
\item Enhver række forskellig fra 0, ligger over enhver nul-række.
\item Den ledende indgang i en række forskellig fra 0, ligger til i en søjle højre for søjlen med den ledende indgang i rækken ovenfor.
\item Hvis en søjle indeholder en ledende indgang, vil alle indgang under den ledende i søjlen være lig 0.
\end{enumerate}
\label{defn_trappe}
\end{defn}

De positioner i matricen, hvor der er en ledende indgang, kaldes pivot-indgange. De søjler der indeholder en pivot-indgang kaldes pivot-søjler. De variable, hvis søjler ikke har pivot, kaldes frie variable og hvis en matrix indeholder frie variable betyder det, at der er uendeligt mange løsninger, fordi resultatet ikke afhænger af de frie variable. Samtidig siges et ligningssystem at være uløseligt, hvis der er pivot i sidste søjle, da det ville betyde at der skulle opnås et resultat forskelligt fra 0, ved kun at multiplicere med 0, hvilket ikke er muligt.\\
Når en matrx er på trappeform kan der udføres flere rækkeoperationer for at få den på reduceret trappeform.

\begin{defn}
Hvis en matrix overholder betingelserne for at være på trappeform og samtidig overholder nedenstående to betingelser, siges matricen at være på reduceret trappeform.
\begin{enumerate}
\item Hvis en søjle indeholder en ledende indgang i en hvilken som helst række, skal alle andre indgange i søjlen være 0.
\item den ledende indgang i hver ikke-nul-række skal være 1.
\end{enumerate}
\end{defn}

Når der skal findes en løsning på et ligningssystem skrives det først på reduceret trappeform, derefter kan hver variabel isoleres og dermed findes en generel løsning, i form af en vektor som beskrevet tidligere. 
 
\begin{eks}
Et eksempel på en matrix på trappeform ses herunder. I matricen er pivotindgangene (1,1), (2,2) og (3,4).
\begin{align*}
A= \begin{bmatrix}
2 & 4 & 2 & 6 & 8 \\
0 & 3 & 6 & 9 & -3 \\
0 & 0 & 0 & 6 & 2
\end{bmatrix}. 
\end{align*}
Denne matrix reduceres nu ved hjælp af rækkeoperationer til reduceret trappeform.\\
\begin{align*}
\begin{bmatrix}
2 & 4 & 2 & 6 & 8 \\
0 & 3 & 6 & 9 & -3 \\
0 & 0 & 0 & 6 & 12
\end{bmatrix}
\sim
\begin{bmatrix}
1 & 2 & 1 & 3 & 4 \\
0 & 3 & 6 & 9 & -3 \\
0 & 0 & 0 & 6 & 12
\end{bmatrix}
\sim
\begin{bmatrix}
1 & 2 & 1 & 3 & 4 \\
0 & 1 & 2 & 3 & -1 \\
0 & 0 & 0 & 1 & 2
\end{bmatrix}
\sim \\
\begin{bmatrix}
1 & 0 & -3 & -3 & 6 \\
0 & 1 & 2 & 3 & -1 \\
0 & 0 & 0 & 1 & 2
\end{bmatrix}
\sim
\begin{bmatrix}
1 & 0 & -3 & 0 & 12 \\
0 & 1 & 2 & 3 & -1 \\
0 & 0 & 0 & 1 & 2
\end{bmatrix}
\sim
\begin{bmatrix}
1 & 0 & -3 & 0 & 12 \\
0 & 1 & 2 & 0 & -7 \\
0 & 0 & 0 & 1 & 2
\end{bmatrix}.
\end{align*}
Nu er matricen på reduceret trappeform. Som det ses, er det lige meget hvad der står i de søjler der ikke er pivotsøjler. Dette ligningssystem har derfor uendeligt mange løsninger. Nu isoleres alle variable så en løsning kan findes:
\begin{align*}
x_1 &= 12 + 3x_3 \\
x_2 &= -7 - 2x_3 \\
x_3 &= x_3 \\
x_4 &= 2
\end{align*}
Dette skrives så på vectorform:
\begin{align*}
\begin{bmatrix}
x_1 \\
x_2 \\
x_3 \\
x_4 
\end{bmatrix}
= \begin{bmatrix}
12 + 3x_3 \\
-7 - 2x_3 \\
x_3 \\
2
\end{bmatrix}
= \begin{bmatrix}
12 \\
-7 \\
0 \\
2
\end{bmatrix}
+ x_3 \begin{bmatrix}
3 \\
-2 \\
1 \\
0
\end{bmatrix}.
\end{align*}
\end{eks}

De variabler der ikke har en pivotsøjle, kaldes frie variable, fordi deres værdier ikke påvirker ligningssystemet. 
Et ligningssystem kan siges at være lineært afhængig eller lineært uafhængig.

\begin{defn}[lineær uafhængighed]
Vektorerne $\vec{v}_1, \dots ,\vec{v}_k$ kaldes lineært uafhængige hvis ligningen
\begin{align*}
x_1 \cdot \vec{v}_1+x_2 \cdot \vec{v}_2 + \dots + x_k \cdot \vec{v}_k =  \vec{0}
\end{align*}
kun har den trivielle løsning
\begin{align*}
x_1=x_2= \dots =x_k=0
\end{align*}
ellers kaldes de lineært afhængige.
\label{defn_lin_uafh}
\end{defn}

Det kan altså siges, at et ligningssystem er lineært uafhængigt hvis der kun er en løsning, så der ingen frie variable er. Hvis et ligningssystem har frie variable er der uendeligt mange løsninger og det kaldes derfor lineært afhængigt. 

\subsection{Span}
\begin{defn}[Span]
Lad $S=\{\vec{v_1},...,\vec{v_k}\}$ være en ikke tom mængde af vektorer, hvor $\vec{v_i} \in \mathds{R}^n$ for $i = 1,..,k$. 
Da er \textbf{spannet af $S$} mængden af vektorer
\begin{align*}
span(S) = \{\vec{u}| \vec{u}=\sum_{i=0}^k c_i \vec{v_i}, \vec{v_i} \in S, c_i \in \mathds{R}\}
\end{align*} 
\label{def:span}
\end{defn}
Bemærk at $\sum_{i=0}^k c_i \vec{v_i}$ er det samme som at multiplicere en matrix og en vektor, det betyder at en vektor $\vec{u}$ tilhøre spannet af en mængde vektorer $S$, hvis og kun hvis der er en løsning til ligningen $A\vec{x} = \vec{v}$, hvor at vektorene fra $S$ udgør søjlerne i matricen $A$.
Løsningen vil være vektoren $\vec{x}=\vec{c}$, hvis $i$te indgang vil være skalaren $c_i$.

\begin{stn}
Lad $A$ være en $m\times n$ matrix og $S_A= \{A_j| A_j \in A\}$, da er følgende udsagn ækvivalente:
\begin{enumerate}[label=\alph*]
\item $span(S_A) = \mathds{R}^m$
\item $\exists \vec{x} \in \mathds{R}^n \forall \vec{b} \in \mathds{R}^m: \quad A\vec{x}=\vec{b}$
\item $rank(A) = m$
\item Den rækkerederede form af $A$ har ingen nul rækker
\item En hver række i $A$ har en pivot indgang
\end{enumerate}
\end{stn}
\begin{proof}
Først vises (a) <=> (b):
Lad $span(S_A) = \mathds{R}^m$ så følger det af Definition \ref{def:span} at $A\vec{x}= \vec{b}$ har en løsning for ethvert $\vec{b} \in \mathds{R}^m$.
På samme måde hvis $A\vec{x}= \vec{b}$ har en løsning for ethvert $\vec{b} \in \mathds{R}^m$, medfører det at $span(S_A) = \mathds{R}^m$. 
(c) <=> (d): 
Lad $rank(A) = m$, da følger det af Definitionen for rang at antallet af nulrækker i den rækkereduserede form af matricen $A$ er $m-m = 0$.
På samme måde må rangen af $A$ være lig antallet af rækker, hvis der ikke er nogle nulrækker i den rækkereduceret form af $A$.
(d) <=> (e):
Lad den rækkereducerede form af $A$ have ingen nul rækker, da må der være en pivot indgang i hver række. 
Da der skal være en indgang pr. række med værdi forskellig $0$. 
På samme måde gælder at hvis der er en pivot indgang i en række kan det ikke være en nul række, når matricen bringes på rækkereduceret form, derfor må (e) => (d).
(b) => (d):
Antag for modstrid at $A$ indeholder en nulrække på rækkereduceret form, lad denne række være $\vec{a_i}^T = \vec{0}$, da vil $A\vec{x} = \vec{b}$ kun have en løsning hvis $b_i=0$, hvor $b_i$ er den $i$te indgang i $\vec{b}$, da $b_i = \vec{a_i}^T \vec{x} = \vec{0} \vec{x} = 0$. 
Derfor er der ikke en løsning til ethvert $\vec{b} \in \mathds{R}^m$, hvorfor der er opstået modstrid.
(e) => (b): 
Betragt total matricen $[A \vec{b}]$, da der er en pivot indgang i hver række, medføre det at totalmatricen ikke kan indeholde en række på formen $[\vec{0}^T, b_i]$, hvor $b_i \neq 0$, derfor må der være en løsning til ligningsstemmet $A\vec{x} = \vec{b}$.
Da enhver indgang i $\vec{b}$ kan udtrykkes $b_i = \sum_{j=1}^n a_{ij} x_i = k \cdot x_i$, hvor $a_{ij}$ er den $j$te indgang i den $i$te række, og $k$ er en konstant forskellig fra nul.
Da et hvert tal kan udtrykkes som produktet af to realle tal, må der altid være en løsning til $A\vec{x}=\vec{b}$.
\end{proof}


\begin{stn}
Lad $S = \{v_1,...,v_k\}$ og $S_u = \{v_1,...,v_k, u\}$ være mængder af vektorer i $\mathds{R}^n$, da $span(S) = span(S_u)$, hvis og kun hvis $u \in span(S)$.
\end{stn}
\begin{proof}
Antag først at $\vec{u} \in span(S)$, da er $\vec{u}$ en linear kombination af $v_1,..., v_k$, hvorfor
\begin{align*}
span(S_u) &= \{ \vec{b} \in \mathds{R}^n| \exists \vec{x} \in \mathds{R}^n: \, \sum_{j=1}^k c_j v_{ij} + c_{k+1} u_i  = b_i\}
\\&= \{ \vec{b} \in \mathds{R}^n| \exists \vec{x} \in \mathds{R}^n: \, \sum_{j=1}^k c_j v_{ij} + c_{k+1} \sum_{j=1}^k C_j v_{ij}  = b_i\}
\\&= \{ \vec{b} \in \mathds{R}^n| \exists \vec{x} \in \mathds{R}^n: \, \sum_{j=1}^k K_j v_{ij}  = b_i\} = span(S)
\end{align*}
hvor $c_j, C_j, c_{k+1}, K_j$ er vilkårlige skalare.
\\ Antag $span(S) = span(S_u)$, dvs. at de to mængder udspænder den samme mængde af vektorer, hvorfor at de samme vektore som er en linear kombination af $\vec{v_1},...,\vec{v_k}, \vec{u}$ også er en linear kombination $\vec{v_1},..., \vec{v_k}$, hvorfor $\vec{u}$ må være en linear kombination af  $\vec{v_1},..., \vec{v_k}$, derfor følger det af Definition \ref{def:span} at $\vec{u} \in span(S).$
\label{stn:akvivalentespan}
\end{proof}

\section{Underrum}
\begin{defn}
Lad $W$ være en mængde af vektorer $\vec{v_k} \in \mathds{R}^n$, da er $W$  \textbf{underrum} til $\mathds{R}^n$, hvis:
\begin{enumerate}[label=\alph*]
\item $\vec{0} \in W$
\item $\vec{u}+\vec{v} \in W \quad \forall \vec{u}, \vec{v} \in W$
\item $c \cdot \vec{v} \in W \quad \forall \vec{v} \in W, \forall c \in \mathds{R}$
\end{enumerate}
\label{def:underrum}
\end{defn}
(b) kaldes at være lukket under vektor addition (c) lukket under skalar multiplikation

\begin{stn}
Lad $S=\{\vec{v_1},...,\vec{v_k}\} \subseteq \mathds{R}^n$, da er $span(S)$ et underrum til $\mathds{R}^n$
\end{stn}
\begin{proof}
For at vise at $span(S)$ er et underrum til $\mathds{R}^n$ skal det vises at $span(S)$ overholder alle betingelserne i Definition \ref{def:underrum}.
Først vises betingelse (b), lad  derfor $\vec{u}, \vec{v} \in span(S)$, da vil 
\begin{align*}
\vec{u}+\vec{v}= \sum_{i=1}^k c_i \vec{v_i} + \sum_{i=1}^k c'_i \vec{v_i} = \sum_{i=1} c_i\cdot c_i' \vec{v_i},
\end{align*}
hvor $c_i, c_i'$ er skalare.
Der med er $\vec{u}+\vec{v}$ en linear kombination af $\vec{v_1},...,\vec{v_k}$, hvorfor $\vec{u}+\vec{v} \in span(S)$, og $span(S)$ er lukket under vektor addition.
\\ Så vises at $span(S)$ er lukket under skalar multiplikation, lad derfor $c, c_i$ være skalare, da vil
\begin{align*}
c\vec{v}= c\sum_{i=1}^k c_i \vec{v_i}  = \sum_{i=1} c \cdot c_i \vec{v_i}.
\end{align*}
Hvorfor at $c\vec{v} \in span(S)$, og betingelse (c), er opfyldt.
\\Tilsidst vises det at $\vec{0} \in span(S)$.
Da nul vektoren kan skrives som den linear kombination af vektorene i $S$, $\vec{0} = \sum_{i=1}^k 0 \vec{v_i}$, medfører det at $\vec{0} \in span(S)$, hvorfor at $span(S)$ overholder betingelse (a), og dermed er $span(S)$ et underrum til $\mathds{R}^n$.
\end{proof}

\begin{defn}
Lad $A$ være en $n\times m$ matrix, da er \textbf{søjlerummet} af $A$ $Col A = span(\{A_j | j =1,...,n\}$
\label{def:sojlerum}
\end{defn} 


\section{Basis}

\begin{defn}
Lad $S =\{v_1,...,v_k\}$ hvor $\vec{v_1},...,\vec{v_k} \in \mathds{R}^n$ er lineært uafhængige, og lad $V$ være et underrum til $\mathds{R}^n$, da er $S$ en \textbf{basis} til $V$, hvis $V = span(S)$.
\label{def:basis}
\end{defn}

\begin{stn}
Lad $A$ være en $n \times m$ matrix da vil søjlerne med en pivot indgang udgøre en basis for $Col (A)$
\end{stn}

\begin{proof}
Antag uden at miste generealitet at de først $k$ søjler i $A$ har en pivot indgang, da er $A_j$, for $j=1,...,k$ lineært uafhængige, mens $A_j$, for $j = k+1,...,n$ vil være en linear kombination af $\{A_j | j = 1,...,k\}$. 
Derfor følger det af Sætning \ref{stn:akvivalentespan} at $span(\{A_j| j =1,...,n\}) =span(\{A_j| j=1,...,k\})$.
Hvorfor det følger af Definition \ref{def:sojlerum} at $Col A = span(\{A_j| j=1,...,k\})$, og da alle vektorer i $\{A_j | j = 1,...,k\}$ er lineært uafhængige følger det af Definition \ref{def:basis} at $\{A_j | j = 1,...,k\}$ er en basis for $Col A$.
\end{proof}
