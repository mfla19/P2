\section{Lineære ligningssystemer}
Lineære ligninger, der indeholder ukendte variabler, kan skrives på formen

\begin{align*}
a_1x_1+a_2x_2+ \dots +a_nx_n = b,
\end{align*}

hvor $a_1, a_2, \dots , a_n$ og $b$ er reelle tal. 
Her kaldes $a_1,a_2, \dots , a_n$ koefficienter og $b$ er en konstant. En lineær ligning kunne for eksempel se sådan ud:

\begin{align*}
7x_1+3x_2-5x_3 = 10.
\end{align*}

Lineære ligninger må ikke indeholde to variable multipliceret, kvadratroden af en variabel, eller andet der gør den ikke-lineær. \\
Et sæt af $m$ lineære ligninger, der indeholder de samme $n$ variable, hvor både $n$ og $m$ er positive heltal, kaldes et lineært ligningssystem. Lineære ligningssystemer skrives på formen

\begin{align*}
a_{1,1}x_1+a_{1,2}x_2+ &\dots +a_{1,n}x_n = b_1\\
a_{2,1}x_1+a_{2,2}x_2+ &\dots +a_{2,n}x_n = b_2\\
&\vdots \\
a_{m,1}x_1+a_{m,2}x_2+ &\dots +a_{m,n}x_n = b_m
\end{align*}

Koefficienterne i et lineært ligningssystem kan skrives i en matrix, mens variable og konstanterne skrives som vektorer. Et lineært ligningsystem kan så skrives op som en matrix ligning på formen $A \vec{x} = \vec{b}$, hvor

\begin{align*}
A= \begin{bmatrix}
a_{1,1} & a_{1,2} & \dots & a_{1,n} \\
a_{2,1} & a_{2,2} & \dots & a_{2,n} \\
\vdots  &         &       & \vdots  \\
a_{m,1} & a_{m,2} & \dots & a_{m,n}
\end{bmatrix}, \ 
x= \begin{bmatrix}
x_1 \\
x_2 \\
\vdots \\
x_n
\end{bmatrix} og \ 
b= \begin{bmatrix}
b_1 \\
b_2 \\
\vdots \\
b_n
\end{bmatrix}
\end{align*}

Søjlerne i $A$ indeholder koefficienterne $x_1$, $x_2$, $\dots $, $x_n$ og kaldes derfor koefficientmatricen til det lineære ligningssystem. 
Den information, der er nødvendig for at løse et lineært ligningssystem, kan samles i en totalmatrix på formen:
\[
\left[
\begin{array}{cccc|c}
a_{1,1} & a_{1,2} & \dots & a_{1,n} & b_1 \\
a_{2,1} & a_{2,2} & \dots & a_{2,n} & b_2 \\
\vdots  &         &       &         & \vdots \\
a_{m,1} & a_{m,2} & \dots & a_{m,n} & b_n
\end{array}
\right]
\]

Totalmatricen laves ved at tilføje vektor $\vec{b}$ til koefficientmatricen $A$. Totalmatricen noteres således som $[A \ \vec{b}]$. \\

Løsningen til et lineært ligningssystem er en vektor i $\mathds{R}^n$, som ser således ud:

\begin{align*}
\begin{bmatrix}
s_1 \\
s_2 \\
\vdots \\
s_n
\end{bmatrix}
\end{align*}

Hvis $A$ er en $m \times n$ matrix, er en vektor $\vec{u}$  i $\mathds{R}^n$ en løsning til $A \vec{x}= \vec{b}$ hvis og kun hvis $A \vec{u}= \vec{b}$.

\begin{eks}
Følgende lineære ligningssystem er givet

\begin{align*}
8x_1+6x_2+4x_3 = 50 \\
2x_1+4x_2+6x_3 = 30.
\end{align*}

Dette kan så skrives i en matrix,

\begin{align*}
\begin{bmatrix}
8 & 6 & 4 & 50 \\
2 & 4 & 6 & 30
\end{bmatrix}.
\end{align*}

Løsningen til dette ligningssystem bliver følgende vektor

\begin{align*}
\begin{bmatrix}
2 \\
5 \\
1
\end{bmatrix}.
\end{align*}

Sætter man løsningen ind som variable fås følgende resultat,

\begin{align*}
8 \cdot 2 + 6 \cdot 5 + 4 \cdot 1 = 50 \\
2 \cdot 2 + 4 \cdot 5 + 6 \cdot 1 = 30.
\end{align*}

Det ses, at vektoren giver en rigtig løsning til ligningssystemet, fordi alle ligningerne går op når vektoren indsættes. 

\end{eks}