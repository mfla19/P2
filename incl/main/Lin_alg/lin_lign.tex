\section{Lineære ligningssystemer}\label{afsnit:lign_sys}
Lineære ligninger, der indeholder ukendte variable, kan skrives på formen

\begin{align*}
a_1x_1+a_2x_2+ \dots +a_nx_n = b,
\end{align*}

hvor $a_1, a_2, \dots , a_n$ og $b$ er reelle tal. 
Her kaldes $a_1,a_2, \dots , a_n$ koefficienter og $b$ er en konstant. En lineær ligning kan for eksempel se således ud:

\begin{align*}
7x_1+3x_2-5x_3 = 10.
\end{align*}

Lineære ligninger må ikke indeholde en multiplikation af to variable, kvadratroden af en variabel, eller andet, der gør ligningen ikke-lineær. \\
Et sæt af $m$ lineære ligninger, der indeholder de samme $n$ variable, hvor både $n$ og $m$ er positive heltal, kaldes et lineært ligningssystem. Lineære ligningssystemer skrives på formen

\begin{align*}
a_{1,1}x_1+a_{1,2}x_2+ &\dots +a_{1,n}x_n = b_1\\
a_{2,1}x_1+a_{2,2}x_2+ &\dots +a_{2,n}x_n = b_2\\
&\vdots \\
a_{m,1}x_1+a_{m,2}x_2+ &\dots +a_{m,n}x_n = b_m
\end{align*}

Koefficienterne i et lineært ligningssystem kan skrives i en matrix, mens variable og konstanterne skrives som vektorer. Et lineært ligningsystem kan så skrives op som en matrix ligning på formen $A \vec{x} = \vec{b}$, hvor

\begin{align*}
A= \begin{bmatrix}
a_{1,1} & a_{1,2} & \dots & a_{1,n} \\
a_{2,1} & a_{2,2} & \dots & a_{2,n} \\
\vdots  &         &       & \vdots  \\
a_{m,1} & a_{m,2} & \dots & a_{m,n}
\end{bmatrix}, \ 
x= \begin{bmatrix}
x_1 \\
x_2 \\
\vdots \\
x_n
\end{bmatrix} og \ 
b= \begin{bmatrix}
b_1 \\
b_2 \\
\vdots \\
b_n
\end{bmatrix}
\end{align*}

Søjlerne i $A$ indeholder koefficienterne $x_1$, $x_2$, $\dots $, $x_n$ og kaldes derfor koefficientmatricen til det lineære ligningssystem. 
Den information, der er nødvendig for at løse et lineært ligningssystem, kan samles i en totalmatrix på formen:
\[
\left[
\begin{array}{cccc|c}
a_{1,1} & a_{1,2} & \dots & a_{1,n} & b_1 \\
a_{2,1} & a_{2,2} & \dots & a_{2,n} & b_2 \\
\vdots  &         &       &         & \vdots \\
a_{m,1} & a_{m,2} & \dots & a_{m,n} & b_n
\end{array}
\right]
\]

Totalmatricen laves ved at tilføje vektor $\vec{b}$ til koefficientmatricen $A$. Totalmatricen noteres således som $[A \ \vec{b}]$. Den lodrette streg mellem koefficienterne og konstanterne er ikke nødvendig, men kan tilføjes for at vise, at der er tale om en totalmatrix.\\

Løsningen til et lineært ligningssystem er en vektor i $\mathds{R}^n$, som ser således ud:

\begin{align*}
\begin{bmatrix}
s_1 \\
s_2 \\
\vdots \\
s_n
\end{bmatrix}
\end{align*}

Hvis $A$ er en $m \times n$ matrix, er en vektor $\vec{u}$  i $\mathds{R}^n$ en løsning til $A \vec{x}= \vec{b}$ hvis og kun hvis $A \vec{u}= \vec{b}$.

\begin{eks}
Følgende lineære ligningssystem er givet

\begin{align*}
8x_1+6x_2+4x_3 = 50 \\
2x_1+4x_2+6x_3 = 30.
\end{align*}

Dette kan så skrives i en matrix,

\begin{align*}
\begin{bmatrix}
8 & 6 & 4 & 50 \\
2 & 4 & 6 & 30
\end{bmatrix}.
\end{align*}

Følgende vektor er en løsning til dette ligningssystem.

\begin{align*}
\begin{bmatrix}
2 \\
5 \\
1
\end{bmatrix}.
\end{align*}

Sætter man løsningen ind som variable fås følgende resultat,

\begin{align*}
8 \cdot 2 + 6 \cdot 5 + 4 \cdot 1 = 50 \\
2 \cdot 2 + 4 \cdot 5 + 6 \cdot 1 = 30.
\end{align*}

Det ses, at vektoren giver en rigtig løsning til ligningssystemet, fordi alle ligningerne går op når vektoren indsættes. 

\end{eks}

\subsection{Elementære rækkeoperationer}
Når et lineært ligningssystem skal løses, skal flere trin gennemføres. Først skrives ligningssystemet op som en totalmatrix. Herefter findes matricen på trappeform og dernæst rækkereduceret trappeform ved hjælp af rækkeoperationer. De tre typer af elementære rækkeoperationer gennemgås i Definition \ref{defn_elemen_operation}. \\

\begin{defn}[Elementære rækkeoperationer]
Der findes 3 elementære rækkeoperationer
\begin{enumerate}
\item Ombytning af to rækker. (Ombytning)
\item Multiplicering af en række med en skalar. (Scaling)
\item Lægge en række ganget en skalar til en anden række. (Udskiftning)
\end{enumerate}
\label{defn_elemen_operation}
\end{defn}

\begin{eks}[Rækkeoperationer]
\textbf{Ombytning}\\
Herunder ses et eksempel på to rækker der ombyttes i en matrix.\\

\begin{align*}
A= \begin{bmatrix}
0 & 2 & 3 \\
1 & 4 & 7
\end{bmatrix}
\sim \begin{bmatrix}
1 & 4 & 7 \\
0 & 2 & 3
\end{bmatrix} = B.
\end{align*}

\textbf{Scaling} \\
Et eksempel på scaling ses nedenfor, hvor første række multipliceres med $2$.\\

\begin{align*}
A= \begin{bmatrix}
2 & 1 & 3 \\
0 & 4 & 8 \\
1 & 7 & 1
\end{bmatrix}
\sim \begin{bmatrix}
4 & 2 & 6 \\
0 & 4 & 8 \\
1 & 7 & 1
\end{bmatrix} = B.
\end{align*}

\textbf{Udskiftning}\\
Et eksempel på udskiftning ses herunder, hvor $2$ gange række 2 trækkes fra række 1.\\

\begin{align*}
A= \begin{bmatrix}
1 & 4 & 7 \\
0 & 2 & 3
\end{bmatrix}
\sim \begin{bmatrix}
1 & 0 & 1 \\
0 & 2 & 3
\end{bmatrix} = B.
\end{align*}
\end{eks}

Målet med elementære rækkeoperationer er, at opnå trappeform og derefter reduceret trappeform. En ledende indgang i en matrix, er den første indgang i en række der ikke er lig 0. 

\begin{defn}[Trappeform]
En matrix siges at være på trappeform hvis følgende tre betingelser er opfyldt. 
\begin{enumerate}
\item Enhver række forskellig fra 0, ligger over enhver nul-række.
\item Den ledende indgang i en række forskellig fra 0, ligger i en søjle til højre for søjlen med den ledende indgang i rækken ovenfor.
\item Hvis en søjle indeholder en ledende indgang, vil alle indgange under den ledende i søjlen være lig 0.
\end{enumerate}
\label{defn_trappe}
\end{defn}

De positioner i matricen, hvor der er en ledende indgang, kaldes pivot-indgange. De søjler der indeholder en pivot-indgang kaldes pivot-søjler. De variable, hvis søjler ikke har pivot, kaldes frie variable og hvis en matrix indeholder frie variable betyder det, at der er uendeligt mange løsninger, fordi resultatet ikke afhænger af de frie variable. Samtidig siges et ligningssystem at være uløseligt, hvis der er pivot i sidste søjle, da det ville betyde at der skulle opnås et resultat forskelligt fra 0, ved kun at multiplicere med 0, hvilket ikke er muligt.\\
Når en matrx er på trappeform kan der udføres flere rækkeoperationer for at få den på reduceret trappeform.

\begin{defn}
Hvis en matrix overholder betingelserne for at være på trappeform og samtidig overholder nedenstående to betingelser, siges matricen at være på reduceret trappeform.
\begin{enumerate}
\item Hvis en søjle indeholder en ledende indgang i en hvilken som helst række, skal alle andre indgange i søjlen være $0$.
\item Den ledende indgang i hver ikke-nul-række skal være $1$.
\end{enumerate}
\end{defn}

Når der skal findes en løsning på et ligningssystem skrives det først på reduceret trappeform, hvorefter hver variabel isoleres. 
Dermed findes en generel løsning i form af en vektor som beskrevet tidligere. % det er da ikke tidligere beskrevet?

\begin{eks}
Et eksempel på en matrix på trappeform ses herunder. I matricen er pivotindgangene $(1,1)$, $(2,2)$ og $(3,4)$.
\begin{align*}
A= \begin{bmatrix}
2 & 4 & 2 & 6 & 8 \\
0 & 3 & 6 & 9 & -3 \\
0 & 0 & 0 & 6 & 2
\end{bmatrix}. 
\end{align*}
Denne matrix reduceres nu ved hjælp af rækkeoperationer til reduceret trappeform.\\
\begin{align*}
\begin{bmatrix}
2 & 4 & 2 & 6 & 8 \\
0 & 3 & 6 & 9 & -3 \\
0 & 0 & 0 & 6 & 12
\end{bmatrix}
&\sim
\begin{bmatrix}
1 & 2 & 1 & 3 & 4 \\
0 & 3 & 6 & 9 & -3 \\
0 & 0 & 0 & 6 & 12
\end{bmatrix}
\sim
\begin{bmatrix}
1 & 2 & 1 & 3 & 4 \\
0 & 1 & 2 & 3 & -1 \\
0 & 0 & 0 & 1 & 2
\end{bmatrix}
\sim \\
\begin{bmatrix}
1 & 0 & -3 & -3 & 6 \\
0 & 1 & 2 & 3 & -1 \\
0 & 0 & 0 & 1 & 2
\end{bmatrix}
&\sim
\begin{bmatrix}
1 & 0 & -3 & 0 & 12 \\
0 & 1 & 2 & 3 & -1 \\
0 & 0 & 0 & 1 & 2
\end{bmatrix}
\sim
\begin{bmatrix}
1 & 0 & -3 & 0 & 12 \\
0 & 1 & 2 & 0 & -7 \\
0 & 0 & 0 & 1 & 2
\end{bmatrix}.
\end{align*}
Nu er matricen på reduceret trappeform. Som det ses, er det ligegyldigt, hvad der står i de søjler, der ikke er pivot-søjler. Dette ligningssystem har derfor uendeligt mange løsninger. Nu isoleres alle variable, så en løsning kan findes:
\begin{align*}
x_1 &= 12 + 3x_3 \\
x_2 &= -7 - 2x_3 \\
x_3 &= x_3 \\
x_4 &= 2
\end{align*}
Dette skrives så på vektorform:
\begin{align*}
\begin{bmatrix}
x_1 \\
x_2 \\
x_3 \\
x_4 
\end{bmatrix}
= \begin{bmatrix}
12 + 3x_3 \\
-7 - 2x_3 \\
x_3 \\
2
\end{bmatrix}
= \begin{bmatrix}
12 \\
-7 \\
0 \\
2
\end{bmatrix}
+ x_3 \begin{bmatrix}
3 \\
-2 \\
1 \\
0
\end{bmatrix}.
\end{align*}
\end{eks}

De variable, der ikke har en pivot-søjle, kaldes frie variable, fordi deres værdier ikke påvirker ligningssystemet. \\

I forlængelse heraf, kan man kigge på matricens \textbf{rang} og \textbf{nullitet}. 
\begin{defn}
Lad $A$ være en $m \times n$ matrix på reduceret trappeform. 
Da er \textbf{rangen} af $A$ defineret som antallet af pivot-søjler i $A$, og noteres $rang(A)$. 
\textbf{Nulliteten} af $A$ noteres $null(A)$ og er defineret som antallet af ikke-pivot-søjler i $A$, altså $n - rang(A)$. \\
Derudover gælder $rang(A) + null(A) = n$.
\end{defn}


Et ligningssystem kan siges at være lineært afhængig eller lineært uafhængig.

\begin{defn}[lineær uafhængighed]
Vektorerne $\vec{v}_1, \dots ,\vec{v}_k$ kaldes lineært uafhængige hvis ligningen
\begin{align*}
x_1 \cdot \vec{v}_1+x_2 \cdot \vec{v}_2 + \dots + x_k \cdot \vec{v}_k =  \vec{0}
\end{align*}
kun har den trivielle løsning
\begin{align*}
x_1=x_2= \dots =x_k=0
\end{align*}
ellers kaldes de lineært afhængige.
\label{defn_lin_uafh}
\end{defn}

Det kan altså siges, at et ligningssystem er lineært uafhængigt, hvis der kun er en løsning, så der ingen frie variable er. Hvis et ligningssystem har frie variable, er der uendeligt mange løsninger og kaldes derfor lineært afhængigt. 


