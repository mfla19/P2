\section{Lineære ligningssystemer}
Lineære ligninger, der indeholder ukendte variabler, kan skrives på formen

\begin{align*}
a_1x_1+a_2x_2+ \dots +a_nx_n = b,
\end{align*}

hvor $a_1, a_2, \dots , a_n$ og $b$ er reelle tal. 
Her kaldes $a_1,a_2, \dots , a_n$ koefficienter og $b$ er en konstant. En lineær ligning kunne for eksempel se sådan ud:

\begin{align*}
7x_1+3x_2-5x_3 = 10.
\end{align*}

Lineære ligninger må ikke indeholde to variable multipliceret, kvadratroden af en variabel, eller andet der gør den ikke-lineær. \\
Et sæt af $m$ lineære ligninger, der indeholder de samme $n$ variable, hvor både $n$ og $m$ er positive heltal, kaldes et lineært ligningssystem. Lineære ligningssystemer skrives på formen

\begin{align*}
a_{11}x_1+a_{12}x_2+ &\dots +a_{1n}x_n = b_1\\
a_{21}x_1+a_{22}x_2+ &\dots +a_{2n}x_n = b_2\\
&\vdots \\
a_{m1}x_1+a_{m2}x_2+ &\dots +a_{mn}x_n = b_m
\end{align*}

