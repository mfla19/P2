\chapter{Lineær algebra}
Dette kapitel er skrevet med udgangspunkt i \citep{lial}

Lineær algebra bruges til at løse lineære ligningssystemer. 
En af de mest nyttige algoritmer, til at løse sådanne ligningssystemer, er Gaussisk elimination, som vil blive beskrevet senere i dette kapitel.

\section{Matricer}
Lineære ligningssystemer kan opskrives i matricer. 
En matrix er defineret i Definition \ref{def:matricer}.

\begin{defn} [Matrix]
En matrix er en rektangulær tabel over skalarer. 
Størrelsen på en matrix er $m \times n$, hvor $m$ er antal rækker og $n$ er antal søjler. 
En matrix kaldes kvadratisk hvis $m=n$. 
Skalaren i den $i$'te række og $j$'te søjle kaldes $(i,j)$-indgangen.
\label{def:matricer}
\end{defn}

Herunder ses en matrix $\underset{m \times n}{A}$ og dens indgange, $a_{i,j}$. Den $i$'te række i A betenges $\vec{a}_i$, mens den $j$'te søjle angives $\vec{A}_j$.

\begin{align*}
\underset{m \times n}{A} = \begin{bmatrix}
	a_{1,1} & a_{1,2} & \dots & a_{1,j} & \dots & a_{1,n} \\
	a_{2,1} & \ddots  &       &         &       & \vdots \\
	\vdots  &         & \ddots &        &       & \vdots \\
	a_{i,1} &         &       & a_{i,j} &       & \vdots \\
	\vdots  &         &       &         & \ddots& \vdots \\
	a_{m,1} & \dots   & \dots & \dots   & \dots & a_{m,n} 
\end{bmatrix}
\end{align*}

Matricer kan bruges til at vise mange forskellige ting fra den virkelige verden, det kan være antal af varer solgt fra forskellige butikker, eller prisen på forskellige produkter. 

\begin{eks}
Herunder ses en $4 \times 3$-matrix, der viser antallet af solgte varer fra tre forskellige butikker. Hver række i matricen repræsenterer en bestemt vare, mens hver søjle repræsenterer en butik. Den første række viser altså hvor mange trøjer hver butik har solgt.
\begin{align*}
\begin{matrix}
	Trøjer \\
	Kjoler \\
	Bukser \\
	Jakker
\end{matrix}
\begin{bmatrix}
	24 & 35 & 43 \\
	10 & 47 & 24 \\
	33 & 25 & 32 \\
	28 & 51 & 37
\end{bmatrix}
\end{align*}
I denne matrix kan man se, at der i dens $(2,3)$-indgang står $24$, hvilket betyder, at den tredje butik har solgt $24$ kjoler. Der står i dens $(3,1)$-indgang $33$, hvilket betyder, at den første butik har solgt $33$ par bukser.
\end{eks}

\begin{defn}[Transponeret matrix]
Lad $A$ være en $m \times n$ matrix. Så er den transponerede matrix, $A^T$, en $n \times m$ matrix, hvor hver indgang $(j,i)$ i $A^T$, er den $(i,j)$'te indgang i $A$
\label{def:(transmatrix)} 
\end{defn}
Det vil sige at rækkerne i $A$, bliver til søjlerne i $A^T$, og at søjlerne i $A$ bliver til rækkerne i $A^T$.

\begin{eks}
Givet en matrix $A$ 	bestemmes nu den transponerede
\begin{align*}
A = \begin{bmatrix}
	5 & 3 & 3 \\
	1 & 2 & 5
\end{bmatrix}
\end{align*}

\begin{align*}
A^T = \begin{bmatrix}
	5 & 1  \\
	3 & 2  \\
	3 & 5
\end{bmatrix}
\end{align*}
Det ses at $A$ er en $2 \times 3$ matrix og $A^T$ er en $3 \times 2$ matrix. 
\end{eks}


En bestemt type af matrix er en identitetsmatrix, der betegnes $I_n$. 

\begin{defn} [Identitetsmatrix]
For hvert positivt heltal $n$, er $n \times n$ identitetsmatricen, $I_n$, den $n \times n$ matrix, hvor hver søjle er standardvektorerne $\vec{e_1}$, $\vec{e_2}$, $\dots$, $\vec{e_n}$ i $\mathds{R}^n$.\\

alternativ definition:\\
For hvert positivt heltal $n$, er $n \times n$ identitetsmatricen, $I_n$, defineret som\\ $I_n = \rvect{\vec{e_1} & \vec{e_2} & \dots &  \vec{e_n}}$, hvor $\vec{e_1}$, $\vec{e_2}$, $\dots$, $\vec{e_n}$ er standardvektorerne i $\mathds{R}^n$
\label{def:imatrix}
\end{defn}

En identitetsmatrix med $3$ rækker og $3$ søjler, ser således ud:
\begin{align*}
I_3 = \begin{bmatrix}
	1 & 0 & 0 \\
	0 & 1 & 0 \\
	0 & 0 & 1 
\end{bmatrix}
\end{align*}

En anden bestemt type matrix er en delmatrix, som er en form for undermatrix. 
\begin{defn} [Delmatrix]
En matrix $A'$ er en delmatrix af $A$, hvis $A'$ kan dannes ved at fjerne hele søjler og/eller hele rækker fra $A$.
\label{delmatrix}
\end{defn}

Givet en matrix $A$ kan der dannes delmatricen $A'$.

\begin{align*}
A &= \begin{bmatrix}
	1 & 2 & 3 \\
	4 & 5 & 6 \\
	7 & 8 & 9 
\end{bmatrix}\\
A' &= \begin{bmatrix}
	5 & 6 \\
	8 & 9
\end{bmatrix}
\end{align*}

\section{Vektorer}
En vektor er en matrix, der enten kun har en søjle eller en række. Vektorer kan også repræsenteres geometrisk hvor en vektor er en pil med en retning og en længde.
Det vil sige, at en vektor er et særtilfælde af en matrix, og alle regneregler for matricer, gør sig derfor gældende for vektorer, men ikke alle regneregler for vektorer kan bruges for matricer.
En vektor med kun én søjle kaldes en søjlevektor, mens en vektor med kun én række kaldes en rækkevektor.
Vektorer anvendes blandt andet til at repræsentere rækker og søjler i matricer. 
I dette tilfælde er vektorerne derved delmatricer af den originale matrix.
Indgangene i en vektor kaldes vektorens vektorkomponenter, og svarer til vektorens koordinater i den geometriske repræsentation.

\begin{defn}[Søjlevektor]
Lad $A$ være en $m \times n$ matrix, så vil en \textbf{søjlevektor} være en delmatrix med kun en søjle, og med alle søjlens rækkekomponenter.
\begin{align*}
\vec{v}=\vec{A}_j = 
\begin{bmatrix}
a_{1,j}\\
a_{2,j}\\
\vdots \\
a_{m,j} \\
\end{bmatrix},\qquad  i\in \{1,...,n\}
\end{align*}
Vektoren $\vec{v}$ siges at have dimension $m$ og tilhører  $\mathds{R}^m$.
\end{defn}

På samme måde kan en rækkevektor defineres


\begin{defn}[Rækkevektor]
Lad $A$ være en $m \times n$ matrix, så vil en \textbf{rækkevektor} $\vec{v}$ være en delmatrix med kun en række og med alle rækkens søjlekomponenter.
\begin{align*}
\vec{v}=\vec{a}_i = 
\rvect{a_{i,1} & a_{i,2} & \dots & a_{i,n}},
\qquad  j\in \{1,...,m\}
\end{align*}
Vektoren $\vec{v}$ siges at have dimension $n$ og tilhører $\mathds{R}^n$.
\end{defn}
Bemærk at hvis henholdsvis $m$ eller $n$ er $1$, så er matricen i forvejen en vektor.\\

\subsection{Standard basis vektorer}
Standard basis vektorer er vektorer, hvis koordinater alle er $0$ pånær én, der har længden $1$ ud af aksen i det rum, $\mathds{R}^n$, vektoren er i. Standard basis vektorer i $\mathds{R}^n$ skrives som $\vec{e_i}$, hvor $i\in \{1,..,n\}$, hvor $i$ er den vektorkomponent, som har værdien $1$. 

\begin{defn}[Standard basis vektorer]
Lad $\vec{e}_{i}\in\mathds{R}^n$ være en vektor hvis $j$'te indgang er givet ved
\begin{align*}
\vec{e}_{i_{j}}=\begin{cases} 0, \quad j \neq i
\\ 1 , \quad j = i \end{cases}.
\end{align*}

\end{defn}



\section{Regneoperationer med matricer}
Givet to $m \times n$ matricer $A$ og $B$ kan der udføres forskellige regneoperationer. 
Summen af to matricer findes ved at addere en indgang i $A$ med tilsvarende indgang i $B$, så $A+B$ er en $m \times n$ matrix, hvor indgang $(i,j)$ er $a_{i,j}+b_{i,j}$. 
Det samme gør sig gældende ved substraktion. \\
Givet en $m \times n$ matrix $A$ og en skalar $c$, er produktet af skalaren og matricen, $cA$, en $m \times n$ matrix, hvor indgangene er $c$ gange den tilsvarende indgang i $A$. \\

\begin{stn}
Lad $A$, $B$ og $C$ være $m \times n$ matricer, og lad $s$ og $t$ være tilfældige skalarer. 
Så gælder følgende
\begin{enumerate}[label=(\alph*)]
\item $A + B = B + A$
\item $(A + B) + C = A + (B + C)$
\item $A + O = A$
\item $A + (-A) = O$
\item $(st) A = s (tA)$
\item $s(A + B) = sA + sB$
\item $(s+t)A = sA + tA$
\end{enumerate}
\label{stn_regn}
\end{stn}

\begin{proof} 
Lad $A$, $B$ og $C$ være $m \times n$ matricer. Lad matricen $O$ være en nulmatrix. \\
(a) Det vises, at enhver indgang i $A + B$ er tilsvarende i $B + A$. Betragt indgangene $a_{i,j}$ og $b_{i,j}$. Summen af $a_{i,j} + b_{i,j}$ er det samme som $b_{i,j} + a_{i,j}$. \\
(b) Det vises, at enhver indgang i $(A + B) + C$ er den samme som den tilsvarende indgang i $A + (B + C)$. Ligesom i (a) tages der udgangspunkt i indgang $(i,j)$. Summen af $(a_{i,j} + b_{i,j}) + c_{i,j}$ er det samme som $a_{i,j} + (b_{i,j} + c_{i,j})$. Ifølge den associative lov, er det uden betydning hvor paranteserne er placeret i et additionsudtryk. Derfor må indgangen $(i,j)$ i $(A + B) + C$ være lig indgang $(i,j)$ i $A + (B + C)$. \\
(c) For enhver indgang i $A$, $a_{i,j}$ skal denne adderes med $0$. $a_{i,j}+0=a_{i,j}$. Derfor må $A + O = A$. \\
(d) For enhver indgang i $A$, $a_{i,j}$ skal denne fratrækkes samme indgang i $A$, $a_{i,j}$. Da \\ $a_{i,j} - a_{i,j} = 0$, må $A + (-A) = O$. \\
(e) Her tages der udgangspunkt i den (i,j)-indgang af A, $a_{i,j}$. Det ses så, at udsagnet bliver til $(st)a_{i,j} = s(ta_{i,j})$ når flere tal multipliceres, er det lige meget i hvilken rækkefølge, dermed er (e) bevist. \\
(f) Hver indgang i A, $a_{i,j}$ lægges sammen med den tilsvarende indgang i B, $b_{i,j}$, dette giver venstresiden $s(a_{i,j}+b_{i,j})$. Da det er tilladt at multiplicere ind i en parantes, kan udtrykket skrives sådan, $s(a_{i,j}+b_{i,j})=sa_{i,j}+sb_{i,j}$, dermed er (f) bevist. \\
(g) På samme måde som før tages der udgangspunkt i den (i,j)-indgang af A, $a_{i,j}$. Udtrykket på venstresiden er så $(s+t)a_{i,j}$. Igen kan der multipliceres ind i parantesen og dermed fås følgende, $(s+t)a_{i,j}=sa_{i,j}+ta_{i,j}$, hvilket beviser (g).
\end{proof}

\begin{eks}
For at vise eksempler på nogle af de ovenstående regneoperationer, tages der udgangspunkt i matricerne A og B, samt skalaren s, hvor $s(A+B)$ skal findes.
\begin{align*}
A= \begin{bmatrix}
	2 & 3 & 4 \\
	5 & -2 & 1 	
\end{bmatrix},  
B= \begin{bmatrix}
	1 & 2 & -1 \\
	-3 & 4 & 0
\end{bmatrix},
s=2
\end{align*}
Først udføres regneoperation (a) fra Sætning \ref{stn_regn},
\begin{align*}
A+B= \begin{bmatrix}
	2 & 3 & 4 \\
	5 & -2 & 1 	
\end{bmatrix}  
+ \begin{bmatrix}
	1 & 2 & -1 \\
	-3 & 4 & 0
\end{bmatrix}
= \begin{bmatrix}
	3 & 5 & 3 \\
	2 & 2 & 1
\end{bmatrix}.
\end{align*}
Herfter udføres (f),
\begin{align*}
s(A+B)= 5 \times \left( \begin{bmatrix}
	2 & 3 & 4 \\
	5 & -2 & 1 	
\end{bmatrix}  
+ \begin{bmatrix}
	1 & 2 & -1 \\
	-3 & 4 & 0
\end{bmatrix} \right)
= 5 \times \begin{bmatrix}
	3 & 5 & 3 \\
	2 & 2 & 1
\end{bmatrix}
= \begin{bmatrix}
	15 & 25 & 15 \\
	10 & 10 & 5
\end{bmatrix}.
\end{align*}
\end{eks}


\subsection{Matrix produkt}
Multiplikation af to matricer gøres ikke ved af multiplicere på tilsvarende indegange i to matricer. Det at gange to matricer sammen defineres herunder. 
\begin{defn} [Matrix produkt]
Lad $A$ være en $m \times n$ matrix og $B$ være en $n \times p$ matrix. Da er en produktet $A \cdot B$ en $m \times p$ matrix, hvor indgangene er givet ved: 

$$ab_{i,j} = \vec{a}_i \cdot \vec{B}_j$$

hvor $\vec{a}_i$ er den $i$'te række i $A$, og $\vec{B}_j$ er den $j$'te søjle i $B$
\label{def:(matrixprodukt)}
\end{defn}
Det er vigtigt at antallet af søjler i $A$ er det samme som antallet af rækker i $B$. Er det ikke tilfældet, så er matrix produktet ikke defineret. Matrix produktet er oftest ikke kommutativt. Ligningen $AB=AB$ er derfor ikke altid gældende. 
\begin{eks}
Nu tages udgangspunkt i to matricer $A$ og $B$. 
\begin{align*}
\underset{2 \times 3}{A}= \begin{bmatrix}
	\bf{1} & \bf{3} & \bf{-2} \\
	5 & 4 & 0 	
\end{bmatrix},
\underset{3 \times 4}{B}= \begin{bmatrix}
	\bf{2} & 3 & -1 & 3 \\
	\bf{1} & 4 & 5 & 5\\
	\bf{1} & 0 & 4 & 2
\end{bmatrix}  
\end{align*}
For at beregne indgang $(1,1)$ benyttes række 1 i $A$ og søjle 1 i $B$. 
$$ab_{1,1}=1\cdot 2+3\cdot 1-2 \cdot 1 = 3$$ 
De resterende indgange beregens på samme måde. 
Matrix produkt af $A$ og $B$ er dermed givet ved:
\begin{align*}
\underset{2 \times 4}{AB}= \begin{bmatrix}
	\bf{3} & 15 & 6 & 14 \\
	14 & 31 & 15 & 35
\end{bmatrix}  
\end{align*}
\end{eks}

\section{Lineære ligningssystemer}
Lineære ligninger, der indeholder ukendte variabler, kan skrives på formen

\begin{align*}
a_1x_1+a_2x_2+ \dots +a_nx_n = b,
\end{align*}

hvor $a_1, a_2, \dots , a_n$ og $b$ er reelle tal. 
Her kaldes $a_1,a_2, \dots , a_n$ koefficienter og $b$ er en konstant. En lineær ligning kunne for eksempel se sådan ud:

\begin{align*}
7x_1+3x_2-5x_3 = 10.
\end{align*}

Lineære ligninger må ikke indeholde to variable multipliceret, kvadratroden af en variabel, eller andet der gør den ikke-lineær. \\
Et sæt af $m$ lineære ligninger, der indeholder de samme $n$ variable, hvor både $n$ og $m$ er positive heltal, kaldes et lineært ligningssystem. Lineære ligningssystemer skrives på formen

\begin{align*}
a_{1,1}x_1+a_{1,2}x_2+ &\dots +a_{1,n}x_n = b_1\\
a_{2,1}x_1+a_{2,2}x_2+ &\dots +a_{2,n}x_n = b_2\\
&\vdots \\
a_{m,1}x_1+a_{m,2}x_2+ &\dots +a_{m,n}x_n = b_m
\end{align*}

Koefficienterne i et lineært ligningssystem kan skrives i en matrix, mens variable og konstanterne skrives som vektorer. Et lineært ligningsystem kan så skrives op som en matrix ligning på formen $A \vec{x} = \vec{b}$, hvor

\begin{align*}
A= \begin{bmatrix}
a_{1,1} & a_{1,2} & \dots & a_{1,n} \\
a_{2,1} & a_{2,2} & \dots & a_{2,n} \\
\vdots  &         &       & \vdots  \\
a_{m,1} & a_{m,2} & \dots & a_{m,n}
\end{bmatrix}, \ 
x= \begin{bmatrix}
x_1 \\
x_2 \\
\vdots \\
x_n
\end{bmatrix} og \ 
b= \begin{bmatrix}
b_1 \\
b_2 \\
\vdots \\
b_n
\end{bmatrix}
\end{align*}

Søjlerne i $A$ indeholder koefficienterne $x_1$, $x_2$, $\dots $, $x_n$ og kaldes derfor koefficientmatricen til det lineære ligningssystem. 
Den information, der er nødvendig for at løse et lineært ligningssystem, kan samles i en totalmatrix på formen:
\[
\left[
\begin{array}{cccc|c}
a_{1,1} & a_{1,2} & \dots & a_{1,n} & b_1 \\
a_{2,1} & a_{2,2} & \dots & a_{2,n} & b_2 \\
\vdots  &         &       &         & \vdots \\
a_{m,1} & a_{m,2} & \dots & a_{m,n} & b_n
\end{array}
\right]
\]

Totalmatricen laves ved at tilføje vektor $\vec{b}$ til koefficientmatricen $A$. Totalmatricen noteres således som $[A \ \vec{b}]$. \\

Løsningen til et lineært ligningssystem er en vektor i $\mathds{R}^n$, som ser således ud:

\begin{align*}
\begin{bmatrix}
s_1 \\
s_2 \\
\vdots \\
s_n
\end{bmatrix}
\end{align*}

Hvis $A$ er en $m \times n$ matrix, er en vektor $\vec{u}$  i $\mathds{R}^n$ en løsning til $A \vec{x}= \vec{b}$ hvis og kun hvis $A \vec{u}= \vec{b}$.

\begin{eks}
Følgende lineære ligningssystem er givet

\begin{align*}
8x_1+6x_2+4x_3 = 50 \\
2x_1+4x_2+6x_3 = 30.
\end{align*}

Dette kan så skrives i en matrix,

\begin{align*}
\begin{bmatrix}
8 & 6 & 4 & 50 \\
2 & 4 & 6 & 30
\end{bmatrix}.
\end{align*}

Løsningen til dette ligningssystem bliver følgende vektor

\begin{align*}
\begin{bmatrix}
2 \\
5 \\
1
\end{bmatrix}.
\end{align*}

Sætter man løsningen ind som variable fås følgende resultat,

\begin{align*}
8 \cdot 2 + 6 \cdot 5 + 4 \cdot 1 = 50 \\
2 \cdot 2 + 4 \cdot 5 + 6 \cdot 1 = 30.
\end{align*}

Det ses, at vektoren giver en rigtig løsning til ligningssystemet, fordi alle ligningerne går op når vektoren indsættes. 

\end{eks}

\subsection{Invers matrix}

Hvis matrix produktet af to matricer, A og B, giver identitetsmatricen, sigen B matrix at være den inverse matrix til A. 
\begin{defn}[Invers matrix]
Lad $A$ og $B$ være kvadratiske $n \times n$ matricer. Lad $AB=BA=I_n$, hvor $I_n$ er identitetsmatricen. Så er A invertibel og B er den inverse matrix til $A$. $B$ noteres da $A^{-1}$. 
\label{def(inversmatrix)}
\end{defn}
Den inverse til matrix til $A$, er en entydig matrix. Forestiller man sig at der findes to inverse matricer til $A$, $B$ og $C$. Vil de to være den samme da: 
\begin{align*}
B=BI_n=B(AC)=(BA)C=I_nC=C
\end{align*}
Den inverse matrix kan bruges til at løse specifikke ligningssystemer. Lad $A$ være en $n \times n$ matrix, og $\vec{b}$ være en vektor i $\mathds{R}^n$. Betragt ligningssystemet: 
\begin{align*}
A \vec{x} &= \vec{b}\\
A^{-1} A \vec{x} &= A^{-1} \vec{b}\\
I_n \vec{x} &= A^{-1} \vec{b}.
\end{align*} 
Ved at gange den inverse matrix på begge sider af lighedstegnet, opnåes et udtryk for den entydige løsning til ligningssystemet: 
\begin{align}
\vec{x} &= A^{-1} \vec{b}.
\end{align} 

\begin{stn}
Lad $A$ være en $n \times n$ matrix. Der gælder at: 
\begin{enumerate}[label=(\alph*)]
\item $A$ er invertibel hvis og kun hvis den reducerede trappeform af $A$ er identitesmatricen, $I_n$
\item Hvis $A$ er invertibel, så vil de samme elementære rækkeoperationer som reducerer $A$ til $I_n$ føre $I_n$ over i $A^{-1}$.  
\end{enumerate}
\begin{align*}
\begin{bmatrix}
A & I_n
\end{bmatrix} \sim \dots \sim
\begin{bmatrix}
I_n & A^{-1}
\end{bmatrix}
\end{align*}
\label{stn:inversmatrix}
\end{stn}

\begin{proof}
(a) Først lad $A$ være invertibel. Betragt nu en vektor $\vec{x}$ i $\mathds{R}^n$, som opfylder $A\vec{x}=\vec{0}$. Så fåes det at ligning $(4.1)$, at $\vec{x}=A^{-1} \vec{0}=\vec{0}$. Da løsningen af  $A\vec{x}=\vec{0}$ er $\vec{0}$ må $rang(A)=n$. Det betydet at antallet af pivot-søjler er det samme som antellet af søjler, samt antallet af rækker. Den reducerede trappeform af $A$ må derfor være identitetsmatricen. 

Lad nu den reducerede trappeform af $A$ være identitesmatricen. Så må der findes en invertibel $n \times n$ matrix $P$ således at: $PA=I_n$
\begin{align*}
A=I_nA=(P^{-1}P)A=P^{-1}(PA)=P^{-1}I_n=P^{-1}
\end{align*}
Da $P$ er invertibel må $P{-1}$ også være invertibel. Derfor er $A$ invertibel. 

(b) Betragt matricen $[A \quad I_n]$. Enhver $n \times n$ matrix, $A$, kan skrives som en matrix, $R$, på reduceret trappeform, ved hjælp af elementære rækkeoperationer. Udføres de samme rækkeoperationer på $n \times 2n$ matricen, $[A \quad I_n]$, tranformeres denne til $[R \quad B]$. Betragt en invertibel matrix $P$, som opfyldet at: $P[A \quad I_n]=[R \quad B]$. Så følger det at:  
\begin{align*}
[R \quad B]=P[A \quad I_n]=[PA \quad PI_n]=[PA \quad P]. 
\end{align*}
Så er $PA=R$ og $P=B$. Hvis $R \neq I_n$ så er $A$ ikke invertibel. Derfor må $R=I_n$, og dermed er $PA=I_n\Leftrightarrow P=A^{-1}$. P er da lig med både $A^{-1}$ og $B$, og $A^{-1}=B$. 
$$[A \quad I_n]\sim[R \quad B]=[PA \quad P]=[I_n \quad A^{-1}]$$
\end{proof}

\begin{eks}
Givet en matrix A ønskes det nu at finde en inverse. 
\begin{align*}
A= \begin{bmatrix}
1 & 3 \\
1 & 2
\end{bmatrix}
\end{align*}
Først opstilles $[A \quad I_n]$ matricen: 
\begin{align*}
\begin{bmatrix}
1 & 3 & 1 & 0 \\
1 & 2 & 0 & 1
\end{bmatrix}.
\end{align*}
Nu udførers elementære rækkeoperationer således at $A$ transformeres til $I_n$. 
\begin{align*}
\begin{bmatrix}
1 & 3 & 1 & 0 \\
1 & 2 & 0 & 1
\end{bmatrix}.
\sim \begin{bmatrix}
1 & 3 & 1 & 0 \\
0 & -1 & -1 & 1
\end{bmatrix}.
\sim \begin{bmatrix}
1 & 3 & 1 & 0 \\
0 & 1 & 1 & -1
\end{bmatrix}.
\sim \begin{bmatrix}
1 & 0 & -2 & 3 \\
0 & 1 & 1 & -1
\end{bmatrix}.
\end{align*}
Det er nu muligt at aflæse $A^{-1}$:
\begin{align*}
A^{-1} =\begin{bmatrix}
-2 & 3\\
1 & -1 
\end{bmatrix}.
\end{align*}
\end{eks}

\subsection{Determinant}
Determinanten af en matrix er en scalar, der blandt andet kan bruges til at finde ud af om en matrix er invertibel eller ej.
Først vises, hvordan determinanten af en $2 \times 2$-matrix findes.

\begin{defn}
For en matrix $A$, 
\begin{align*}
A= \begin{bmatrix}
a & c \\
b & d
\end{bmatrix},
\end{align*}
er determinanten defineret ved $det(A)=a \cdot d - b \cdot c$.
\end{defn}

Det er altså relativt nemt at finde determinanten for en $2 \times 2$-matrix. 
Når determinanten så er fundet, kan denne bruges til at finde ud af, om matricen er invertibel. 
Samtidig kan determinanten også bruges til at finde den inverse matrix. 

\begin{stn}
$A$ er invertibel hvis og kun hvis $det(A) \neq 0$ og $A^{-1} = \frac{1}{det(A)} \cdot \begin{bmatrix}
d & -c \\
-b & a
\end{bmatrix}$.
\end{stn}

Determinanten kan findes for alle $N \times N$-matricer ved brug af undermatricer.

\begin{defn}
$A_{ij} =$ matricen $A$, men uden $i$'te række og $j$'te søjle. 
\end{defn}

\begin{eks}
Her ses et eksempel på en undermatrix
\begin{align}
A=\begin{bmatrix}
a & d & g \\
b & e & h \\
c & f & i
\end{bmatrix}
\: A_{12}=\begin{bmatrix}
d & g \\
f & i
\end{bmatrix}.
\end{align}
\end{eks}

\begin{defn}
For $\underset{N \times N}{A}=(a_{ij})$ defineres 
\begin{align*}
det(A)=a_{1,1} \cdot (-1)^{1+1} \cdot det(A_{1,1}) + a_{1,2} \cdot (-1)^{1+2} \cdot det(A_{1,2}) + \dots + a_{1,N} \cdot (-1)^{1+N} \cdot det(A_{1,N}).
\end{align*}
Dette kan også skrives, 
\begin{align*}
det(A)=\sum_{j=1}^{N} a_{1,j} \cdot (-1)^{1+j} \cdot det(A_{1,j}).
\end{align*}
\end{defn}

\textbf{mangler bevis}

For $N \times N$ matricer kan det være omfattende at finde determinanten. Det kræver $N!$ multiplikationer at berenge determinanten af en $N \times N$-matrix. I en $2 \times 2$-matrix kræver det to multiplikationer, i en $3 \times 3$-matrix kræver det 6, fordi alle indgang i den valgte række eller søjle ganges med den tilhørende undermatrix, dette giver $3 \cdot 2$ multiplikationer. Tilføjes endnu en række og søjle så,  ganges $4$ på de tidligere $6$ multiplikationer og det samlede antal bliver så $24$ og sådan fortsætter det.\\

\begin{stn}
Når determinanten i en $N \times N$-matrix beregnes, kan det frit vælges hvilken række eller søjle der tages udgangspunkt i. Tages der udgangspunkt i den $i$'te række ser formlen sådan ud;
\begin{align*}
det(A)=\sum_{j=1}^{N}a_{i,j} \cdot (-1)^{i+j} \cdot det(A_{i,j}),
\end{align*}
her er $i$ konstant. Tages der udgangspunkt i den $j$'te søjle ser formlen sådan ud;
\begin{align*}
det(A)=\sum_{i=1}^{N}a_{i,j} \cdot (-1)^{i+j} \cdot det(A_{i,j}),
\end{align*}
her er $j$ konstant.
\end{stn}

Da det kan være meget omfattende at beregne determinanten for en stor matrix, kan det være smart, at tage udgangspunkt i en række eller søjle med mange nuller.
Dette betyder at flere led i udregningen går ud og dermed formindskes antallet af multiplikationer. 

\begin{eks}
Givet en matrix A,
\begin{align*}
A=\begin{bmatrix}
2 & 3 & 0 & 1 \\
0 & 4 & 7 & 2 \\
1 & 0 & 0 & 3 \\
0 & 5 & 0 & -1
\end{bmatrix}
\end{align*}
findes determinanten ved hjælp af den række eller søjle med flest nuller, i dette tilfælde er det den tredje søjle.
\begin{align*}
det(A)&= 7 \cdot (-1)^{2+3} \cdot det \left( \begin{bmatrix}
2 & 3 & 1 \\
1 & 0 & 3 \\
0 & 5 & -1
\end{bmatrix} \right)+0+0+0 \\
&= -7 \cdot \left( 1 \cdot (-1)^{2+1} \cdot det \left(
\begin{bmatrix}
3 & 1 \\
5 & -1
\end{bmatrix} \right) + 0 + 3 \cdot (-1)^{2+3} \cdot det \left( 
\begin{bmatrix}
2 & 3 \\
0 & 5
\end{bmatrix} \right) \right) \\
&= -7 \cdot (-(3 \cdot (-1)-5 \cdot 1)-3 \cdot(2 \cdot 5 - 0 \cdot 3)) \\
&= -7 \cdot (8-30) \\
&= -7 \cdot (-22) \\
&= 154
\end{align*}
Determinanten af $A$ er altså $154$.
\end{eks}

Specielt for matricer på trappeform er det nemt at finde determinanten, da man her altid vil kunne vælge en søjle der kun har 1 indgang der er forskellig fra nul. 
Dette betyder at det er muligt, bare at gange alle indgange i diagonalen for at finde determinanten. 
Determinanten af en matrix kan dog godt ændre sig hvis der udføres rækkeoperationer, derfor er det ikke muligt bare at reducere en matrice til trappeform og så finde determinanten. 
De elementære rækkeoperationer påvirker determinanten på forskellige måder. 

\begin{stn}
\begin{itemize}
\item \textbf{Ombytning:} Ændrer fortegn på determinanten. 
\item \textbf{Scaling:} Ganger determinanten med k (det k der ganges på den række scalingen udføres på).
\item \textbf{Udskiftning:} Dette ændrer ikke determinanten.
\end{itemize}
\end{stn}

\begin{proof}
\textbf{ikke færdigt bevis}
Lad $A$ være en $N \times N$-matrix, med rækkerne $\vec{a_1}, \vec{a_2}, \dots , \vec{a_N}$. \\
Først bevises (a). 
Det vises først
Så bevises (b). 
Lad $B$ være der fås ved at gange hver indgang i en række $p$ i A med en skalar $k$. 
Da fås 
\begin{align*}
det(B)=\sum_{j=1}^{N}k \cdot a_{i,j} \cdot (-1)^{i+j} \cdot det(A_{i,j}),
\end{align*}
hvilket er det samme som 
\begin{align*}
det(B)=k \cdot \sum_{j=1}^{N}a_{i,j} \cdot (-1)^{i+j} \cdot det(A_{i,j}).
\end{align*}
determinanten for $A$ er
\begin{align*}
det(A)=\sum_{j=1}^{N}a_{i,j} \cdot (-1)^{i+j} \cdot det(A_{i,j})
\end{align*}
udfra det ses det, at determinanten af $B$ opnås ved at gange determinanten af $A$ med $k$, dermed gælder det, at
\begin{align*}
det(B)=k \cdot det(A)
\end{align*}
Til sidst bevises (c). 
Lad $B$ være den matrix der fås ved at bytte rækkerne $p$ og $q$, hvor $p<q$. 
Lad nu $q=p+1$, så er $a_{p,q}=b_{q,p}$ og $A_{p,q}=B_{q,p}$
\end{proof}

\begin{stn}
Lad $A$ og $B$ være to kvadratiske matricer af samme størrelse, så gælder:
\begin{enumerate}
\item A er invertibel $\Leftrightarrow$ $det(A) \neq 0$
\item $det(A \cdot B) = det(A) \cdot det(B)$
\item $det(A^T)=det(A)$
\item $det(A^{-1})=\frac{1}{det(A)}$
\end{enumerate}
\end{stn}

\begin{proof}
\textbf{ikke færdigt bevis}
Først bevises (1), (2) og (3) hvis $A$ er invertibel.\\
\end{proof}
Rummet $\mathds{R}^n$ er mængden af vektorer af dimension $n$, og på samme måde som med mængder generelt, er det muligt at betragte delmængder af $\mathds{R}^n$.
Et special tilfælde af disse delmængder er et underrum.
\begin{defn}[Underrum]
Lad $W$ være en mængde af vektorer $\vec{v_1},...,\vec{v_k} \in \mathds{R}^n$, da er $W$  \textbf{underrum} til $\mathds{R}^n$, hvis:
\begin{enumerate}[label=\alph*]
\item $\vec{0} \in W$
\item $\vec{u}+\vec{v} \in W \quad \forall \vec{u}, \vec{v} \in W$
\item $c \cdot \vec{v} \in W \quad \forall \vec{v} \in W, \forall c \in \mathds{R}$
\end{enumerate}
\label{def:underrum}
\end{defn}
En anden delmængde er et span.
\begin{defn}[Span]
Lad $S=\{\vec{v_1},...,\vec{v_k}\}$ være en ikke tom mængde af vektorer, hvor $\vec{v_i} \in \mathds{R}^n$ for $i = 1,..,k$. 
Da er \textbf{spannet af $S$} mængden af vektorer
\begin{align*}
span(S) = \{\vec{u} \mid \vec{u}=\sum_{i=0}^k c_i \vec{v_i}, \quad \vec{v_i} \in S, \quad c_i \in \mathds{R}\}.
\end{align*} 
og $S$ siges at udspænde dets span.
\label{def:span}
\end{defn}
Det vises nu, at spannet opfylder kravene for et underrum.
\begin{stn}[Span er et underrum]
Lad $S=\{\vec{v_1},...,\vec{v_k}\} \subseteq \mathds{R}^n$, da er $span(S)$ et underrum til $\mathds{R}^n$
\label{stn:spanunderrum}
\end{stn}
\begin{proof}
For at vise at $span(S)$ er et underrum til $\mathds{R}^n$ skal det vises, at $span(S)$ overholder alle betingelserne i Definition \ref{def:underrum}.
Først vises betingelse (b), lad  derfor $\vec{u}, \vec{w} \in span(S)$, da vil 
\begin{align*}
\vec{u}+\vec{w}= \sum_{i=1}^k c_i \vec{v_i} + \sum_{i=1}^k c'_i \vec{v_i} = \sum_{i=1} c_i\cdot c_i' \vec{v_i},
\end{align*}
hvor $c_i, c_i'$ er skalare.
Dermed er $\vec{u}+\vec{w}$ en linearkombination af $\vec{v_1},...,\vec{v_k}$, hvorfor $\vec{u}+\vec{v} \in span(S)$, og $span(S)$ er lukket under vektor addition.
\\ Så vises, at $span(S)$ er lukket under skalar multiplikation, lad derfor $c, c_i$ være skalare, da vil
\begin{align*}
c\vec{w}= c\sum_{i=1}^k c_i \vec{v_i}  = \sum_{i=1} c \cdot c_i \vec{v_i}.
\end{align*}
Hvorfor $c\vec{w} \in span(S)$, og betingelse (c), er opfyldt.
\\Til sidst vises det, at $\vec{0} \in span(S)$.
Da nulvektoren kan skrives som linearkombinationen af vektorene i $S$; $\vec{0} = \sum_{i=1}^k 0 \vec{v_i}$, medfører det, at $\vec{0} \in span(S)$, hvorfor $span(S)$ overholder betingelse (a), og dermed er $span(S)$ et underrum til $\mathds{R}^n$.
\end{proof}
Det betyder, at spannet er et underrum udspændt af alle de mulige lineære kombinationer af en mængde vektorer. 
Derfor må to mængder vektorer, som er lineært afhængige af hinanden, udspænde samme underrum.
\begin{stn}[Ækvivalente span]
Lad $S = \{\vec{v_1},...,\vec{v_k}\}$ og $S_u = \{\vec{v_1},...,\vec{v_k}, \vec{u}\}$ være mængder af vektorer i $\mathds{R}^n$, da $span(S) = span(S_u)$, hvis og kun hvis $u \in span(S)$.
\label{stn:akvivalentespan}
\end{stn}
\begin{proof}
Antag først, at $\vec{u} \in span(S)$, da er $\vec{u}$ en linearkombination af $v_1,..., v_k$, hvorfor
\begin{align*}
span(S_u) &= \{ \vec{b} \in \mathds{R}^n\mid \exists \vec{x} \in \mathds{R}^n: \, \sum_{i=1}^k c_i \vec{v_i} + c_{k+1} \vec{u}  =\vec{b}\}
\\&= \{ \vec{b} \in \mathds{R}^n\mid \exists \vec{x} \in \mathds{R}^n: \, \sum_{i=1}^k c_i \vec{v_i} + c_{k+1} \sum_{i=1}^k C_i \vec{v_i} = \vec{b}\}
\\&= \{ \vec{b} \in \mathds{R}^n\mid \exists \vec{x} \in \mathds{R}^n: \, \sum_{i=1}^k K_j \vec{v_i} = \vec{b}\} = span(S)
\end{align*}
hvor $c_j, C_j, c_{k+1}, K_j$ er vilkårlige skalare.
\\ Antag $span(S) = span(S_u)$, dvs. at de to mængder udspænder den samme mængde af vektorer, hvorfor de samme vektorer som er en linearkombination af $\vec{v_1},...,\vec{v_k}, \vec{u}$ også er en linearkombination $\vec{v_1},..., \vec{v_k}$, dermed må $\vec{u}$ være en linearkombination af  $\vec{v_1},..., \vec{v_k}$, hvorefter det følger af Definition \ref{def:span}, at $\vec{u} \in span(S)$.
\end{proof}
Derfor må den mindste mængde vektorer, som udspænder et underrum, være lineært uafhængige.
Denne mængde af vektorer kaldes en basis.
\begin{defn}[Basis]
Lad $S =\{v_1,...,v_k\}$, hvor $\vec{v_1},...,\vec{v_k} \in \mathds{R}^n$ er lineært uafhængige, og lad $V$ være et underrum til $\mathds{R}^n$, da er $S$ en \textbf{basis} til $V$, hvis $V = span(S)$.
\label{def:basis}
\end{defn}
Antallet af vektorerne i basen vil være konstant.
\begin{stn}[Kardinalteten af en basis]
Lad $V$ være et ikke tomt underrum til $\mathds{R}^n$, og lad $B$ og $B'$ udgøre en basis for $V$, da vil $|B|=|B'|$.
\label{stn:basiskardinalitet}
\end{stn}
\begin{proof}
Lad $B$ bestå af $k$ vektorer og $B'$ af $p$ vektorer, antag for modstrid, at $k < p$, da vil der eksistere to matricer $A_{B}$ og $A_{B'}$, hvis søjler er vektorerne fra henholdsvis $B$ og $B'$.
Hvis $rank(A_{B}) = rank(A_{B'}) = k$ da vil vektorerne i $B'$ være lineært afhængig, hvorfor $B'$ ikke er en basis.
Hvis $rank(A_{B}) < rank(A_{B'} )$ da vil der eksistere en vektor $\vec{b} \in \mathds{R}^n$, hvor der eksistere en løsning til $A_{B'}\vec{x} = \vec{b}$, men ikke til ligningssystemet $A_B \vec{x}=\vec{b}$, hvorfor $span(B) \neq span(B')$, derfor kan både $B$ og $B'$ ikke være basis til $V$ af Definition \ref{def:basis}.
Derfor følger det at $p=k$.
\end{proof}
Hvilket fører til denne definition.
\begin{defn}[Dimension]
Lad $B$ være en basis for et ikke tomt underrum $V$ til $\mathds{R}^n$, da er \textbf{dimensionen} af $V$ givet ved $\dim{V} = |B|$
\label{def:dim}
\end{defn}
Selvom der ikke kan være forskellige antal vektorer i en basis, kan forskellige vektorer udgøre en basis for samme underrum.
Den mest kendte basis for $\mathds{R}^n$ er standardvektorene, og ved notationen $\rvect{a & b}^T$ forståes den lineære kombination
\begin{align*}
\begin{bmatrix} a \\ b \end{bmatrix} = a\begin{bmatrix} 1 \\0 \end{bmatrix} + b \begin{bmatrix} 0 \\ 1 \end{bmatrix}.
\end{align*}
Standard vektorerne udgør det, som kaldes en ortonormalbasis.




