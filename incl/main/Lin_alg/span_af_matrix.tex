\section{Span og basis for en matrix}
På samme måde som at en matrix kan skabes ud fra en delmængde af vektorer, så kan en delmængde af vektorer generes udfra søjlerne eller rækkerne i en matrix. 
Søjlerne i en matrix med rang $m$ vil udspænde $\mathds{R}^m$.
\begin{stn}[Span af $\mathds{R}^n$]
Lad $A$ være en $m\times n$ matrix og $S_A= \{A_j| A_j \in A\}$, da er følgende udsagn ækvivalente:
\begin{enumerate}[label=\alph*]
\item $span(S_A) = \mathds{R}^m$
\item $\exists \vec{x} \in \mathds{R}^n \forall \vec{b} \in \mathds{R}^m: \quad A\vec{x}=\vec{b}$
\item $rank(A) = m$
\item Den rækkerederede form af $A$ har ingen nul rækker
\item En hver række i $A$ har en pivot indgang
\end{enumerate}
\end{stn}
\begin{proof}
Først vises (a) <=> (b):
Lad $span(S_A) = \mathds{R}^m$ så følger det af Definition \ref{def:span} at $A\vec{x}= \vec{b}$ har en løsning for ethvert $\vec{b} \in \mathds{R}^m$.
På samme måde hvis $A\vec{x}= \vec{b}$ har en løsning for ethvert $\vec{b} \in \mathds{R}^m$, medfører det at $span(S_A) = \mathds{R}^m$. 
\\(c) <=> (d): 
Lad $rank(A) = m$, da følger det af definitionen for rang at antallet af nulrækker i den rækkereducerede form af matricen $A$ er $m-m = 0$.
På samme måde må rangen af $A$ være lig antallet af rækker, hvis der ikke er nogle nulrækker i den rækkereduceret form af $A$.
\\(d) <=> (e):
Lad den rækkereducerede form af $A$ have ingen nul rækker, da må der være en pivot indgang i hver række. 
Da der skal være en indgang pr. række med værdi forskellig $0$. 
På samme måde gælder at hvis der er en pivot indgang i en række kan det ikke være en nul række, når matricen bringes på rækkereduceret form, derfor må (e) => (d).
\\(b) => (d):
Antag for modstrid at $A$ indeholder en nulrække på rækkereduceret form, lad denne række være $\vec{a_i}^T = \vec{0}$, da vil $A\vec{x} = \vec{b}$ kun have en løsning hvis $b_i=0$, hvor $b_i$ er den $i$te indgang i $\vec{b}$, da $b_i = \vec{a_i}^T \vec{x} = \vec{0} \vec{x} = 0$. 
Derfor er der ikke en løsning til ethvert $\vec{b} \in \mathds{R}^m$, hvorfor der er opstået modstrid.
\\(e) => (b): 
Betragt total matricen $[A \vec{b}]$, da der er en pivot indgang i hver række, medføre det at totalmatricen ikke kan indeholde en række på formen $[\vec{0}^T, b_i]$, hvor $b_i \neq 0$, derfor må der være en løsning til ligningsstemmet $A\vec{x} = \vec{b}$.
Da enhver indgang i $\vec{b}$ kan udtrykkes $b_i = \sum_{j=1}^n a_{ij} x_i = k \cdot x_i$, hvor $a_{ij}$ er den $j$te indgang i den $i$te række, og $k$ er en skalar.
Da et hvert tal kan udtrykkes som produktet af to realle tal, må der altid være en løsning til $A\vec{x}=\vec{b}$.
\end{proof}
Underrummet udspændt af søjlerne i $A$ kaldes for søjlerummet af $A$, mens rækkerummet af $A$ er udspændt af $A$s rækker.
\begin{defn}[Søjlerum og Rækkerum]
Lad $A$ være en $n\times m$ matrix, da er \textbf{søjlerummet} af $A$ $Col A = span(\{A_j | j =1,...,n\})$, mens \textbf{rækkerummet} af $A$ er $Row A = span(\{\vec{a_i}|i=1,...,m\})$.
\label{def:sojlerum}
\end{defn} 
Det betyder at de lineært uafhængige søjler og rækker vil udgøre en basis for henholdsvis søjlerummet og rækkerummet, hvilket er de rækker og søjler med pivot indgang. 
De næste to resultater vil kun blive vist for søjlerummet, da de er ækvivalente til rækkerummet og det samme gælder for deres beviser.
\begin{kor}
Lad $A$ være en $n \times m$ matrix da vil søjlerne med en pivot indgang udgøre en basis for $Col (A)$
\label{kor:pivotsojle}
\end{kor}
\begin{proof}
Antag uden at miste generealitet at de først $k$ søjler i $A$ har en pivot indgang, da er $A_j$, for $j=1,...,k$ lineært uafhængige, mens $A_j$, for $j = k+1,...,n$ vil være en linear kombination af $\{A_j | j = 1,...,k\}$. 
Det følger derfor af Sætning \ref{stn:reduceringbasis} at $Col A = span(\{A_j| j=1,...,k\})$.
\end{proof}
Det betyder at dimensionen af søjlerummet og rækkerummet er lig antallet af pivotindgange i matricen, hvilket pr defintion er rangen af matricen.
\begin{kor}
Lad $A$ være en $n\times m $ matrix da er $\dim{col A} = rank (A)$
\end{kor}
\begin{proof}
Af Korolar \ref{kor:pivotsojle} følger at søjlerne i $A$ med pivot indgang udgør en base for $Col A$, da antallet af søjler med pivot indgang er lig rangen af $A$ må $\dim{Col A} = rank (A)$ i følge Definition \ref{def:dim}.
\end{proof}


%\begin{kor}
%Lad $A$ være en $n\times m $ matrix, da udgør rækkerne med pivot indgang en base for rækkerummet for $A$
%\end{kor}
%\begin{proof}
%Antag uden at miste generealitet at de først $k$ rækker i $A$ har en pivot indgang, da er $\vec{a_i}^T$, for $i=1,...,k$ lineært uafhængige, mens $\vec{a_i}^T$, for $i = k+1,...,n$ vil være en linear kombination af $\{\vec{a_i}^T | i = 1,...,k\}$. 
%Det følger derfor af Sætning \ref{stn:reduceringbasis} at $\{\vec{a_i}^T | i = 1,...,k\}$ vil være en basis for rækkerummet af $A$.
%\end{proof}