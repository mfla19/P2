\begin{defn}[Linje]
Lad $P\subseteq \mathds{R}^n $ være en polyide, da indeholder $P$ en \textbf{linje}, hvis $\vec{x}+\lambda\vec{d} \in P$ for alle $\lambda \in \mathds{R}^n$, hvor $\vec{x}\in P$ og $\vec{d} \in \mathds{R}^n$.
\end{defn}



%\begin{defn}
%Lad $P=\{\vec{x} \in \mathds{R}^n| A \vec{x} \geq b, \vec{x} \geq \vec{0}\} \neq \emptyset$ være en polyide, svarende til bibetingelserne til det lineære minimerings problem $\vec{c}^T\vec{x}$.
%Da siges en vektor $\vec{x^*}\in P$ at være \textbf{optimal}, hvis $\vec{c}^T\vec{x^*}\leq \vec{c}^T\vec{x}$ for alle $\vec{x}\in P$
%\end{defn}

\begin{stn}[Eksistens af optimal løsning]
Lad $P=\{\vec{x} \in \mathds{R}^n| A \vec{x} = b, \vec{x} \geq \vec{0}\} \neq \emptyset$ være en polyide, svarende til bibetingelserne til det lineære minimerings problem $\vec{c}^T\vec{x}$. Hvis $P$ ikke indeholder en linje, da vil der eksistere en optimal løsning $\vec{x^{**}}$ som er en basis mulig løsning.
\label{stn:eksistens}
\end{stn}

\begin{proof}
Da $P$ ikke er tom må der eksistere en vektor $\vec{x} \in P$.
Antag at $\vec{x}$ ikke er en basis mulig løsning, da vil $I = \{i | \vec{a_i}\vec{x} = b_i\}$, hvor $\vec{a_i}\vec{x}=b_i$ er lineære uafhængige bibetingelser, indeholde færre end $n$ elementer. 
Nu konstrueres en basis mulig løsning $\vec{x^*}$, med udgangspunkt i $\vec{x}$ så $\vec{c}^T\vec{x^*}\leq \vec{c}^T\vec{x}$.
Da $|I|<n $ må $span(\{a_i | i \in I\})$ være et ægte underrum til $\mathds{R}^n$, hvorfor der eksistere en vektor $\vec{d} \in \mathds{R}^n$ så $\vec{a_I}\vec{d}=0$ for alle $i \in I$. 
Vælg nu fortegn på $\vec{d}$ så $\vec{c}\vec{d} \leq 0$, da vil $\vec{c}(\vec{x}+\lambda\vec{d}) \leq \vec{c}\vec{x}$ hvor $\lambda$ er en positiv skalar.
Da $P$ er begrænset, og derfor ikke indeholder en linje, må der være et $\lambda '$ som gør at $\vec{x}+\lambda '\vec{d} \not P$, hvorfor at der for et $\lambda^*$, må gælde at $\vec{a_j}(\vec{x}+\lambda^* \vec{d}) = b_j$, for $j \notin I$.
Bemærk at $\vec{x}+\lambda^*\vec{d} \in P$, da $\vec{a_i}(\vec{x}+\lambda\vec{d})= \vec{a_i}\vec{v} = b_i$, for $i \in I$, hvorfor alle aktive bibetingelser stadig er overholdt.
Antag nu at $\vec{d}$ er lineært afhængig af andre bibetingelser end $\vec{a_j}$ da vil $\vec{a_j}$ også være lineært afhæng af dem, hvorfor det følger af Sætning $P=Q$ at man kan se bort fra dem, og $\vec{x}+\lambda \vec{d} \in P$. 
Da $\vec{d}$ er lineært afhængig af $\vec{a_j}$, må det medfører at da $\vec{d}$ er lineært uafhængig af $\vec{a_i}$ så må $\vec{a_j}$ også være det. 
Derfor kan $j$ tilføjes til $I$. 
Gentag til at $I$ indeholder $n$ lineært uafhængige vektorer, hvorefter at en basis mulig løsning er konstrueret.
Da $\vec{x^*}$ er konstrueret ud fra en vilkårlig vektor $\vec{x}\in P$ medføre det at en mulig basis løsning altid vil opfylde $\vec{c}^T\vec{x^*} \leq \vec{c}\vec{x}$ for en hver ikke basis mulig løsning. 
Da der kun er en endelig mængde mulig basis løsninger må der være en vektor $\vec{x^{**}}$ som opfylder at $\vec{c}^T\vec{x^{**}}\leq \vec{c}^T\vec{x^*}$ for alle andre basis mulige løsninger, hvorfor at $\vec{x^{**}}$ er en optimal løsning.
\end{proof}

\begin{kor}
Lad $f(\vec{x}) = \vec{c}^T\vec{x}$ betegne et lineært minimerings programmerings problem på augumenteret form, med mulige løsnings mængde $\mathcal{F} \neq \emptyset$. 
Da vil der eksistere en optimal løsning, som er en mulig basis løsning, hvis der eksistere $K \in \mathds{R}$ så $f(\mathcal{F}) \leq K$ 
\end{kor}

\begin{proof}
Lad  $K \in \mathds{R}$ så $f(\mathcal{F}) \leq K$, da må der for et hvert $\vec{x} \in P$ og $\vec{d} \in \mathds{R}^n$ eksistere et $\lambda$ så $f(\vec{x}+\lambda \vec{d}) = K$, og da $f$ er lineær følger det at $\mathcal{F}$ ikke kan indholde en linje. 
Da det lineære programerings problem er på augumenteret form betyder det at $\mathcal{F}=P=\{\vec{x} \in \mathds{R}^n| A \vec{x} = b, \vec{x} \geq \vec{0}\} \neq \emptyset$ følger det af Sætning \ref{stn:eksistens}, at er eksistere en optimal løsning, som er en mulig basis løsning til det lineære minimerings programmerings problem.
\end{proof}