\section{Simplex ud spændt af Basisløsninger}
\begin{defn}[Konveks form]
Et lineært programmeringsproblem på formen:
\begin{center}
\begin{tabular}{l	>{$}l<{$}}
Minimer			& \vec{c}^T\vec{x} \\
med hensyn til 	& A\vec{x} = \vec{b}\\
og				& \vec{e}^T\vec{x} = 1\\
og 				& \vec{x} \geq \vec{0}, 
\end{tabular}
\end{center}
for $\vec{e} =\rvect{1 & \cdots & 1 }^T$,  siges at være på \textbf{konveks form}, og bibetingelsen $\vec{e}^T\vec{x} = 1$ kaldes konveksbibetingelsen.
\end{defn}.

\begin{stn}[Konveks form]
Et hvert lineært programmeringsproblem med $\vec{b}\neq \vec{0}$ kan omskrives til konveks form.
\end{stn}
\begin{proof}
I Kapitel 
er det gennemgået hvordan alle lineært programmerings problemer kan skrives på standard form med ligheder, derfor er det kun nødvendigt at vise at et hvert lineært programmerings problem på standard form med ligheder kan omskrives til konveks form.
\\ Lad derfor $\vec{x} \neq \vec{0}$ være en løsning til et lineært programmerings problem på standard form med ligheder, da vil 
\begin{align*}
\vec{e}^T \vec{x} = \lambda,
\end{align*}
hvor lambda er en positiv skalar.
Da vil 
\begin{align*}
\vec{e}^T\vec{x}' = \vec{e}^T\frac{1}{\lambda}\vec{x} = 1.
\end{align*}
For at sørge for at den konvekse form har samme løsningsmængde som det lineære programmerings problem på standard form med ligheder, multipliceres $A$ med $\lambda$, hvorefter at
\begin{align*}
A' \vec{x}' = \lambda A \frac{1}{\lambda} \vec{x} = A \vec{x} = \vec{b}.
\end{align*}
Lad nu $\vec{x} = \vec{0}$ være en løsning, da vil
\begin{align*}
A \vec{x} = \vec{0} = \vec{b}.
\end{align*}
Da det er antaget at $\vec{b} \neq \vec{0}$, må et hvert lineært programmeringsproblem med $\vec{b}\neq \vec{0}$ kan omskrives til konveks form.
\end{proof}

\begin{defn}[Værdi af $\vec{x}$]
Lad $f$ være en objektfunktion da er $f(\vec{x}) = z_x$  \textbf{værdien af $\vec{x}$}
\end{defn}

\begin{stn}
Lad $\vec{x}$ være en basisløsning til et lineært problem på konveks form, så $x_i = 0$ for $i \notin I_B = \{B(1),..., B(m)\}$. Så  $\vec{B}_i  = \rvect{\vec{A}_i & c_i}^T$ for $i \in I_B$ en simplex $S_x$, så $\vec{b}_x = \rvect{\vec{b}& z_x}^T \in S_x$
\end{stn}
Bemærk at da problemet er på konveks form vil $|I_B| = m+1$ hvis $A$ er en $n\times m$ matrix, da det i følge Sætning 
kan antages at alle rækker er lineært uafhængige med $\vec{e}$.
\begin{proof}
For at $\rvect{\vec{A}_i & c_i}^T$ for $i \in I_B$ udspænder en simplex, skal vektorene være affint lineært uafhængige.
Derfor vises først at $\rvect{\vec{A}_i & c_i}^T$ for $i \in I_B$ er affint lineært uafhængige.
Antag for modstrid at de ikke er, da vil der eksistere skalare forskelligt fra $0$ så
\begin{align*}
\sum_{i = 1}^{m} \lambda_i (\vec{A}_{B(i)} - \vec{A}_{B(m+1)} =  \vec{0} \qquad \wedge \qquad \sum_{i=1}^{m} \lambda_i (c_i - c_{m+1})= 0.
\end{align*}
Betragt nu kun $\sum_{i = 1}^{m} \lambda_i (\vec{A}_{B(i)} - \vec{A}_{B(m+1)} =  \vec{0}$, det medføre at
\begin{align*}
\sum_{i = 1}^{m} \lambda'_i \vec{A}_{B(i)} = \vec{A}_{B(m+1)},
\end{align*}
hvor $\lambda'_i = \lambda_i/(\sum_{i=1}^m \lambda_i$.
Derfor følger derfor at hvis de ikke er affint lineært uafhængige, så er $\vec{A}_{B(m+1)}$ en linear kombination af $\vec{A}_{B(i)}$ for $i  \in i,..., m$.
Det strider mod at $\vec{x}$ er en basisløsning, og søjlerne $\vec{A}_i$ svare til basis variablene for $i \in I_B$, derfor må vektorerne være affint lineært uafhængige. 
Dermed udgør alle konveksekombinationer af $B_i$ for $i \in I_B$ en simplex, $S_x$.
\\ Så vises det at $\vec{b} \in S_x$. 
Det følger af Definition
at $\vec{b}_x\in S_x$ hvis der eksistere skalare der opfylder $\sum_{i=1}^{m+1} \lambda_i = 1$, så $\sum_{i=1}^{m+1}\lambda_i B_{B(i)}  = \vec{b}_x$.
Da $\vec{x}$ er en basisløsning følger det at $B \vec{x}_B = \vec{b}_x$ og da $\vec{x}$ er betinget af konveksbetingelsen og $x_i = 0 $ for $i \notin I_B$ må $\sum_{i=1}^{m+1} x_{B(i)} = 1$, hvorfor det følger at $\vec{b}_x \in S_x$.
\end{proof}

\begin{defn}[Simplex forbundet med basisløsning]
Lad $\vec{x}$ være en basisløsning så $x_i = 0$ hvis $i \notin I_B$, da er $S_x$ \textbf{Simplexen forbundet med basisløsning $\vec{x}$} hvis simplexen er lig konvekshuldet udspændt af søjlerne af $\rvect{\vec{A}_i & c_i}^T$ for $i \in I_B$.
\end{defn}

\begin{prop}
Afstanden mellem to simplex $S_x$ og $S_y$ forbundet med basisløsningerne $\vec{x}$ og $\vec{y}$ er $z_x - z_y$.
\end{prop}

\begin{proof}
For at vise proportionen findes længden af $\rvect{B_x & \vec{c}_B}^T \vec{x} - \rvect{B_y & \vec{c}_y}^T \vec{y}$
\begin{align*}
 \Vert \rvect{B_x & \vec{c}_B}^T \vec{x} - \rvect{B_y & \vec{c}_y}^T \vec{y} \Vert & =  \Vert \rvect{\vec{b} & z_x}^T  - \rvect{\vec{b} & z_y}^T  \Vert
 \\ & = \Vert \rvect{0 & z_x - z_y}^T \Vert = z_x - z_y.
\end{align*}
Dermed kan det konkluderes at afstanden mellem to simplex $S_x$ og $S_y$ forbundet med basisløsningerne $\vec{x}$ og $\vec{y}$ er $z_x - z_y$.
\end{proof}

