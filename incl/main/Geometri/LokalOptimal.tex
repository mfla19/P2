\section{Konveks}
I lineær programmering ønskes en funktion optimeret, der ledes derfor efter optimale løsninger.
Betragt derfor beviset for Sætning \ref{stn:eksistens}, her konstrueres en bedre løsning ved at lægge en retningsvektor til en vilkårlig løsning, så summen af retningsvektoren og løsningen har en mindre funktionsværdi end løsningen. 
Retningen af retningsvektoren kaldes en mulig retning.
\begin{defn}[Mulig retning]
Lad $\vec{x} \in P$. En vektor $\vec{d}$ er en \textbf{mulig retningen} fra $\vec{x}$, hvis der eksistere en positiv skalar $\theta > 0$ således at $\vec{x}+\theta\vec{d} \in P$ 
\end{defn}
Men hvordan kan det vides, at en optimal løsning er opnået, hvis der ikke er en mulig retning som reducere objektfunktionen og ikke blot en lokal optimal løsning.
\begin{defn}[Lokalt optimal løsning]
Lad $\vec{x} \in P$ være en løsnings til et lineært programerigns problem med objektfunktion $f$ da er $\vec{x}$ en lokal optimal løsning, hvis $f(\vec{x}) \leq f(\vec{y})$ for $|\vec{y}-\vec{x}|< \epsilon$ for et $\epsilon > 0$ og $\vec{y} \in P$.
\end{defn}
For at undersøge om en lokal optimal løsning er optimal, er det nødvendigt at undersøge om løsningsmængden er konveks.
\begin{defn} [Konveks mængde]
Lad $S \subset \mathds{R}^n$  da er $S$ konveks, hvis der $\forall \vec{x}, \vec{y} \in S$ og et vilkårligt $\lambda \in [0,1]$ gælder at $\lambda \vec{x} + (1-\lambda) \vec{y} \in S$.
\label{def:Konveks}
\end{defn}
Med andre ord så vil ligningen $\lambda \vec{x} + (1-\lambda) \vec{y}$, hvor $\lambda$ ligger i intervallet fra $0$ til $1$, danne en linje af elementer mellem $\vec{x}$ og $\vec{y}$, hvis alle disse elementer er inden som området $S$, så er mængden konveks.
Det viser sig at løsningsmængden for et lineært programmerings problem besider denne egenskab.
\begin{stn}
Lad $P =\{ \vec{x} \in \mathds{R}^n | A \vec{x} \geq \vec{b}\} $ være en polyede, da er $P$ konveks.
\label{stn:polykon}
\end{stn}
\begin{proof}
Lad $\vec{x}, \vec{y} \in P=\{ \vec{x} \in \mathds{R}^n | A \vec{x} \geq \vec{b}\}$ være to vilkårlige vektorer, da gælder at $A\vec{x} \geq \vec{b}$ hvilket medfører at $\lambda A \vec{x} \geq \lambda\vec{b}$, hvor $\lambda \in [0,1]$ er en skalar. 
På ligefod må der derfor gælde at $(1-\lambda)A\vec{y} \geq (1-\lambda)\vec{b}$.
De to uligheder adderes nu
\begin{align*}
\lambda A \vec{x} + (1-\lambda) A \vec{y} \geq \lambda \vec{b} + (1 - \lambda) \vec{b}
\\  A (\lambda\vec{x} + (1-\lambda)\vec{y}) \geq \vec{b}.
\end{align*}
Derfor må $\lambda\vec{x} + (1-\lambda)\vec{y} \in P$.
\end{proof}
LIgesom at en mængde kan være konveks, så kan en funktion også være det.
\begin{defn}[Konveks funktion]
Lad $f:A\to B$ være en funktion, da er $f$ en \textbf{konveks funktion} hvis
\begin{align*}
	f(\lambda x + (1-\lambda) y) \leq \lambda f(x) + (1-\lambda)f(y), \qquad \forall x,y \in A, \forall \lambda \in [0,1]
\end{align*}
\end{defn}
En funktion er konveks, hvis linjen mellem to punkter enten følger grafen for funktionen eller ligger over grafen. 
En lineær funktion, og dermed også objektfunktionen til et lineært programmerings problem er konveks.
\begin{stn}
En lineær funktion er en konveksfunktion.
\label{stn:funkon}
\end{stn}
\begin{proof}
Lad $f:A\to B$ være en lineær funktion, da følger det af Definition
at 
\begin{align*}
f(\lambda x + (1-\lambda) y)  = f(\lambda x) + f((1-\lambda)y)  = \lambda f(x) + (1-\lambda)f(y)
\end{align*}
for alle $x,y \in A$ og $\lambda \in \mathds{R}$. 
Derfor må det også gælde for $\lambda \in [0,1] \subset \mathds{R}$, hvorfor det kan konkluderes at $f$ er konveks.
\end{proof}
Da løsningsmængden og objektfunktionen begge er konvekse, så vil en lokal optimal løsning altid være en optimal løsning.
\begin{stn}
Lad $\vec{x} \in P$ være en lokal optimal løsnings til et lineært programerignsproblem  med objektfunktion $f$.
Da er  $f(\vec{x}) \leq f(\vec{y})$ for alle $\vec{y} \in P$.
\end{stn}
\begin{proof}
Antag at $\vec{x}^* \in P$ er en optimal løsning.
Da $P$ er en polyeder, følger det af Sætning \ref{stn:polykon}, at $P$ er konveks, hvormed $\lambda \vec{x}^* + (1-\lambda)\vec{x} \in P$. 
Da $\lambda$ kan vælges til at være vilkårligt tæt på nul, må $|\lambda \vec{x}^* + (1-\lambda)\vec{x} - \vec{x}| < \epsilon$ da
\begin{align*}
 |\lambda \vec{x}^* + (1-\lambda)\vec{x} - \vec{x}| = | \lambda \vec{x}^* - \lambda\vec{x}| = \lambda|\vec{x}^* - \vec{x}| < \epsilon.
\end{align*}
Da $\vec{x}$ er en lokal optimal løsning, følger det, at
\begin{align*}
f(\vec{x}) \leq f(\lambda \vec{x}^* + (1-\lambda)\vec{x}).
\end{align*}
Da $f$ er lineær følger det af Sætning \ref{stn:funkon}, at $f$ er konveks, hvorfor
\begin{align*}
f(\vec{x}) \leq f(\lambda \vec{x}^* + (1-\lambda)\vec{x}) &\leq \lambda f(\vec{x}^*) + (1-\lambda)f(\vec{x}) \qquad \Rightarrow
\\ f(\vec{x}) - f(\vec{x}) &\leq \lambda f(\vec{x}^*) + (1-\lambda)f(\vec{x}) \qquad \Rightarrow
\\ 0 & \leq \lambda( f(\vec{x}^*) - f(\vec{x})),
\end{align*}
hvis $\lambda$ er lille nok.
Da $\vec{x}^*$ er en optimal løsning, betyder det at $f(\vec{x}^*) \leq f(\vec{x})$, dermed vil højre siden af uligheden $0 \leq \lambda( f(\vec{x}^*) - f(\vec{x}))$ være negativ, med mindre at $f(\vec{x}^*) =f(\vec{x})$. 
Det kan derfor konkluderes, hvis $\vec{x}$ er en lokal optimal løsning, så er $\vec{x}$ en optimal løsning.
\end{proof}

\section{Konsvekshuld}
En egneskab ved konvekse mængder er at enhver konveks kombination,
\begin{defn}[Konveks kombination]
Lad $\vec{x}^1, ...,\vec{x}^k \in \mathds{R}^n$, og $\lambda_1,..., \lambda_k \geq 0 $ være skalare, som opfylder $\sum_{i=1}^k \lambda_i =1$ da er $\sum_{i=1}^k \lambda_i \vec{x}^1$ en \textbf{konveks kombination}.
\label{def:KonveksKombination}
\end{defn}
af elementer fra mængden er et nyt element i den konvekse mængde.
\begin{stn}[Konveks kombination]
Lad $S\subset \mathds{R}^n$ være en konveks mængde, da
\begin{align*}
	\sum_{i=1}^k \lambda_i \vec{x}^i \in S, \qquad \vec{x}^1, ...,\vec{x}^k \in S.
\end{align*}
\label{stn:KonveksKombination}
\end{stn}
\begin{proof}
For at vise Sætning \ref{stn:KonveksKombination} gøres brug af et induktionsbevis.
Lav derfor induktionsstarten ved at betragte den konvekse kombination $\sum_{i=1}^2 \lambda_i \vec{x}^1$.
Da $\sum_{i=1}^2 \lambda_i = \lambda_1 + \lambda_2 = 1$ ifølge Definition \ref{def:KonveksKombination} (a), må $\lambda_2 = (1 - \lambda_1)$.
Det indsættes nu i den konvekse kombination af $\vec{x}^1$ og $\vec{x}^2$, hvorfor at $\lambda_1 \vec{x}^1+ (1-\lambda_1) \vec{x}^2$.
Da $S$ er konveks følger det af Definition \ref{def:Konveks} at $\sum_{i=1}^2 \lambda_i \vec{x}^1 \in S$.
\\Antag derefter at $\sum_{i=1}^k \lambda_i \vec{x}^i \in S$ som induktionshypotesen.
\\ Det vises nu at induktionshypotesen medfører at Sætning \ref{stn:KonveksKombination} også gælder for $k+1$.
Lav derfor induktionstrinnet ved at betragte 
\begin{align*}
	\sum_{i=1}^{k+1} \lambda_i \vec{x}^i &= \lambda_{k+1}\vec{x}^{k+1} + \sum_{i=1}^k \lambda_i \vec{x}^i
	\\ &= \lambda_{k+1}\vec{x}^{k+1} + \frac{1-\lambda_{k+1}}{1-\lambda_{k+1}} \sum_{i=1}^k \lambda_i \vec{x}^i
	\\ &= \lambda_{k+1}\vec{x}^{k+1} + (1-\lambda_{k+1}) \sum_{i=1}^k \frac{1}{1-\lambda_{k+1}} \lambda_i \vec{x}^i
\end{align*}
Observer da at da $\sum_{i=1}^{k+1} \lambda_i \vec{x}^i $ gælder
\begin{align*}
	\sum_{i=1}^{k+1} \lambda_i  & = 1
	\\ \sum_{i=1}^{k} \lambda_i &= 1 - \lambda_{k+1}
	\\ \frac{1}{1-\lambda_{k+1}} \sum_{i=1}^{k} \lambda_i &= \frac{1-\lambda_{k+1}}{1-\lambda_{k+1}} = 1.
\end{align*}
Derfor er $\sum_{i=1}^k \frac{1}{1-\lambda_{k+1}} \lambda_i \vec{x}^i$ en konveks kombination af $k$ elementer, hvorfor det følger af induktionshypotesen at $\sum_{i=1}^k \frac{1}{1-\lambda_{k+1}} \lambda_i \vec{x}^i \in S$, hvorfor at $\sum_{i=1}^{k+1} \lambda_i \vec{x}^i \in S$, og sætningen er bevist.
\end{proof}
En samling af konvekse kombinationer, kaldes et konvekshuld.
\begin{defn}[Konveks huld]
Lad $\vec{x}^1, ...,\vec{x}^k \in \mathds{R}^n$, da er $C_{x} = \{\sum_{i=1}^k \lambda_i \vec{x}^1| \vec{x}^1, ...,\vec{x}^k \in \mathds{R}^n, \sum_{i=1}^k \lambda_i =1\}$ et \textbf{konveks huld} for vektorene $\vec{x}^1, ...,\vec{x}^k$. 
\label{def:Konvekshuld}
\end{defn}
Efter som at enhver konveks kombination af elementer fra en konveks mængde er et element i mængden, vil det give mening at konvekshuldet også var en konveks mængde.
\begin{stn}[Konvekse mængder]
Konveks huldet $C_x = \{\sum_{i=1}^k \lambda_i \vec{x}^1| \vec{x}^1, ...,\vec{x}^k \in \mathds{R}^n, \sum_{i=1}^k \lambda_i =1\}$ over en endelig mængde vektorer, er en konveks mængde
\end{stn}
\begin{proof}
Lad $\vec{z}, \vec{y}\in C_x = \{\sum_{i=1}^k \lambda_i \vec{x}^i| \vec{x}^1, ...,\vec{x}^k \in \mathds{R}^n, \sum_{i=1}^k \lambda_i =1\}$ være vilkårlige vektorer da må $\vec{z}= \sum_{i=1}^k \gamma_i \vec{x}^i, \vec{y}= \sum_{i=1}^k \eta_i \vec{x}^i$ for $\sum_{i=1}^k \gamma_i = 1$ og  $\sum_{i=1}^k \eta_i = 1$. 
Derfor må
\begin{align*}
	\lambda \vec{z} + (1- \lambda) \vec{y} &= \lambda\sum_{i=1}^k \gamma_i \vec{x}^i + (1-\lambda)\sum_{i=1}^k \eta_i \vec{x}^i
	\\ &=\sum_{i=1}^k (\lambda \gamma_i+(1-\lambda)\eta_i )\vec{x}^i,
\end{align*}
For $\lambda \in [0,1]$.
Betragt nu konstanterne 
\begin{align*}
	\sum_{i=1}^k (\lambda \gamma_i+(1-\lambda)\eta_i ) &= \lambda \sum_{i=1}^k \gamma_i + (1 - \lambda) \sum_{i=1}^k \eta_i 
	\\ &= \lambda \cdot 1 + (1 - \lambda) \cdot 1 = 1
\end{align*}
Hvorfor at $\lambda \vec{z} + (1- \lambda) \vec{y} $ er en konveks kombination af vektorene $\vec{x}^1, ...,\vec{x}^k $, ifølge Definition \ref{def:KonveksKombination}. 
Derfor må $ \lambda \vec{z} + (1- \lambda) \vec{y} \in C_x$, hvorfor at $C_x$ er konveks ifølge Definition \ref{def:Konveks}.
Og sætningen er bevist.
\end{proof}
Et særtilfælde af konvekshuld er kaldet en simplex.
\begin{defn}[Simplex]
Lad $C_x$ være et konveks huld, af $k+1$ affint lineært uafhængige vektorer, da er $C_x$ en $k$-dimentionel \textbf{Simplex}.
\end{defn}
Simplex metoden bygger på at betragte forskellige simplexer, og så finde den simplex, hvis skæring med $\vec{b}$ er lig den optimale løsning, det kan lade sig gøre da søjlerne i en basismatrix til en basisløsning udspænder en simplex.
\section{Simplex ud spændt af Basisløsninger}
\begin{defn}[Konveks form]
Et lineært programmeringsproblem på formen:
\begin{center}
\begin{tabular}{l	>{$}l<{$}}
Minimer			& \vec{c}^T\vec{x} \\
med hensyn til 	& A\vec{x} = \vec{b}\\
og				& \vec{e}^T\vec{x} = 1\\
og 				& \vec{x} \geq \vec{0}, 
\end{tabular}
\end{center}
for $\vec{e} =\rvect{1 & \cdots & 1 }^T$,  siges at være på \textbf{konveks form}, og bibetingelsen $\vec{e}^T\vec{x} = 1$ kaldes konveksbibetingelsen.
\end{defn}.

\begin{stn}[Konveks form]
Et hvert lineært programmeringsproblem med $\vec{b}\neq \vec{0}$ kan omskrives til konveks form.
\end{stn}
\begin{proof}
I Kapitel 
er det gennemgået hvordan alle lineært programmerings problemer kan skrives på standard form med ligheder, derfor er det kun nødvendigt at vise at et hvert lineært programmerings problem på standard form med ligheder kan omskrives til konveks form.
\\ Lad derfor $\vec{x} \neq \vec{0}$ være en løsning til et lineært programmerings problem på standard form med ligheder, da vil 
\begin{align*}
\vec{e}^T \vec{x} = \lambda,
\end{align*}
hvor lambda er en positiv skalar.
Da vil 
\begin{align*}
\vec{e}^T\vec{x}' = \vec{e}^T\frac{1}{\lambda}\vec{x} = 1.
\end{align*}
For at sørge for at den konvekse form har samme løsningsmængde som det lineære programmerings problem på standard form med ligheder, multipliceres $A$ med $\lambda$, hvorefter at
\begin{align*}
A' \vec{x}' = \lambda A \frac{1}{\lambda} \vec{x} = A \vec{x} = \vec{b}.
\end{align*}
Lad nu $\vec{x} = \vec{0}$ være en løsning, da vil
\begin{align*}
A \vec{x} = \vec{0} = \vec{b}.
\end{align*}
Da det er antaget at $\vec{b} \neq \vec{0}$, må et hvert lineært programmeringsproblem med $\vec{b}\neq \vec{0}$ kan omskrives til konveks form.
\end{proof}

\begin{defn}[Værdi af $\vec{x}$]
Lad $f$ være en objektfunktion da er $f(\vec{x}) = z_x$  \textbf{værdien af $\vec{x}$}
\end{defn}

\begin{stn}
Lad $\vec{x}$ være en basisløsning til et lineært problem på konveks form, så $x_i = 0$ for $i \notin I_B = \{B(1),..., B(m)\}$. Så  $\vec{B}_i  = \rvect{\vec{A}_i & c_i}^T$ for $i \in I_B$ en simplex $S_x$, så $\vec{b}_x = \rvect{\vec{b}& z_x}^T \in S_x$
\end{stn}
Bemærk at da problemet er på konveks form vil $|I_B| = m+1$ hvis $A$ er en $n\times m$ matrix, da det i følge Sætning 
kan antages at alle rækker er lineært uafhængige med $\vec{e}$.
\begin{proof}
For at $\rvect{\vec{A}_i & c_i}^T$ for $i \in I_B$ udspænder en simplex, skal vektorene være affint lineært uafhængige.
Derfor vises først at $\rvect{\vec{A}_i & c_i}^T$ for $i \in I_B$ er affint lineært uafhængige.
Antag for modstrid at de ikke er, da vil der eksistere skalare forskelligt fra $0$ så
\begin{align*}
\sum_{i = 1}^{m} \lambda_i (\vec{A}_{B(i)} - \vec{A}_{B(m+1)} =  \vec{0} \qquad \wedge \qquad \sum_{i=1}^{m} \lambda_i (c_i - c_{m+1})= 0.
\end{align*}
Betragt nu kun $\sum_{i = 1}^{m} \lambda_i (\vec{A}_{B(i)} - \vec{A}_{B(m+1)} =  \vec{0}$, det medføre at
\begin{align*}
\sum_{i = 1}^{m} \lambda'_i \vec{A}_{B(i)} = \vec{A}_{B(m+1)},
\end{align*}
hvor $\lambda'_i = \lambda_i/(\sum_{i=1}^m \lambda_i$.
Derfor følger derfor at hvis de ikke er affint lineært uafhængige, så er $\vec{A}_{B(m+1)}$ en linear kombination af $\vec{A}_{B(i)}$ for $i  \in i,..., m$.
Det strider mod at $\vec{x}$ er en basisløsning, og søjlerne $\vec{A}_i$ svare til basis variablene for $i \in I_B$, derfor må vektorerne være affint lineært uafhængige. 
Dermed udgør alle konveksekombinationer af $B_i$ for $i \in I_B$ en simplex, $S_x$.
\\ Så vises det at $\vec{b} \in S_x$. 
Det følger af Definition
at $\vec{b}_x\in S_x$ hvis der eksistere skalare der opfylder $\sum_{i=1}^{m+1} \lambda_i = 1$, så $\sum_{i=1}^{m+1}\lambda_i B_{B(i)}  = \vec{b}_x$.
Da $\vec{x}$ er en basisløsning følger det at $B \vec{x}_B = \vec{b}_x$ og da $\vec{x}$ er betinget af konveksbetingelsen og $x_i = 0 $ for $i \notin I_B$ må $\sum_{i=1}^{m+1} x_{B(i)} = 1$, hvorfor det følger at $\vec{b}_x \in S_x$.
\end{proof}

\begin{defn}[Simplex forbundet med basisløsning]
Lad $\vec{x}$ være en basisløsning så $x_i = 0$ hvis $i \notin I_B$, da er $S_x$ \textbf{Simplexen forbundet med basisløsning $\vec{x}$} hvis simplexen er lig konvekshuldet udspændt af søjlerne af $\rvect{\vec{A}_i & c_i}^T$ for $i \in I_B$.
\end{defn}

\begin{prop}
Afstanden mellem to simplex $S_x$ og $S_y$ forbundet med basisløsningerne $\vec{x}$ og $\vec{y}$ er $z_x - z_y$.
\end{prop}

\begin{proof}
For at vise proportionen findes længden af $\rvect{B_x & \vec{c}_B}^T \vec{x} - \rvect{B_y & \vec{c}_y}^T \vec{y}$
\begin{align*}
 \Vert \rvect{B_x & \vec{c}_B}^T \vec{x} - \rvect{B_y & \vec{c}_y}^T \vec{y} \Vert & =  \Vert \rvect{\vec{b} & z_x}^T  - \rvect{\vec{b} & z_y}^T  \Vert
 \\ & = \Vert \rvect{0 & z_x - z_y}^T \Vert = z_x - z_y.
\end{align*}
Dermed kan det konkluderes at afstanden mellem to simplex $S_x$ og $S_y$ forbundet med basisløsningerne $\vec{x}$ og $\vec{y}$ er $z_x - z_y$.
\end{proof}

