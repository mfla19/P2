\section{Hvad er en løsning?}


\begin{defn}[Ekstremuspunkt]
Lad $\vec{x} \in P$, hvor $P$ er en polyeder og $\lambda \in [0,1]$ være en skalar.
Da er $\vec{x}$ et ekstremuspunkt hvis $\nexists \vec{y}, \vec{z} \in P$ så $\vec{x} = \lambda \vec{y}+ (1- \lambda)\vec{z}$ for $\vec{y}\neq \vec{x}$ og $\vec{z}\neq \vec{x}$.
\end{defn}

\begin{defn}[Knude]
Lad $\vec{x} \in P$, hvor $P$ er en polyeder, da er $\vec{x}$ en knude, hvis $\exists \vec{c} $ så $\vec{c}^T \vec{x} < \vec{c}^T\vec{y}$ $\forall \vec{y} \in P$ for $\vec{y} \neq \vec{x}$.
\end{defn}


\begin{defn}[Aktive krav]
Lad $\vec{x}^* \in \mathds{R}^n$ opfylde at $\vec{a}_i^T\vec{x}^* = b_i$, hvor $\vec{a}_i$ svare til en bibetingelse, da siges $\vec{a_i}^T$ at være \textbf{aktiv}.
\end{defn}

\begin{stn}[Lineær uafhængige rækker]
Lad $\vec{x}^* \in \mathds{R}^n$ og $I = \{i | \vec{a_i}^T \vec{x}^*\geq b_i\}$ , da er følgende ækvivalent
\begin{enumerate}[label=(\alph*)]
\item Mængden $R_a =\{\vec{a_i}| i\in I\}$ består af $n$ elementer, som er lineært uafhængige.
\item Spannet $span(\vec{a_i}) = \mathds{R}^n$, for $a_i \in R_a$.
\item Ligningssystemet $\vec{a_i}^T \vec{x}^* = b_i$, for $a_i \in R_a$ har en unik løsning.
\end{enumerate}
\end{stn}

\begin{proof}
(a) => (c): Lad $A$ være en $n \times n$ matrix bestående af rækkerne $\vec{a_i}^T \in R_a$, og antag at alle $\vec{a_i}$ er lineært uafhængige. 
Antag nu for modstrid at ligningssystemet $A\vec{x} = \vec{b}$, hvor $\vec{b}$ har indgangende $b_i$ for $i \in I$, har to løsninger. 
Da vil $A \vec{x} = \vec{b}$ og $A \vec{y} = \vec{b}$ hvorfor $A\vec{x-y} = \vec{0}$.
Derfor må søjlerne enten være lineært afhængige, eller $\vec{x-y} = \vec{0}$.
Da der er $n$ lineært uafhængige rækker i $A$ må $rank(A) = n$, hvorfor alle søjler er lineært uafhængige, da $A$ er en $n \times n$ matrice.
Det medfører at $\vec{x-y} = \vec{0}$, hvorfor det kan konkluderes at løsningen til ligningssystemet er unikt da $ \vec{x}=\vec{y}$.
\\(c) => (a):Antag at der er en unik løsning til $A$ det medfører at $rank(A) = n$, hvorfor alle rækker må være lineært uafhængige, , da $A$ er en $n \times n$ matrice. % s. 78-79 LinAlg
\\ (a) => (b): Da $R_a$ består af $n$ linært uafhængige $n$-dimentionelle vektorer, må $R_a$ udgøre en base for $\mathds{R}^n$, hvorfor $span(R_a) = \mathds{R}^n$.
\\ (b) => (a): Antag for modstrid, at vektorene i $R_a$ ikke er lineært uafhængige, da vil der eksistere en vektor $\vec{a_i'} \in R_a$ som er en linear kombination af de andre vektorer, der må derfor eksistere en vektor $A\vec{x} = \vec{0}$, hvor $\vec{x} \neq \vec{0}$, da det kun kan lade sig gøre hvis $\vec{x}$ er ortogonal med alle rækker i $A$, kan $\vec{x}$ ikke være en linear kombination af vektorene i $R_a$.
Hvorfor at $\vec{x} \notin span(R_a)$.
Derfor kan $span(R_a) \neq \mathds{R}^n$, hvorfor der er opstået modstrid.
\end{proof}





\begin{defn}[Basis løsning]
$P$, $\vec{x}^*\in \mathds{R}^n$, da er $\vec{x}^*$ en \textbf{Basis løsning} hvis:
\begin{enumerate}[label=(\alph*)]
\item Alle ligheds krav er aktive
\item der er mindst $n$ lineært uafhængige aktive krav
\end{enumerate}
og en \textbf{Mulig basis løsning} hvis:
\begin{itemize}
\item $\vec{x}^*$ opfylder alle krav.
\end{itemize}
\end{defn}


    

\begin{stn}[Løsningerne er de samme]
$\vec{x}^* \in P$ da er følgende ækvivalent:
\begin{enumerate}[label=(\alph*)]
\item ekstremuspunkt
 \item node
 \item mulig basis løsning
\end{enumerate}
\end{stn}


\begin{kor}
Endeligt antal krav medføre endeligt antal basisløsninger
\end{kor}

\begin{defn}[Naboløsninger]
Naboløsninger, har samme aktive krav pånær et
\end{defn}

\begin{stn}[Er en basisløsning når:]
Krav $A\vec{x}=b$, $\vec{x}\geq 0$, hvor $A$ er $m\times n$ matrix med lineært uafhænginge rækker. Da er $\vec{x}^*\in \mathds{R}^n$  basis løsning hviss $A\vec{x}^*=b, \vec{x}^* \geq 0$, og der $\exists B(1), ..., B(m)$ så
\begin{enumerate}[label=(\alph*)]
\item kolonerne $A_{B(1)}, ..., A_{B_(m)}$ er lineært uafhænginge
\item $x_i = 0$ hvis $i \neq B(1),...,B(m)$
\end{enumerate}
\end{stn}
