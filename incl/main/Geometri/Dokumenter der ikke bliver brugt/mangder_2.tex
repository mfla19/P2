\chapter{Geometri}
\section{Hvilke typer af mængder beskæftiger vi os med?}
\begin{defn} [Polyhedron]
Et \textbf{Polyhedron} er en mængde 
\begin{align*}
 P =\{ \vec{x} \in \mathds{R}^n | A \vec{x} \geq \vec{b}, \vec{b}\in \mathds{R}^m\},
\end{align*}
hvor at $A$ er en $m \times n$ matrice.
\end{defn}

\begin{defn} [Begrænset]
Lad $S \subset \mathds{R}^n$, da er $S$ begrænset, hvis der eksistere en konstant $K$ så $\forall \vec{x} \in S: \vec{x} \leq K$
\end{defn}

\begin{defn}
Lad $ \vec{a} \in \mathds{R}^n$, $\vec{a}\neq \vec{0}$ og $b$ være en skalar, da kaldes en mængde for:
en \textbf{Hyperplane} hvis $\{ \vec{x} \in \mathds{R}^n | \vec{a}^{T}\vec{x} = \vec{b}\}$.
\\ en \textbf{Halfspace} hvis $\{ \vec{x} \in \mathds{R}^n | \vec{a}^{T} \vec{x} \geq \vec{b}\}$
\end{defn}


\begin{defn} [Konveks]
Lad $S \subset \mathds{R}^n$  da er $S$ konveks, hvis der $\forall \vec{x}, \vec{y} \in S$ og et vilkårligt $\lambda \in [0,1]$ gælder at $\lambda \vec{x} + (1-\lambda) \vec{y} \in S$.
\label{def:Konveks}
\end{defn}

\begin{defn}[Konveks kombination og konveks huld]
Lad $\vec{x}^1, ...,\vec{x}^k \in \mathds{R}^n$, og $\lambda_1,..., \lambda_k \geq 0 $ være skalare, som opfylder $\sum_{i=1}^k \lambda_i =1$ da er
\begin{enumerate}[label=(\alph*)]
\item $\sum_{i=1}^k \lambda_i \vec{x}^1$ en \textbf{konveks kombination}.
\\ \item $C_{x} = \{\sum_{i=1}^k \lambda_i \vec{x}^1| \vec{x}^1, ...,\vec{x}^k \in \mathds{R}^n, \sum_{i=1}^k \lambda_i =1\}$ et \textbf{konveks huld} for vektorene $\vec{x}^1, ...,\vec{x}^k$. 
\end{enumerate}
\label{def:KonveksKombination}
\end{defn}

\begin{stn}[Konveks kombination]
Lad $S\subset \mathds{R}^n$ være en konveks mængde, da
\begin{align*}
	\sum_{i=1}^k \lambda_i \vec{x}^i \in S, \qquad \vec{x}^1, ...,\vec{x}^k \in S.
\end{align*}
\label{stn:KonveksKombination}
\end{stn}

\begin{proof}
For at vise Sætning \ref{stn:KonveksKombination} gøres brug af et induktionsbevis.
Lav derfor induktionsstarten ved at betragte den konvekse kombination $\sum_{i=1}^2 \lambda_i \vec{x}^1$.
Da $\sum_{i=1}^2 \lambda_i = \lambda_1 + \lambda_2 = 1$ ifølge Definition \ref{def:KonveksKombination} (a), må $\lambda_2 = (1 - \lambda_1)$.
Det indsættes nu i den konvekse kombination af $\vec{x}^1$ og $\vec{x}^2$, hvorfor at $\lambda_1 \vec{x}^1+ (1-\lambda_1) \vec{x}^2$.
Da $S$ er konveks følger det af Definition \ref{def:Konveks} at $\sum_{i=1}^2 \lambda_i \vec{x}^1 \in S$.
\\Antag derefter at $\sum_{i=1}^k \lambda_i \vec{x}^i \in S$ som induktionshypotesen.
\\ Det vises nu at induktionshypotesen medfører at Sætning \ref{stn:KonveksKombination} også gælder for $k+1$.
Lav derfor induktionstrinnet ved at betragte 
\begin{align*}
	\sum_{i=1}^{k+1} \lambda_i \vec{x}^i &= \lambda_{k+1}\vec{x}^{k+1} + \sum_{i=1}^k \lambda_i \vec{x}^i
	\\ &= \lambda_{k+1}\vec{x}^{k+1} + \frac{1-\lambda_{k+1}}{1-\lambda_{k+1}} \sum_{i=1}^k \lambda_i \vec{x}^i
	\\ &= \lambda_{k+1}\vec{x}^{k+1} + (1-\lambda_{k+1}) \sum_{i=1}^k \frac{1}{1-\lambda_{k+1}} \lambda_i \vec{x}^i
\end{align*}
Observer da at da $\sum_{i=1}^{k+1} \lambda_i \vec{x}^i $ gælder
\begin{align*}
	\sum_{i=1}^{k+1} \lambda_i  & = 1
	\\ \sum_{i=1}^{k} \lambda_i &= 1 - \lambda_{k+1}
	\\ \frac{1}{1-\lambda_{k+1}} \sum_{i=1}^{k} \lambda_i &= \frac{1-\lambda_{k+1}}{1-\lambda_{k+1}} = 1.
\end{align*}
Derfor er $\sum_{i=1}^k \frac{1}{1-\lambda_{k+1}} \lambda_i \vec{x}^i$ en konveks kombination af $k$ elementer, hvorfor det følger af induktionshypotesen at $\sum_{i=1}^k \frac{1}{1-\lambda_{k+1}} \lambda_i \vec{x}^i \in S$, hvorfor at $\sum_{i=1}^{k+1} \lambda_i \vec{x}^i \in S$, og sætningen er bevist.
\end{proof}


\begin{stn}[Konvekse mængder]
Følgende mængder er konvekse
\begin{enumerate}[label=(\alph*)]
\item $A \cap B$, hvis mængderne $A,B$ er konvekse
\\  \item polyhedronet $P =\{ \vec{x} \in \mathds{R}^n | A \vec{x} \geq \vec{b}\} $
\\ \item Konveks huldet $C_x = \{\sum_{i=1}^k \lambda_i \vec{x}^1| \vec{x}^1, ...,\vec{x}^k \in \mathds{R}^n, \sum_{i=1}^k \lambda_i =1\}$ over en endelig mængde vektorer
\end{enumerate}
\end{stn}

\begin{proof}
Først vises udsagn (a).
Lad $\vec{x}, \vec{y} \in A,B$ være to vilkårlige vektorer, da både $A$ og $B$ er konvekse må $\lambda\vec{x} + (1-\lambda) \vec{y} \in A, B$, hvor $ \lambda \in [0,1]$ er en skalar.
Derfor må $\lambda\vec{x} + (1-\lambda) \vec{y} \in  A \cap B$, hvilket medføre af Definition \ref{def:Konveks} at $A \cap B$ er konveks.
\\Så vis udsagn (b).
Lad $\vec{x}, \vec{y} \in P=\{ \vec{x} \in \mathds{R}^n | A \vec{x} \geq \vec{b}\}$ være to vilkårlige vektorer, da gælder at $A\vec{x} \geq \vec{b}$ hvilket medfører at $\lambda A \vec{x} \geq \lambda\vec{b}$, hvor $\lambda \in [0,1]$ er en skalar. 
På ligefod må der derfor gælde at $(1-\lambda)A\vec{y} \geq (1-\lambda)\vec{b}$.
De to uligheder adderes nu
\begin{align*}
\lambda A \vec{x} + (1-\lambda) A \vec{y} \geq \lambda \vec{b} + (1 - \lambda) \vec{b}
\\  A (\lambda\vec{x} + (1-\lambda)\vec{y}) \geq \vec{b}.
\end{align*}
Derfor må $\lambda\vec{x} + (1-\lambda)\vec{y} \in P$.
\\Tilsidst vises udsagn (c).
Lad $\vec{z}, \vec{y}\in C_x = \{\sum_{i=1}^k \lambda_i \vec{x}^i| \vec{x}^1, ...,\vec{x}^k \in \mathds{R}^n, \sum_{i=1}^k \lambda_i =1\}$ være vilkårlige vektorer da må $\vec{z}= \sum_{i=1}^k \gamma_i \vec{x}^i, \vec{y}= \sum_{i=1}^k \eta_i \vec{x}^i$ for $\sum_{i=1}^k \gamma_i = 1$ og  $\sum_{i=1}^k \eta_i = 1$. 
Derfor må
\begin{align*}
	\lambda \vec{z} + (1- \lambda) \vec{y} &= \lambda\sum_{i=1}^k \gamma_i \vec{x}^i + (1-\lambda)\sum_{i=1}^k \eta_i \vec{x}^i
	\\ &=\sum_{i=1}^k (\lambda \gamma_i+(1-\lambda)\eta_i )\vec{x}^i,
\end{align*}
For $\lambda \in [0,1]$.
Betragt nu konstanterne 
\begin{align*}
	\sum_{i=1}^k (\lambda \gamma_i+(1-\lambda)\eta_i ) &= \lambda \sum_{i=1}^k \gamma_i + (1 - \lambda) \sum_{i=1}^k \eta_i 
	\\ &= \lambda \cdot 1 + (1 - \lambda) \cdot 1 = 1
\end{align*}
Hvorfor at $\lambda \vec{z} + (1- \lambda) \vec{y} $ er en konveks kombination af vektorene $\vec{x}^1, ...,\vec{x}^k $, ifølge Definition \ref{def:KonveksKombination}. 
Derfor må $ \lambda \vec{z} + (1- \lambda) \vec{y} \in C_x$, hvorfor at $C_x$ er konveks ifølge Definition \ref{def:Konveks}.
Og sætningen er bevist.
\end{proof}


