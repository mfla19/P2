\chapter{Case}
Teorien omkring store-M metoden vil nu blive anvendt til optimeringen af en specifik case. Problemet som skal optimeres, er tildelingen af arbejdsopgaver til ansatte i en virksomhed. 
I denne case har de ansatte forskellig løn, maksimalt antal timer, samt forskellige effektiviteter i løsningen af de forskellige opgaver. 
Yderligere skal hver opgave løses et bestemt antal gange.
Optimeringsproblemet går med udgangspunkt i disse betingelser ud på at minimere firmaets udgifter til løn. Netop store-M metoden anvendes, da problemet ikke har en åbenlys basisløsning og da problemet herved kan løses med et enkelt optimeringsproblem.

Lad en virksomhed have $G$ ansatte, hvor den $i$'te ansatte maksimalt må arbejde $T_i$ timer, hvor $i=1,...,G$. Virksomheden har $H$ arbejdsopgaver, som hver skal løses $O_j$ gange, hvor $j=1,...,H$.
Herom gælder det at $p_{ij}$ er mængden af tid den $i$'te ansatte bruger på den $j$'te opgave, hvilket den ansatte gør med en effektivitet $n_{ij}$. Derudover har hver ansatte en individuel løn $L_i$.
Dette kan alt sammen ses i følgende programmeringsproblem, som søger at minimere udgifterne:

\begin{center}
	\begin{tabular}{l	>{$}l<{$}}
Minimer			&\sum_{i=1}^G L_i \left( \sum_{j=1}^H p_{ij} \right)\\
\rule{0pt}{4ex}Med hensyn til 	&T_i \geq \sum_{j=1}^H p_{ij},\\
				&O_{j} = \sum_{i=1}^G n_{ij} p_{ij}\\
og $p_{ij} \geq 0.$
	\end{tabular}
\end{center}

Derved består det nødvendige input af vektorer $\vec{L}$, $\vec{T}$ og $\vec{O}$ for henholdsvis løn, timetal og opgavekrav, samt matricen $N$, for effektiviteterne af de ansatte til de forskellige opgaver:
\begin{align*}
	N=\kbordermatrix{
	Ansatte \backslash Opgaver & 1 & 2 & \dots & H\\
	1		&	n_{1,1}	&	n_{1,2}	&	\dots	&	n_{1,H}\\
	2		&	n_{2,1}	&	n_{2,2}	&	\dots	&	n_{2,H}\\
	\vdots	&	\vdots	&	\vdots	&		& 	\vdots\\
	G		&   n_{G,1}	&	n_{G,2}	&	\dots	&	n_{G,H}
	}
\end{align*}


Ved anvendelse af store M metoden til løsningen af dette problem, skal der anvendes slack-variable og kunstige variable. Enhver ulighed skal omskrives til en lighed, hvilket kræver indførslen af en slack-variabel. Da begrænsningerne for de ansattes timetal er mindreendbetingelser kan disse omskrives til:
$$T_i = \sum_{j=1}^H p_{ij}+s_i$$
I store M metoden kræver denne begrænsning ikke en ekstra variabel med cost M, da $s_i$ gerne må være over 0 i den endelige løsning.

Begrænsninger af typen
$O_{j} = \sum_{i=1}^G n_{ij} p_{ij},$
kræver en ekstra variabel $a_j$ med cost M, for at kunne danne en begyndende basisløsning. Derved omskrives disse begrænsninger til
$$O_{j} = \sum_{i=1}^G n_{ij} p_{ij}+a_j$$
disse variable $a_j$ må ikke have en værdi over 0 i den endelige basis, da de originale ligheder derved ikke overholdes. Derfor får disse variable en cost M, så de reduceres til 0, hvis dette er muligt.

Derved kan problemet omskrives til:
\begin{center}
	\begin{tabular}{l	>{$}l<{$}}
Minimer			&\sum_{i=1}^G L_i \left( \sum_{j=1}^H p_{ij} \right)+M\sum_{j=1}^H a_j\\
\rule{0pt}{4ex}Med hensyn til 	&T_i = \sum_{j=1}^H p_{ij} + s_i\\
				&O_{j} = \sum_{i=1}^G n_{ij} p_{ij}+a_j\\
og $p_{ij} \geq 0.$
	\end{tabular}
\end{center}

Der vil derved være $G \cdot H$ variable $p_{ij}$, $G$ slack-variable $s_i$, og $H$ kunstige variable $a_j$. Dette giver i alt $G \cdot H+G+H$ variable i $G+H$ betingelser. 

\section{Konstruktion af simplex tabellen for casen}
Ved konstruktionen af den fulde tabel tilføjes endnu en række og søjle. Derved får tabellen størrelse $(G+H+1) \times (G\cdot H+G+H+1)$. 
Løsningsvektoren og cost vektoren bliver derved:

\begin{align*}
\vec{x}^T &= \ \ \rvect{p_{11} ... p_{1H} & ... & p_{G1} ... p_{GH} & s_1 ... s_G & a_1 ... a_H}.\\
\vec{c}^T &=\kbordermatrix{
& \times H & & \times H & \times G & \times H \\
&L_1 & ... & L_G & 0 & M
},
\end{align*}
hvor f.eks. $\times H$ betyder at denne indgang udgør $H$ indgange af vektoren.

Ved introduktionen af slack-variable og kunstige variable, kan disse derved udgøre den første basis for ligningssystemet. Derved bliver basisindeks og basisvektor til:
\begin{align*}
&\vec{I}_B^T=\rvect{s_1 ... s_G & a_1 ... a_H},\\
&\vec{x}_B^T=\rvect{T_1 ... T_G & O_1 ... O_H},
\end{align*}
da $p_{ij}=0$ for $i=1,2,...,G$ og $j=1,2,...,H$, og da leddene $s_i$ og $a_j$ derved er de eneste ikke-nul led i betingelserne.


Betingelserne kan indskrives da i matricen $A$. For et problem med $G=2$ og $H=3$ gælder det derved at
\begin{align*}
B^{-1}A=\kbordermatrix{
&p_{11} & p_{12} & p_{13} & p_{21} & p_{22} & p_{23} & s_1 & s_2 & a_1 & a_2 & a_3\\
&1       & 1      & 1      & 0      & 0      & 0      & 1 & 0 & 0 & 0 & 0 \\
&0       & 0      & 0      & 1      & 1      & 1      & 0 & 1 & 0 & 0 & 0 \\
&n_{11}  & 0      & 0      & n_{21} & 0      & 0      & 0 & 0 & 1 & 0 & 0 \\
&0       & n_{12} & 0      & 0      & n_{22} & 0      & 0 & 0 & 0 & 1 & 0 \\
&0       & 0      & n_{13} & 0      & 0      & n_{23} & 0 & 0 & 0 & 0 & 1
}_,
\end{align*}
da $B^{-1}=B=I_{G+H}$ og da det derved gælder at $B^{-1}A=I_{G+H}A=A$.\\

Den reducerede cost for en ændring af $x_j$ er givet som $\Delta c_j=\vec{c}_j-\vec{c}_BB^{-1}\vec{A}_j.$

For denne case og opstilling af $B^{-1}A$ er det muligt at simplificere denne udregning.

Variablen $p_{ij}$ har i $A$ og $\vec{c}$ et indeks $k=(i-1)\cdot H+j$. Den reducerede cost for en sådan variabel kan findes som:
\begin{align*}
	\Delta c_{k} \ &=  \vec{c}_{k}-\vec{c}_B B^{-1}\vec{A}_{k}\\
	&= \ L_i-\vec{c}_B \vec{A}_{k} \\
	&= \ L_i-M \cdot n_{ij},
\end{align*}
da søjlevektoren for en given variabel $p_{ij}$ kun har 2 ikke-nul indgange, hvoraf den korresponderende cost for den ene af dem er lig 0.

Derved bliver den reducerede cost vektor til:
\begin{align*}
\bar{c}=	& \kbordermatrix{
& & & & \times (G+H)\\
&L_1-Mn_{11} \ \dots \ L_1-Mn_{1H} & \dots & L_G-Mn_{G1} \ \dots \  L_G-Mn_{GH} & 0}
\end{align*}

Den første simplex tabel kan hermed for et problem med $G=2$ og $H=3$ opskrives som:\\

\scalebox{0.8}{
\begin{tabular}{| >{$}l<{$} | >{$}l<{$}>{$}l<{$}>{$}l<{$}>{$}l<{$}>{$}l<{$}>{$}l<{$}>{$}l<{$}>{$}l<{$}>{$}l<{$}>{$}l<{$}>{$}l<{$} |}
\hline
-(O_1+O_2+O_3)M	&L_1-Mn_{11} &L_1-Mn_{12} &L_1-Mn_{13} &L_2-Mn_{21} &L_2-Mn_{22} &L_1-Mn_{23} &0 &0 &0 &0 &0\\
\hline
T_1	&1       & 1      & 1      & 0      & 0      & 0      & 1 & 0 & 0 & 0 & 0 \\
T_2 &0       & 0      & 0      & 1      & 1      & 1      & 0 & 1 & 0 & 0 & 0 \\
O_1	&n_{11}  & 0      & 0      & n_{21} & 0      & 0      & 0 & 0 & 1 & 0 & 0 \\
O_2	&0       & n_{12} & 0      & 0      & n_{22} & 0      & 0 & 0 & 0 & 1 & 0 \\
O_3 &0       & 0      & n_{13} & 0      & 0      & n_{23} & 0 & 0 & 0 & 0 & 1\\
\hline
\end{tabular}
}


\section{Løsning af problemet}
%For at løse optimeringsproblemet dannes den fulde tabel ligesom i fuld tabel metoden, hvorved den også kan løses på samme måde.
For netop denne case er der nogle faktorer, som betyder at problemet bliver simplificeret lidt.
Da alle indgange i $B^{-1}$ og $\vec{x}_B$ er positive, vil alle rækker ifølge Sætning \ref{stn:lexi} forblive lexi-positive, hvorved der ikke er brug for at bytte om på søjlerne, for at kunne anvende den lexikografiske metode.
Da der findes en unik variabel i hver betingelse, garanterer dette at rækkerne er lineært uafhængige, hvorved de fundne basisløsninger nødvendigvis har dimension 0. Derved er det ikke nødvendigt at tage højde for eventuelle tilfælde med lineært afhængige rækker. Da alle variable har en cost større end 0, og da ingen af disse kan være negative, er det heller ikke muligt at opnå tilfælde med uendeligt lav cost. Derved er det udelukkende nødvendigt at undersøge 2 scenarier: 
\begin{itemize}
\item Problemet har en optimal løsning.
\item Problemet har ingen løsning.
\end{itemize}


Til løsningen af problemer af denne type er der dannet følgende algoritmer med funktioner, som finder en optimal løsning. Algoritme \ref{alg:lexi}, finder den lexi-mindste række i tabellen for de rækker som har en indgang over nul i pivotsøjlen. 

Algoritme \ref{alg:storem2} er den overordnede algoritme, som laver nye iterationer indtil en optimal løsning er fundet. Yderligere defineres to konstanter m og n, som er henholdsvis højde og bredde af den fulde tabel. Her gælder det i algoritmerne at matrixen A er den fulde tabel.

\newpage

\begin{alg}[label={alg:storem2}]{Store-M algoritmen}
def $\textbf{storeM}$(A,I_B):
    mens sandt: #kører iterationer indtil løsning fundet
    	piv_søjle = 0
    	for søjle i range(1,n-1): 
        	hvis A[0][søjle] < 0: # finder søjle med ${\color{commentgreen} \overline{c}_i}$<0
            	piv_søjle = søjle
            	break
		hvis piv_søjle = 0: #hvis ingen reduceret cost er negative så afsluttes
			break
            
    	piv_række=$\textbf{lexi}$(A,piv_søjle) #finder den lexi-mindste række
            
    	piv_punkt = A[piv_række][piv_søjle] 
    	for søjle i range(n): #dividerer piv række med piv punkt
        	A[piv_række][søjle] /= piv_punkt
            
    	for række i range(m): #trækker piv række fra andre rækker
        	hvis række != piv_række:
            	række_trækfra = A[række][piv_søjle]
            	for søjle i range(n):
                	A[række][søjle] -= række_trækfra*A[piv_række][søjle]

    	I_B[piv_række-1]=piv_søjle #ny variabel indsættes i indeks for basis

	for indeks i $I_B$: 
		hvis indeks > G*H+G: #kunstig variabel stadig i basis. 
			hvis A[indeks+1][0] > 0: #kunstig variabel større end 0 giver uendeligt stor cost
				returner cost = $\infty$
	ellers: #løsning fundet
		returner cost = $-A[0][0]$
\end{alg}

\begin{alg}[label={alg:lexi}]{Lexi pivot algoritme}
def $\textbf{lexi}$(A,piv_søjle):
	minrække = [række for række i $I_B$ for hvilke A[række][piv_søjle] > 0]
	
	for søjle i range(n):
		$\theta$ = [A[række][søjle]/A[række][piv_søjle] for række i minrække]
		minrække = [rækker i minrække for hvilke $\theta$[række] = min($\theta$)]
		hvis længde af minrække == 1:
			returner minrække[0] #række fundet og returneres
\end{alg}




\begin{eks}[Case eksempel]
Lad en virksomhed have 3 medarbejdere, som skal have udført 4 opgaver og lad parametrene være givet som:
\begin{align*}
\vec{L}=&	\rvect{120 & 200}\\
\vec{T}=&	\rvect{15 & 37}\\
\vec{O}=&	\rvect{200 & 60 & 70}\\
N=&\begin{bmatrix}
10	&2	&7\\
15	&3	&8\\
\end{bmatrix}
\end{align*}
Den fulde tabel for dette problem får derved følgende fulde tabel:\\

\scalebox{0.9}{
\begin{tabular}{| >{$}l<{$} | >{$}l<{$}>{$}l<{$}>{$}l<{$}>{$}l<{$}>{$}l<{$}>{$}l<{$}>{$}l<{$}>{$}l<{$}>{$}l<{$}>{$}l<{$}>{$}l<{$} |}
\hline
-330M	&120-10M &120-2M 	&120-7M	&200-15M &200-3M &200-8M & 0 & 0 & 0 & 0 & 0\\
\hline
15		&1		& 1    	& 1    	& 0      & 0     & 0     & 1 & 0 & 0 & 0 & 0\\
37 		&0  	& 0    	& 0     & 1      & 1     & 1     & 0 & 1 & 0 & 0 & 0\\
200		&10  	& 0    	& 0     & 15 	 & 0     & 0     & 0 & 0 & 1 & 0 & 0\\
60		&0  	& 2    	& 0     & 0      & 3 	 & 0     & 0 & 0 & 0 & 1 & 0\\
70 		&0  	& 0		& 7 	& 0      & 0     & 8 	 & 0 & 0 & 0 & 0 & 1\\
\hline
\end{tabular}
}\\

Ved anvendelse af Algoritme \ref{alg:storem2} findes følgende resultat for arbejdstiderne.

\begin{align*}
P=\begin{bmatrix}
5 & 0 & 10\\
10 & 20 & 0
\end{bmatrix}
\end{align*}

Det ses at denne løsning opfylder alle de originale betingelser:

\begin{center}
\begin{tabular}{>{$}l<{$}>{$}l<{$}>{$}l<{$}>{$}l<{$}>{$}l<{$}>{$}l<{$}>{$}l<{$}}
p_{11} 		& p_{12} 	& p_{13} 	& p_{21} 	& p_{22} 	& p_{23} 	&\\
1\cdot 5	& 1 \cdot 0	& 1\cdot 10	&			&			&			& \leq 15	\\
			&			&			&1\cdot 10 	&1\cdot 20	& 1\cdot 0	& \leq 37	\\
10\cdot 5 	&			&			&15\cdot 10 &			&			& =200	\\
			&2 \cdot 0	&			& 			&3\cdot 20	&			& =60	\\
			&			&7 \cdot 10	&			&			&8\cdot 0	& =70	\\
\end{tabular}
\end{center}

\end{eks}

Ved at anvende denne algoritme er det herved nemt at optimere problemer med arbejdsfordeling, da dette yderligere giver en del forsimplinger i anvendelsen af store-M metoden. 
