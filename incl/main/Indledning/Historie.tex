%kilde: Bertsimas
Lige siden industrialiseringen, har fabriker ønsket at optimere deres arbejdsproduktion og maksimere deres indtjeninger. 
Til det formål kan lineær programmering benyttes.
Lineære programmerings problemer blev formuleret af den sovjetisk matematiker, Leonid Kantorovich, da han i optakten til og starten af anden verdenskrig blev  interesseret i, den optimale ressourcefordeling i en planøkonomi, for at formindske sovjetunionens omkostninger og maksimere fjendens udgifter. Kantorovich gav også en løsningsmetode, men denne løsning blev ikke kendt daværende tidspunkt.\\
Samtidig med Kantorovich's formulering og løsning, kom  mange andre løsninger på banen, og en af dem kom fra den Hollandsk-amerikanske økonom ved navn Tjalling Koopmans. Koopmans formulerede dog problemet i forhold til en klassisk økonomi-model. 
Og de to matematikkere Koopmans og Kantorovich delte i 1975 en nobelpris i økonomi.\\
Det næste store gennembrud kom i 1947, en tid, hvor informationsteknologi var i udvikling, her forslog matematikeren George Dantzig  en algoritme, Simplex Metoden, som gjorde løsning af lineære programmeringsproblemer mere praktisk. 
\\%lay
Lineær programmering spillede også en vigtig rolle under den kolde krig, hvor Sovjetunionen havde indemuret Vestberlin.
Derfor kunne vesten ikke transportere resurser ind i Vestberlin på andre måder end via luften.
Fly kunne dog ikke transportere et uendeligt antal varer ind i Vestberlin, fordi flyene både var begrænsede af benzin og plads. 
Transporten af disse resurser var derfor optimeringsproblemet. \citep{bert} \citep{lay}