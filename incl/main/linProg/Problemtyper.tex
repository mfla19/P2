\section{Standard maksimums- og minimumsproblemer}
I et standard maksimumsproblem gælder det, for alle bibetingelserne, at den lineære funktion af variablene er mindre end eller lig en konstant, mens de i et standard minimumsproblem gælder skal være større end eller lig en konstant. For begge problemer er variablene positivt begrænsede. 

Som beskrevet i Afsnit \ref{afsnit:lign_sys} kan et lineært ligningssystem opskrives som et matrix-vektor produkt. Matricen repræsenterer koefficienterne i det lineære ligningssytem, mens variablene skrives som en vektor. Tilsvarende gælder det for objektfunktionen, at denne kan skrives som et produkt af to vektorer. Dette tillader definitionen af et standard maksimumsproblem med disse produkter i Definition \ref{def:std_maksmin}. 

\begin{defn}[Standard maksimums- og minimumsproblemer]
	Lad $\vec{x}= \rvect{x_1 & x_2 & \cdots & x_n}^T$ være \textbf{løsningsvektoren} med koefficienter $\vec{c}= \rvect{c_1 & c_2 & \cdots & c_n}^T$ i objektfunktionen, og lad $m \times n$ matricen $A=[A_{ij}]$ for $i=1,...,m$ og $j=1,...,n$, og lad $\vec{b}=\rvect{b_1 & b_2 & \cdots & b_m}^T$.\\
Da er standard maksimumsproblemet defineret som
\begin{center}
\begin{tabular}{l	>{$}l<{$}}
Maksimer 		& \vec{c}^T\vec{x} \\
med hensyn til 	& A\vec{x} \leq \vec{b}\\
og 				& \vec{x} \geq \vec{0},
\end{tabular}
\end{center}
og standard minimumsproblemet er defineret som
\begin{center}
\begin{tabular}{l	>{$}l<{$}}
Minimer			& \vec{c}^T\vec{x} \\
med hensyn til 	& A\vec{x} \geq \vec{b}\\
og 				& \vec{x} \geq \vec{0}.
\end{tabular}
\end{center}
\label{def:std_maksmin}
\end{defn}
Består ligningssystemet af ligheder, kaldes problemet henholdvist enten et \textbf{standard maksimumsproblem med ligheder}, eller et \textbf{standard minimumsproblem med ligheder}. \\
Ethvert lineært programmeringsproblem kan således omskrives, så det står på standardform. 

\begin{eks}[Standard maksimumsproblem]
Hvis eksempel \ref{eks:maksprob1} skal omskrives til et standard maksimumsproblem, skal alle relationer i bibetingelserne være mindre end eller lig med konstanterne.
Bibetingelse nr. 3 skal derved omskrives. Dette gøres ved at multiplicere begge sider af uligheden med $-1$, da dette vender ulighedstegnet. Derved bliver Eksempel \ref{eks:maksprob1} omskrevet til et standard maksimumsproblem.
\begin{center}
\begin{tabular}{l	>{$}r<{$}	>{$}r<{$}	>{$}l<{$}}
Maksimer 		& 		4x_1	&	+3 x_2	& \\
med hensyn til 	&  \ \ 	-2 x_1	& 	+4 x_2	& \leq 8\\
				&  		x_1		& 	+3 x_2	& \leq 16\\
				&  \ \ 	x_1		& 			& \leq 10\\
og $x_1 \geq 0, x_2\geq 0$.
\end{tabular}
\end{center}
\label{eks:maksprob2}
\end{eks}

Bemærk, at på samme måde som bibetingelserne kan omskrives, kan et minimeringsproblem ligeledes laves til et maksimeringsproblem, ved at multiplicere med $-1$.



\begin{comment}
\begin{defn}[Standard minimum problem]
	Lad $\vec{x}= [x_1, x_2,\cdots, x_n]^T$ være \textbf{løsningsvektoren}, med koefficienter $\vec{c}= [c_1, c_2,\cdots, c_n]^T$ i objektfunktionen, og lad $m \times n$ matrixen $A=[A_{ij}]$ for $i=1,2,\cdots,m$ og $j=1,2,\cdots,n$ være begrænset af konstanterne $\vec{b}=[b_1, b_2,\cdots, b_m]^T$.
	Da er standard minimum problemet defineret som\\
\begin{center}
\begin{tabular}{l	>{$}l<{$}}
Minimer			& \vec{c}^T\vec{x} \\
med hensyn til 	& A\vec{x} \geq \vec{b}\\
og 				& \vec{x} \geq \vec{0}
\end{tabular}
\end{center}
\label{def:std_min}
\end{defn} %Hvorfor byttes der rundt på b og c? i min og max?


%%%%%%%%%%%%%%%%%%%%%%%%%%%%%%%%%%%%%%%%%%%%%%%%%%%%%%%%%%%%%%%%%%%%%%%%%%%%%%%%%%%%%%%%%%%%%%%%%%%%%%%%
%%%%%%%%%%%%%%%%%%%%%%%%%%%%%%%%%%%%%%%%%%%%%%%%%%%%%%%%%%%%%%%%%%%%%%%%%%%%%%%%%%%%%%%%%%%%%%%%%%%%%%%%%%

Det betyder at ethvert lineært programmeringsproblem kan omskrives, så det står på standardform.
\begin{defn}[Standard minimumsproblemer]
Lad $f(\vec{x}) = \vec{c}^T\vec{x}$ betegne objektfunktionen til et lineært minimeringsproblem, for $\vec{x},\vec{c} \in\mathds{R}^n$, lad $m \times n$ matricen $A=[A_{ij}]$ for $i=1,...,m$ og $j=1,...,n$, og lad $\vec{b} \in  \mathds{R}^n$.
Der er standard minimumsproblemet defineret som\\
\begin{center}
\begin{tabular}{l	>{$}l<{$}}
Minimer			& \vec{c}^T\vec{x} \\
med hensyn til 	& A\vec{x} \geq \vec{b}\\
og 				& \vec{x} \geq \vec{0}.
\end{tabular}
\end{center}
Består ligningssystemet af ligheder kaldes problemet; \textbf{standard minimumsproblem med ligheder}.
\label{def:std_maksmin}
\end{defn}
Bemærk at på samme måde som bibetingelserne kan omskrives, kan et minimeringsproblem laves til et maksimeringsproblem ved at multiplicere med $-1$, hvorfor at alle definitioner og sætninger er ækvivalente. Derfor defineres begreber og sætninger kun for minimums problemer, mens eksemplerne illustrere maksimums problemer.
\end{comment}