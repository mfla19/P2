\chapter{Konklusion}
I projektet er der blevet arbejdet med lineære programmerings problemer og deres løsninger.
Gennem kendskabet til geometriske løsninger af lineære programmerings problemer kan casen løses, hvis den indeholder en til 3 variable.
Da de sjældent er tilfældet, introduceres Simplex Metoden, som bygger på den geometrisk metode med at beskrive basisløsninger som simplexer, og 
forskellige implementerings muligheder undersøges. 
Her vurderes, at det ville være favorabelt at benytte Store-M metoden sammen med Den Leksikografiske Pivot Regel til at løse optimeringen af lønningsudgifterne i en virksomhed.
De implementeres derfor i et python program, og resultatet for optimeringen af af lønnings udgifterne af en virksomhed med 2 medarbejdere, som skal have udført 3 opgaver. 
Hvis den ene arbejder 5 timer på første opgave og 10 timer på den tredje, og den anden arbejder 10 timer på den første og 20 timer på den anden vil det resultere i den optimale værdi, som kan beregnes til 7800 kr. Den optimale værdi svare til den mindste løn som skal udbetales.