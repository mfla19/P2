\chapter{Simplex}

Hvis et lineært programmeirngproblem har en optimal mulig løsningen, så findes der en basis mulig løsning som er optimal. 
Simplex metoden søger efter denne optimale basis mulige løsning, ved at bevæge sig mellem de mulige basis løsninger. 
På et tidspunkt vil den optimale løsning være fundet, da der kun er et endeligt antal basis mulige løsninger. 
Denne søgen efter den optimale basis løsning kan gøres ved hjælp af forskellige typer Simplex. 
Nogen af disse vil blive gennemgået i dette kapitel. 

\section{kommer senere}
De algoritmer som bruges til at løse dette problem er ofte struktureret således at de med udgangspunkt en allerede kendte basis mulig løsning. Hvorefter der søges efter en bedere løsning i nærheden. 

Betragt en basis mulig løsning $\vec{x} \in P$. 
Det ønskes at rykke til en anden mulig løsning, dette gøres i rettingen af en vektor $\vec{d} \in \mathds{R}^n$. Det giver ikke mening at betragte de vektorer som med det samme kommer udenfor polyhedronet, $P$. Det er derfor kun de mulige retninger som betrages. 

\begin{defn}[Mulig retning]
Lad $\vec{x}$ være en vektor i $P$. En vektor $\vec{d}$ er en mulig retningen fra $\vec{x}$, hvis der findes en positiv scalar $\theta$ således at $\vec{x}+\theta\vec{d} \in P$ 
\end{defn}

Da polyhedronet er en konveks mængnde må det gælde at, hvis $\vec{x} \in P$ og $\vec{d}$ er en mulig retning, så er $\vec{x}+ \theta^{*} \vec{d} \in P$ for $\theta^{*} \in ] 0,\theta ] $