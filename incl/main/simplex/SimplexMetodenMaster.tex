\chapter{Simplex Metoden}\label{chap:simp}
Dette kapitel er skrevet med udgangspunkt i \citep{bert}, medmindre andet er angivet. \\
Af forrige kapitel fremgik det, at den optimale løsning altid vil være en mulig basisløsning, og af Sætning \ref{stn:kravtilbasis}
fremgår det, hvordan en basisløsning findes. 
Problemet er, at hvis alle basisløsninger skal testes for et standard problem med uligheder, skal optil $\binom{n}{m}$ muligheder testes. 
Derfor er det nødvendigt, at finde en metode, hvorpå den optimale løsning kan findes uden at skulle kigge på hvert enkelt tilfælde. 
En af metoderne, som kan anvendes, er Simplex metoden. I dette kapitel vil Simplex metoden, og forskellige implementeringer af metoden, derfor blive gennemgået med udgangspunkt i .

\section{Udledning af Simplex metoden}
Simplex metoden starter med en basisløsning, og går så i en mulig retning, indtil den finder en ny basisløsning.

\begin{defn}[Mulig retning]
Lad $\vec{x} \in P$. En vektor $\vec{d}$ er en \textbf{mulig retning} fra $\vec{x}$, hvis der eksisterer en skalar $\theta > 0$, således at $\vec{x}+\theta\vec{d} \in P$ 
\end{defn}

Derfor skal den mulige retning introducere en ny basisvariabel til løsningen.

\begin{lma}[Den $j$'te retningsvektor]
Lad $\vec{x} \in P$ være en basisløsning med basis matrix $B$, og lad $\vec{d}_j  \in \mathds{R}^n$ være en mulig retningsvektor, så $\vec{x}' = \vec{x}+ \theta\vec{d}_j$ introducerer basisvariablen $x_j'$ og kun $x_j'$ til løsningen, for en skalar $\theta$.
Da vil $\vec{d}_j$ være givet; så $\vec{d}_{j,B} = -B^{-1}\vec{A}_j$, $d_{j,j} = 1$ og $d_{i,j} = 0$ for $i \neq j$ og $ i,j \notin I_B$.
\label{lma:retningsvektor}
\end{lma}

\begin{proof}
Hvis $\vec{x}'$ skal introducere $x_j'$ til løsningen skal $x_j + \theta d_{j,j} > 0$. 
Da $x_j = 0$ ifølge antagelsen om, at $\vec{x}$ er en basisløsning, og $\theta$ er en skalar, kan $d_j$ sættes lig $1$ uden at miste generalitet. 
På samme måde må $d_i = 0$ for $i \neq j$ og $i \notin I_B$, da $x_i' = 0$, for ikke at introducere $x_i'$ til løsningen.
Da $\vec{d}$ er en mulig retningsvektor, vil $\vec{x}' \in P$, hvormed at $A\vec{x}' = A\vec{x}+ \theta A\vec{d}_j = \vec{b}$. 
Fordi $A\vec{x} = \vec{b}$, medfører det, at $A\vec{d}_j = \vec{0}$.
Det følger derfor, at
\begin{align*}
A\vec{d}_j = B \vec{d}_B + \sum_{i \notin I_B} \vec{A}_id_i = B\vec{d}_B + \vec{A}_j = \vec{0} \Rightarrow
\\ \vec{d}_j = -B^{-1}\vec{A_J}.
\end{align*}
Bemærk at $B^{-1}$ eksisterer ifølge Sætning. 
Lemmaet er hermed bevist.
\end{proof}

Den nye vektor $\vec{x}'$ må kun have $m$ basisvariable for at være en basisløsning, og da $\vec{d}_j$ introducerer en ny variabel til løsningen, skal $\theta^*\vec{d}$ også fjerne en variabel fra  løsningen.

\begin{lma}[Den $j$te skalar]
Lad $\vec{x} \in P$ være en basisløsning med basis matrix $B$, og lad $\vec{d}_j$ være den $j$'te retningsvektor, så $I_d = \{i \mid d_i < 0, \, i  \in I_B\}$, og $\theta^* = \min_{i \in I_d}\{\frac{x_i}{|d_i|}\}=\frac{x_l}{|d_l|} > 0$ . 
Så vil $\vec{x}' = \vec{x}+ \theta^* \vec{d}_j \in P$ fjerne $x_l'$ fra løsningen, hvis $P$ ikke indeholder en positiv halvlinje.
\label{lma:skalar}
\end{lma}

\begin{bem}
Da $\vec{d}_j$ er en mulig retningsvektor, er $x_i\neq 0$ for $i \in I_d$, hvorfor $\theta^* > 0$.
\end{bem}

\begin{proof}
Antag først, at $I_d = \emptyset$, da vil $x_i' \geq 0 \ \forall \ \theta \geq 0$. 
Hvormed, at $\vec{x}' \in P \ \forall \ \theta \geq 0$, og $P$ vil derfor indeholde en positiv halvlinje, hvormed det kan konkluderes, at $P$ ikke er tom, og $\theta^*$ eksisterer.
Derfor må $x_l' = x_l + \theta^* d_l = x_l + \frac{x_l}{|d_l|}d_l = x_l - x_l = 0$ dermed fjerne $\vec{x}'$ $x_l$ fra løsningen.
Til sidst vises, at $\vec{x}' \in P$. Betragt derfor $x_i' = x_i + \theta^* d_i$ for $i \neq l$, $i \in I_d$ da vil
\begin{align*}
x_i' = x_i + \theta^* d_i = x_i + \frac{x_l}{|d_l|}d_i \geq x_i + \frac{x_i}{|d_i|}d_i = 0,
\end{align*}
da $d_i < 0$ og $\frac{x_l}{|d_l|} \leq \frac{x_i}{|d_i|}$, hvorfor alle indgange forbliver ikke-negative, og $\vec{x}'$ er dermed en mulig løsning.
\end{proof}

\begin{bem}
Lad $x'_j = \theta^* >  0$, da $d_j = 1$ og $\theta^* > 0$. Da fjerner $\vec{x}'$ ikke $x_j'$ fra løsningen.
\end{bem}

Dermed kan $\vec{x}'$ at have $m$ basisvariable under antagelsen om, at $\vec{x}'$ ikke er degenerativ. Er $\vec{x}'$ degererate, ville der ikke være et entydigt indeks $l$ som opfyldte at $min_{i \in I_d}\{\frac{x_i}{|d_i|}\}=\frac{x_l}{|d_l|}$, hvormed at flere af basisvariablene vil blive $0$. 

\begin{stn}
Lad $\vec{x}\in P$ være en mulig basisløsning, med basismatrix $B$, da vil $\vec{x}' = \vec{x}+ \theta^*\vec{d}_j$ også være en mulig basisløsning, hvis $P$ ikke indeholder en positiv halvlinje.
\end{stn}

\begin{proof}
For at vise, at $\vec{x}'$ er en basisløsning, bemærkes at $x_i = 0$ for $i \notin I_{B'} = I_B\setminus\{l\}\cup\{j\}$ samt, at basismatricen for $\vec{x}'$ er matricen $B'$ med søjlerne $\vec{A}_i$ for $i \in I_{B'}$. 
Det følger derfor af Sætning \ref{stn:kravtilbasis},
at $\vec{x}'$ er en basisløsning, hvis søjlerne i $B'$ er lineært uafhængige.
Antag derfor for modstrid, at søjlerne er lineært afhængige, da følger det af Definition \ref{defn_lin_uafh},
at
\begin{align*}
 \sum_{i \in I_{B'}} \lambda_i \vec{A}_i = \vec{0}.
\end{align*}
Da $B$ og $B'$ kun er forskellige i en søjle, må
\begin{align*}
 \\ \sum_{i \in I_{B'}}  B^{-1} \lambda_i \vec{A}_i  =\sum_{i \in I_{B'}\setminus \{j\}} \vec{e_i} + B^{-1} \lambda_j \vec{A}_j = \vec{0}.
\end{align*}

Det vil sige, at søjlerne i $B'$ er lineært uafhængige, hvis $A_{jl} = 0$.
Bemærk, at $B^{-1} \lambda_j A_j = - \vec{d}_B$, hvis $l$'te indgang pr. definition er skarpt mindre end $0$, og $A_{jl} \neq 0$.
Alle søjler i $B'$ er altså lineært uafhængige, hvorfor alle rækker er lineært uafhængige, og $\vec{x}'$ er en basisløsning.
Bemærk, at da $\vec{x}\in P$ og $P$ ikke indeholder en positiv halvlinje, vil $\vec{x}' \in P$ ifølge Lemma \ref{lma:skalar}, hvormed $\vec{x}'$ er en mulig basisløsning.
\end{proof}

Det er nu vist, hvordan det er muligt at gå fra én basisløsning til en anden.
Bemærk, at de to løsninger er naboløsninger, da det er antaget, at ingen af dem er degenerative, og der derfor kun bliver introduceret én ny variabel og fjernet en gammel, hvorfor løsningerne må dele samme krav pånær ét. 
Det er nu nødvendigt at finde en måde, så den basisløsning, der findes, minimerer objektfunktionen mere end den foregående. Ellers vil alle basisløsninger stadig skulle tjekkes. 

\begin{stn}[Ændring i omkostning]
Lad $\vec{x}$ være en basisløsning, med basismatrix $B$, da er ændringen i objektfunktionen, ved at introducere $x_j$ til løsningen, givet ved
\begin{align*}
 \Delta c_j = c_j-\vec{c}_B B^{-1}\vec{A_j}.
\end{align*}
\label{stn:Deltac}
\end{stn}

\begin{proof}
Antag først, at $x_j$ ikke er en basisvariabel til at starte med, da vil den $j$'te retningsvektor introducere $x_j$ til løsningen, hvorfor,
\begin{align*}
\Delta c_j = \vec{c}^T(\vec{x}+ \vec{d}) - \vec{c}^T\vec{x} = \vec{c}^T\vec{d} = \sum_{i \in I_B} c_i d_i + c_j.
\end{align*}
Lad nu $\vec{c}_B = \rvect{c_{B(1)}& \cdots & c_{B(m)}}$, da vil
\begin{align}
\Delta c_j =\vec{c}_B\vec{d}_B+ c_j = c_j-\vec{c}_B B^{-1}\vec{A_j}.
\end{align}
Antag nu, at $x_j$ er en basisvariabel, da vil 
\begin{align*}
\Delta c_j = \vec{c}^T\vec{x}- \vec{c}^T\vec{x} = 0.
\end{align*}
Det undersøges derfor, om $ c_j-\vec{c}_B B^{-1}\vec{A_j}= 0$, hvis $x_j$ er en basisvariable.
\begin{align*}
 \Delta c_j = c_j-\vec{c}_B^T B^{-1}\vec{A_j} = c_j - \vec{c}_B^T \vec{e_j} = c_j - c_j = 0.
\end{align*}
Det kan derfor konkluderes, at $\Delta c_j = c_j-\vec{c}_B B^{-1}\vec{A_j}$ for ethvert $x$.
\end{proof}

Det må derfor være bedst at gå enten i den retning, hvor $\Delta c_j$ er størst, eller hvor $\theta^*\Delta c_j$ er størst. 
Hvis der ikke sker en ændring, eller ændringen forstørrer objektfunktionen, må den fundne basisløsning nødvendigvis være optimal.

\begin{stn}[Optimal omkostning]
Lad $\vec{x}$ være en basisløsning med basismatrix $B$, så er $\vec{x}$ optimal, hvis og kun hvis $\Delta c_j \geq 0$ for alle $j$.
\label{stn:optimalDeltaC}
\end{stn}.

\begin{proof}
Antag først, at $\vec{x}$ er optimal, da vil $\vec{c}^T\vec{x} \leq \vec{c}^T\vec{y}$ for alle $\vec{y} \in P$. 
Antag for modstrid, at $\Delta c_j < 0$ for en basisvariabel $x_j$, da følger det, at $\vec{c}^T(\vec{x}+\theta^*\vec{d}) = \vec{c}^T\vec{x} + \Delta c_j \leq \vec{c}^T\vec{x}$, hvilket strider mod antagelsen om, at $\vec{x}$ er optimal.
Antag nu, at $\Delta c_j \geq 0$, og lad $\vec{y}$ være en vilkårlig vektor i $P$, samt lad $\vec{d}=\vec{x}'-\vec{x}$.
Bemærk, at $A\vec{d} = A\vec{x}'- A\vec{x} =  \vec{b} - \vec{b} =\vec{0}$.
Lad nu $\vec{d}_B = \rvect{d_{B(1)} & \cdots & d_{B(m)}}^T$, og lad $N= \{i \leq n| i \neq B(1),...,B(m)\}$, da kan matrix-vektorproduktet omskrives til:
\begin{align*}
	B\vec{d}_B + \sum_{i \in N} \vec{A}_i d_i = \vec{0} \qquad \Rightarrow
	\\ \vec{d}_B = - B^{-1}\sum_{i \in N} \vec{A}_i d_i
\end{align*}
Bemærk, at det benyttes, at $B$ er invertibel ifølge Sætning. %Mangler
Da tages objektfunktionen til $\vec{d}$ for at finde omkostningen, af at gå fra $\vec{x}$ til $\vec{x}'$.
\begin{align*}
 \vec{c}^T\vec{d} &= \vec{c}_B^T\vec{d}_B + \sum_{i \in N} c_i d_i 
 \\&= \vec{c}_B^T(- B^{-1}\sum_{i \in N} \vec{A}_i d_i) + \sum_{i \in N} c_i d_i  
 \\&= \sum_{i \in N} (- \vec{c}_B^T B^{-1} \vec{A}_i d_i) +  c_i d_i 
 \\&= \sum_{i \in N} ( c_i - \vec{c}_B^TB^{-1}\vec{A}_i ) d_i
\end{align*}
Det følger af Sætning \ref{stn:Deltac} at $\Delta c_i = c_i - \vec{c}_B^TB^{-1}\vec{A}_i $, hvorfor
\begin{align*}
\vec{c}^T\vec{d} = \sum_{i \in N} \Delta c_i d_i.
\end{align*}
Da $\vec{x}$ er en basisløsning må $x_i = 0$ for $i \in N$, og da $\vec{x}' \in P$ må $\vec{x}' \geq \vec{0}$. Dermed må $d_i = x_i' - x_i \geq 0$. Da $\Delta c_i$ var antaget at være ikke-negativ, medfører det at
\begin{align*}
\vec{c}^T\vec{d} = \vec{c}^T\vec{y}-\vec{c}^T\vec{x} \geq 0 \qquad \Rightarrow
\\ \vec{c}^T\vec{x} \leq \vec{c}^T\vec{x}'
\end{align*}
for ethvert $\vec{x}' \in P$. Da $\vec{x}'$ var vilkårligt valgt, må $\vec{x}$ være optimal.
\end{proof}

Det kan alt sammen opsummeres til Simplex metoden.

\begin{pro}[label=pro:simplex,style=ingental]{Procedure for Simplex metoden}
1. Vælg en basisløsning $\vec{x}$ med basismatrix $B$
2. Beregn $\Delta c_j$ for alle $j \notin I_B$. 
   Hvis $\Delta c_j\geq 0$ for alle $j \notin I_B$ 
   	   stop, $\vec{x}$ er optimal.
   Hvis $\Delta c_j < 0$ for en $j \notin I_B$
       vælg mindste $c_j$.
3. Find $\vec{d}_j$
   Hvis $d_i \geq 0 $ for alle $i \in I_B$ 
       stop, den optimaleværdi er $- \infty$.
   Hvis $d_i < 0 $ for mindst et $i \in I_B$ 
       vælg $B(l)$ så $\frac{x_{B(l)}}{d_{B(l)}}\leq \frac{x_i}{d_i} $ for ethvert $i \in I_B$
4. Find $\theta^*$
5. Find den nye basisvektor og basismatrix.
6. Gå til step 2.
\end{pro}

Til sidst vises, at Simplex metoden altid vil finde frem til den optimale løsning efter et endeligt antal iterationer.

\begin{stn}
Lad $P \neq \emptyset$, da stopper Simplex metoden efter et endeligt antal iterationer, enten ved at finde den optimale løsning eller at den optimale værdi er $- \infty$.
\end{stn}

\begin{proof}
Antag først, at Simplex metoden stopper efter step $2$ i Program \ref{pro:simplex}, da følger det af Sætning \ref{stn:optimalDeltaC}
at basisløsningen er optimal, og Simplex metoden har derfor fundet den optimale løsning.\\ 
Antag nu, at Simplex metoden stopper efter step $3$ i Program \ref{pro:simplex}, da følger det af beviset for Lemma \ref{lma:skalar}, at $P$ indeholder en linje, hvorfor det følger af Afsnit \ref{sec:eksistens},
at den optimale værdi er $-\infty$.\\ 
Da der kun er en endelig mængde basisløsninger ifølge Korollar \ref{kor:endeligbasis} må Simplex metoden altid stoppe, medmindre den cirkulere igennem de samme basisløsninger.
Da den $j$'te retningsvektor altid vælges så $\Delta c_j < 0$, må den samme basisløsning aldrig kunne blive besøgt mere end én gang, hvorfor der ikke kan opstå en ring, og Simplex metoden må derfor stoppe efter et endeligt antal iterationer.
\end{proof} 

\section{Implementering af Simplex Metoden}

Der findes forskellige måder at implementere Simplex metoden, det kan blandt gøres ved brug af Fuld tabel og Store-M metoden. Desuden kan den leksikografiske pivotregel forhindre, at Simplex metoden har et undeligt antal iterationer. 

\subsection{Fuld tabel}
Fuld tabel metoden er en metode til implementering af Simplex. Et standard minimeringsproblem $\vec{c}^T\vec{x}$, med bibetingelserne $A\vec{x} \geq \vec{b}$, kan ved brug af slack-variable omskrives til $A\vec{x}+\vec{s}=\vec{b}$. Givet en aktuel basis, $B$, kan ligheden også udtrykkes som  $B^{-1}\vec{b}=B^{-1}A\vec{x}$. Fuld tabel metoden bruger $m \times (n+1)$ matricen $B^{-1}[\vec{b} \ \mid \ A]$.\\

\begin{defn}[Simplex tabel]
Lad $z=\vec{c}^T\vec{x}$ være en objektfunktion med lineært uafhængige bibetingelser for et minimeringsproblem $A\vec{x} \geq \vec{b}$, så er en \textbf{Simplex tabel},\\
\begin{center}
\begin{tabular}{| c | c |}
  \hline
  $-\vec{c_B}\vec{x}_B$&$\Delta\vec{c}$ \\ \hline			
  $B^{-1}\vec{b}$ & $B^{-1}A$ \\ \hline
\end{tabular}
\end{center}
hvor $-\vec{c_B}\vec{x}_B$ er det negative af den aktuelle omkostning. 
\end{defn}


På mere udvidet form vil Simplex tabellen se således ud,
\begin{center}
\begin{tabular}{| r|r r r|}
  \hline	
  $-\vec{c_B}\vec{x}_B$&$\Delta c_1 $ & $\dots$ &$\Delta c_n$\\ \hline	
  $x_{B(1)}$ &	| & & |\\	
  $\vdots$  & $B^{-1}\vec{A}_1$ & $\hdots$ & $B^{-1}\vec{A}_n$\\
   $x_{B(m)}$ &	| & & |\\
   \hline
\end{tabular}
\end{center}
Søjlen $B^{-1}\vec{b}$ kaldes den nulte søjle, mens den øverste række kaldes den nulte række.

\begin{pro} [label=pro:simplex,numbers=none,xleftmargin=0em] {Fuld tabel metoden}
1. Start med en tabel til basismatricen $B$ og den tilhørende mulige basisløsning $\vec{x}$.
2. Undersøg den reducerede omkostning i den nulte række. Hvis de alle er ikke-negative, er den aktuelle løsning optimal. Elles vælges et $j$, hvor $\Delta c_j <0$.
3. Betragt vektoren $\vec{u}=B^{-1}\vec{A}_j$, som er den $j$'te søjle i tabellen. Hvis alle komponenterne i $\vec{u}$ er $\quad$ negative, er den optimale omkostning $-\infty$.
4. For hvert $i$, hvor $u_i$ er positiv, beregn forholdet $\frac{x_{B(i)}}{u_i}$. Lad $l$ være indeks for rækken, hvor forholdet er mindst. Søjlen $\vec{A}_{B(l)}$ forlader basen, og $\vec{A}_j$ indtræder i basen. 
5. Læg et multiplum af den $l$'te række til alle rækker, således at $u_l$ bliver $1$ og alle andre indgange i pivot søjlen bliver $0$
6. Gentag proceduren fra trin $2$. 
\end{pro}
For at starte Simplex metoden skal der bruges en mulig basisløsning. Hvis problemet er på formen $A\vec{x} \leq \vec{b}$, og $\vec{b} \geq \vec{0}$, vil der altid kunne findes en mulig basisløsning. Denne findes ved at omskrive problemet ved hjælp af slack-variable $A\vec{x} +\vec{s}= \vec{b}$. Vektoren givet ved $[\vec{x},\vec{s}]^T$, hvor $\vec{x}=\vec{0}$ og $\vec{s}=\vec{b}$, er en mulig basisløsning. 
\begin{eks}[Fuld Tabel Metoden]
Betragt det lineære progammeringsproblem
\begin{center}
\begin{tabular}{ l c c  c  r }
Minimer &$-10x_1$&$-12x_2$ & $-12x_3$&\\
I forhold til: &$x_1$&+$2x_2 $&+$2x_3$ & $\leq 20$\\
&$2x_1$& $+x_2$& $+2x_3$ & $\leq 20$\\
&$2x_1$&$+2x_2$&$+x_3$&$\leq 20$\\
$x_1,x_2,x_3\geq 0$.
\end{tabular}
\end{center}

Nu indførers slack-variable, så ulighederne bliver til ligheder. 
\begin{center}
\begin{tabular}{ l c c  c c c c r }
Minimer &$-10x_1$&$-12x_2$ & $-12x_3$&&&\\
I forhold til: &$x_1$&+$2x_2 $&+$2x_3$ &$+x_4$&& &$=20$\\
&$2x_1$& $+x_2$& $+2x_3$ & & $+x_5$ &&$=20$\\
&$2x_1$&$+2x_2$&$+x_3$&&&$+x_6$&$=20$\\
$x_1,x_2,x_3,x_4,x_5,x_6\geq 0$
\end{tabular}
\end{center}

Trin 1\\
Nu er $\vec{x}^T=[0,0,0,20,20,20]$ en mulig basisløsning, hvor $B(1)=4,B(2)=5$ og $B(3)=6$. Den tilhørende basismatrix er identitesmatricen, $I_3$. 
\begin{center}
\begin{tabular}{|r| r|r r r r r r|}
  \hline	
  &$0$&$-10$ &$-10$&$-10$&$0$&$0$&$0$\\ \hline	
  $x_4=$&$20$&$1$&$2$&$2$&$1$&$0$&$0$\\	
  $x_5=$&$20$&$2$&$1$&$2$&$0$&$1$&$0$\\
  $x_6=$&$20$&$2$&$2$&$1$&$0$&$0$&$1$\\
   \hline
\end{tabular}
\end{center}

Trin 2\\
I den nulte række findes $3$ negative værdier. Den reducerede omkostning for $x_1$ er negativ. Den indtræder derfor i basen. \\
\\
Trin 3\\
$\vec{u}^T=[1,2,2]$. Det ses, at alle komponenterne er positive. \\
\\
Trin 4\\
For at bestemme hvilken række, der skal være pivotrække udregnes forholdet $\frac{x_{B(i)}}{u_i}$. 
\begin{align*}
\frac{x_{B(1)}}{u_1}=\frac{20}{1}=20\\
\frac{x_{B(2)}}{u_2}=\frac{20}{2}=10\\
\frac{x_{B(3)}}{u_3}=\frac{20}{2}=10\\
\end{align*}
Det ses, at forholdet for $i=2$ og $i=3$ er det samme. Da vælges $l$ til at være række $2$. \\
\\
Trin 5\\
Den nye tabel er givet ved:
\begin{center}
\begin{tabular}{|r| r|r r r r r r|}
  \hline	
  &$100$&$0$ &$-7$&$-2$&$0$&$5$&$0$\\ \hline	
  $x_4=$&$10$&$0$&$1.5$&$1$&$1$&$-0.5$&$0$\\	
  $x_1=$&$10$&$1$&$0.5$&$1$&$0$&$0.5$&$0$\\
  $x_6=$&$0$&$0$&$1$&$-1$&$0$&$-1$&$1$\\
   \hline
\end{tabular}
\end{center}
I denne tabel er den reducerede omkostning for $x_2$ og $x_3$ negativ. Derfor gennemgås procedureren igen fra Trin 2.\\
$x_3$ vælges til at indtræde i basen. Efter procedureren er gennemgået er tabellen givet ved. 
\begin{center}
\begin{tabular}{|r| r|r r r r r r|}
  \hline	
  &$120$&$0$ &$-4$&$0$&$2$&$4$&$0$\\ \hline	
  $x_3=$&$10$&$0$&$1.5$&$1$&$1$&$-0.5$&$0$\\	
  $x_1=$&$0$&$1$&$-1$&$0$&$-1$&$1$&$0$\\
  $x_6=$&$10$&$0$&$2.5$&$0$&$1$&$-1.5$&$1$\\
   \hline
\end{tabular}
\end{center}
Den reducerede omkostning for $x_2$ er stadig negativ. Derfor gennemgås procedureren igen fra Trin 2. Når $x_2$ indtræder i basen, er tabellen givet ved. 
\begin{center}
\begin{tabular}{|r| r|r r r r r r|}
  \hline	
  &$136$&$0$ &$0$&$0$&$3.6$&$1.6$&$1.6$\\ \hline	
  $x_3=$&$4$&$0$&$0$&$1$&$0.4$&$0.4$&$-0.6$\\	
  $x_1=$&$4$&$1$&$0$&$0$&$-0.6$&$0.4$&$0.4$\\
  $x_2=$&$4$&$0$&$1$&$0$&$0.4$&$-0.6$&$0.4$\\
   \hline
\end{tabular}
\end{center}
Der er nu ingen negative indgange i den nulte række. Derfor er den nuværende basisløsning, $\vec{x}=(4,4,4,0,0,0)$, den optimale løsning. Omkostningen for objektfunktionen er så $-136$.
\end{eks}


\subsection{Leksikografisk pivotregel}
Den leksikografiske pivotregel er udviklet ud fra observationer af Simplex metodens opførsel. Formålet med metoden er at sikre, at Simplex metoden slutter efter et endeligt antal iterationer. 
Den leksikografiske pivotregel er let at implementere i Fuld tabel metoden. 
\begin{defn}[Leksikografisk mindre]
En vektor $\vec{u} \in \mathds{R}^n$ siges at være \textbf{leksikografisk mindre} end en anden vektor $\vec{v} \in \mathds{R}^n$, hvis $\vec{u} \neq \vec{v}$ og den første ikke-nul komponent af $\vec{u}-\vec{v}$ er negativ 
\begin{align*}
\vec{u} \overset{L}{<} \vec{v}.
\end{align*}
\end{defn}
\begin{eks}
Tag udgangspunkt i de to vektorer
\begin{align*}
\vec{u}^T=[0,5,8]\quad 
\text{og}
\quad \vec{v}^T=[1,3,1]. 
\end{align*}
Da gælder det, at $\vec{u} \overset{L}{<} \vec{v}$, da den første ikke-nul komponent af $\vec{u}-\vec{v}$ er $-1$.
\end{eks}

Fremgangsmåden for den leksikografiske pivotregel er som følger.
  
\begin{pro}{Leksikografisk pivotregel}
Vælg en vilkårlig indgangssøjle $\vec{A}_j$. Det skal gælde, at $\Delta c_j$ er negativ. Lad $\vec{u}=B^{-1}\vec{A}_j$ være den $j$'te søjle i Simplex tabellen.
For hvert $i$, $u_i>0$, divider den $i$'te række med $u_i$ og vælg den leksikografisk mindste række. 
Hvis $l$ er leksikografisk mindst, så udgår den $l$'te basisvariabel, $x_{B(l)}$, af basen. 
\end{pro}

\begin{eks}
Betragt den følgende matrix, hvor den nulte række er undladt. Antag, at pivotsøjlen er den tredje søjle. 

\begin{center}
\begin{tabular}{|l|llll|}
\hline
8  & 0 & 1 & 2  &  \\
7  & 7 & 5 & -2 &  \\
12 & 0 & 4 & 3  &  \\
\hline
\end{tabular}
\end{center}
For at bestemme den variabel, der skal forlade basen, bestemmes nu hvilken række, der er leksikografisk mindst, ved at dividere den første og den tredje række med henholdsvis $u_1$ og $u_3$.
\begin{center}
\begin{tabular}{|l|llll|}
\hline
4  & 0 & $\frac{1}{2}$ & 1  &  \\
*  & * & * & * &  \\
4 & 0 & $\frac{4}{3}$ & 1  &  \\
\hline
\end{tabular}
\end{center}
Her er den den første række leksikografisk mindst, da $\frac{1}{2}<\frac{4}{3}$.  Altså er første række pivotrækken. Og $x_{B(1)}$ forlader basen. 
\end{eks}

Bemærk, at den leksikografiske pivotregel altid fører til et entydigt valg af variabel. Hvis dette ikke var tilfældet, måtte det gælde, at to rækker i tabellen er lineært afhængige. Da ville rangen af $B^{-1}A$ være mindre end $m$. Det samme ville være gældende for $A$. Det er i modstrid med antagelsen om, at $A$ indeholder lineært uafhængige rækker. 

\begin{defn}[Leksikografisk positiv]
En vektor $\vec{u} \neq \vec{0}$ kaldes \textbf{leksikografisk positiv} hvis den første ikke-nul komponent er positiv. \citep{lexipositiv} 
\end{defn}

 
\begin{stn}
Antag, at Simplex tabellen fra algoritmens start kun indeholder leksikografisk positive rækker bortset fra den nulte række. Følges den leksikografiske pivotregel, så vil 
\begin{enumerate}[label=(\alph*)]
\item enhver række i Simplex tabellen, foruden den nulte række, forblive leksikografisk positiv gennem algoritmen. 
\item den nulte række vokse skarpt for hver iteration. 
\item Simplex metoden slutte efter et endeligt antal iterationer. 
\end{enumerate}
\label{stn:lexi}
\end{stn}

\begin{proof}
(a) Antag, at alle rækker, foruden den nulte række, er leksikografisk positive, ved starten af en iteration. Antag, at $x_j$ indgår i basen og pivotrækken er den $l$'te række. 
Så følger det af den leksikografiske pivotregel, at $u_l>0$ og
\begin{align}
\frac{(l'te \quad række)}{u_l} \overset{L}{<} \frac{(i'te \quad række)}{u_i}, \quad \quad \text{hvis} \quad  i \neq l \quad \text{og} \quad u_i>0.
\label{5_2}
\end{align}
For at bestemme den nye tabel, bliver den $l$'te række divideret med et positiv tal $u_l$, og forbliver derfor leksikografisk positiv. 

Betragt nu den $i$'te række og antag, at $u_i<0$. For at få den $(i,j)$'te indgang til at være lig $0$, skal en positivt multiplikation af den $l$'te række lægges til. Da både den $i$'te og den $l$'te række var leksikografisk positive før, vil de ved denne addition forblive leksikografisk positive.

Betragt nu tilfældet hvor $u_i>0$ og $i \neq l$. Da er den nye $i$'te række givet ved. 
\begin{align*}
(\text{ny $i$'te række)}=\text{(gammel $i$'te række)}-\frac{u_i}{u_l}\text{(gammel $l$'te række)}
\end{align*}  
Fordi den leksikografiske ulighed i Ligning \ref{5_2} gælder for de gamle rækker, må den nye $i$'te række også være leksikografisk positiv. \\
(b) Ved starten af en iteration er den reducerede omkostning i pivotsøjlen negativ. For at få den reducerede omkostning til at være lig $0$, skal en positiv multiplikation af pivotrækken lægges til. 
Da de resterende rækker er leksikografisk positive, vil den nulte række vokse leksikografisk. 

(c) Da den nulte række vokser leksikografisk ved hver iteration, vil den aldrig komme tilbage til en tidligere værdi. Den nulte række afhænger af den aktuelle basis, derfor vil den samme basis aldrig blive gentaget. Simplex metoden må da slutte efter et endeligt antal iterationer. 
\end{proof}


\section{Store M-metode}

For at kunne bruge simplex metoden, skal der først findes en mulig basis løsning. 
Hvis der er givet et problem på formen $Ax \leq b$, hvor $b \geq 0$ er det relativt simpelt. 
Her kan introduceres ikke-negative slack-variable $s$ og uligheden kan omskrives til $Ax+s=b$. 
Vektoren $(x,s)$ hvor $x=0$ og $s=b$ er en mulig basis løsning, og den korresponderende basismatrix er dens identitet. 
<<<<<<< HEAD
Generelt er det dog ikke let at finde en mulig basis løsning, og det kræver en løsning til et hjælpende lineært programmeringsproblem. \\
=======
Generelt er det dog ikke let at finde en mulig basis løsning, og det kræver at løsning til et hjælpende lineært programmeringsproblem. \\
>>>>>>> master

Store-M metoden bygger på den to-fase simplex metode, som er en komplet algoritme til at løse lineære programmeringsproblemer på standardform. 
Algoritmen er bygget op i to faser, hvor første fase går ud på .... \\
%?? skal det med

Idéen bag store-M metoden er at introducere en kostfunktion på formen
\begin{align*}
\sum\limits_{j=1}^n c_jx_j + M \sum\limits_{i=1}^m y_i,
\end{align*}
hvor $M$ er en stor positiv konstant, og $y_i$ er samme variable som i den første fase af to-fase simplex. 
Hvis det originale problem har en mulig løsning, og dens optimale kost er endelig, vil de artificial (kunstige?) variable gå mod $0$, når $M$ er tilstrækkelig stor. 
Det betyder, at den originale kostfunktion nu skal minimeres. 
Når den reducerede kost er en funktion af $M$, vil $M$ altid blive behandlet som værende af større værdi, når der skal vurderes, om en reduceret kost er negativ. \\

\begin{eks}
Et lineært programmeringsproblem er givet.

	\begin{center}
	\begin{tabular}{l >{$}r<{$}	>{$}r<{$} >{$}l<{$} >{$}l<{$} r}
	Minimer 		& 	x_1	 & + \ \ x_2 & + \ \ x_3 \\
	med hensyn til 	&  	x_1	 & +   2 x_2 & +   3 x_3 &  	 & = 3 \\
					&  -x_1	 & +   2 x_2 & +   6 x_3 & 		 & = 2 \\
					&  \ \ 	 & \ \ 4 x_2 & +   9 x_3 & 		 & = 5 \\
					&  \ \ 	 & \ \   	 & \ \ 3 x_3 & + x_4 & = 1 \\
	og $x_1, \dots, x_4 \geq 0$.
	\end{tabular}
	\end{center}

Der gives nu følgende problem til hjælp, hvor den unødvendige kunstige variabel $x_8$ er undladt.

	\begin{center}
	\begin{tabular}{l >{$}r<{$}	>{$}r<{$} >{$}l<{$} >{$}r<{$} >{$}r<{$} >{$}r<{$} >{$}r<{$} r}
	Minimer 		&  	x_1	 & + \ \ x_2 & + \ \ x_3 &       & + Mx_5    & + Mx_6    & + Mx_7 \\
	med hensyn til 	&  	x_1	 & +   2 x_2 & +   3 x_3 &       & + \ \ x_5 &           &        & = 3 \\
					&  -x_1	 & +   2 x_2 & +   6 x_3 &       &           & + \ \ x_6 &        & = 2 \\
					&        &     4 x_2 & +   9 x_3 &       &           &           & + \ \ x_7 & = 5 \\
					&   	 &           &     3 x_3 & + x_4 &           &           &       & = 1 \\
	og $x_1, \dots, x_7 \geq 0$.
	\end{tabular}
	\end{center}

%eks ikke færdigt. + tabellerne skal fikses
\end{eks}

hello world!
