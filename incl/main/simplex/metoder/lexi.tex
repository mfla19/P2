\section{Leksikografisk pivot regl}
Den leksikografiske pivot regel er udviklet ud fra observstioner af simplex metodens opførsel. Formålet med metoden er at sikre at Simplex metoden slutter efter et endeligt antal iretationer. Den leksikografiske pivot regel er let at implementere i fuld tabel metoden. 
\begin{defn}[Leksikografisk mindre]
En vektor $\vec{u} \in \mathds{R}^n$ siges at være \textbf{leksikografisk mindre} end en anden vektor $\vec{v} \in \mathds{R}^n$ hvis $\vec{u} \neq \vec{v}$ og den første ikke nul komponent af $\vec{u}-\vec{v}$ er negativ 
\begin{align*}
\vec{u} \overset{L}{<} \vec{v}
\end{align*}
\end{defn}
\begin{eks}
Tag udgangspunkt i de to vektorer: 
\begin{align*}
\vec{u}=
\begin{bmatrix}
0\\
5\\
8\\
2\\
\end{bmatrix}
\text{og}
\vec{v}= 
\begin{bmatrix}
1\\
3\\
1\\
2\\
\end{bmatrix}
\end{align*}
Da gælder det at $\vec{u} \overset{L}{<} \vec{v}$, da den første ikke nul komponent af $\vec{u}-\vec{v}$ er $-1$
\end{eks}

Fremgangsmåden for den leksikografiske pivot regel er som følger.
  
\begin{pro}{Lexi pivot regl}
Vælg en vilkårlig indgangssøjle $\vec{A}_j$, det skal gælde at $\Delta c_j$ er negativ. Lad $\vec{u}=B^{-1}\vec{A}_j$ være den $j$'te søjle i simplex-tabellen.
For hvert $i$, $u_i>0$, divider med den $i$'te række med $u_i$, og vælg den leksikografisk mindste række. Hvis $l$ er leksikografisk mindst, så udgår den $l$'te basis variabel,$x_{B(l)},$ af basen. 
\end{pro}

Bemærk at den leksikografiske pivot regle altid fører til et entydigt valg af variabel. Hvis dette ikke var tilfældet måtte det gælde at to rækker i tabllen er proportionelle. Da ville rangen af $B^{-1}A$ være mindre end $m$. Det samme ville være gældende for $A$. Det er i modstrid med antagelsen om at $A$ indeholder lineært uafhængige rækker. 

\begin{defn}[Lexi positiv]
En vektor $\vec{u} \neq \vec{0}$ kaldes \textbf{leksikografisk positiv} hvis den første ikke nul komponent er positiv. 
\end{defn}

 
\begin{stn}
Antag at simplex tabellen fra algoritmens start kun indeholder leksikografisk positive rækker, bortset fra den nulte række. Følges den leksikografiske pivot regel så: 
\begin{enumerate}[label=(\alph*)]
\item Enhver række i simplex tabellen, foruden den nulte række, forbliver leksikografisk positiv gennem algoritmen. 
\item Den nulte række vokser skarpt for hver iteration. 
\item Simplex metoden slutter efter et endeligt antal iterationer. 
\end{enumerate}
\label{stn:lexi}
\end{stn}

\begin{proof}
(a) Antag at alle rækker, foruden den nulte række, er leksikografisk positive, ved starten af en iteration. Antag at $x_j$ indgår i basen, og pivot rækken er den $l$'te række. Så følger det af den leksikografiske pivot regel at $u_l>0$ og
\begin{align}
\frac{(l'te \quad række)}{u_l} \overset{L}{<} \frac{(i'te \quad række)}{u_i}, \quad \quad \text{hvis} \quad  i \neq l \quad \text{og} \quad u_i>0
\end{align}
For at bestemme den nye tabel bliver den $l$'te række divideret med et positiv tal $u_l$, og forbliver derfor Lexi positiv. 

Betragt nu den $i$'te række og antag at $u_i<0$. For at få den $(i,j)$¨'te indgang til at være lig $0$ skal et positivt multiplikation af den $l$'te række lægges til. Da både den $i$'te og den $l$'te række var leksikografisk positive før, vil de ved denne addition forblive leksikografisk positive.

Betragt nu tilfældet hvor $u_i>0$ og $i \neq l$. Da er den nye $i$'te række givet ved. 
\begin{align*}
(\text{ny $i$'te række)}=\text{(gammel $i$'te række)}-\frac{u_i}{u_l}\text{(gammel $l$'te række)}
\end{align*}  
Fordi den leksikografiske ulighed i Ligning (8.2) gælder for de gamle rækker, må den nye $i$'te række være leksikografisk positiv. 

(b)Ved starten af en iteration er den reducerede omkostning i pivot søjlen negativ. For at får den reducerede omkostning til at være lig $0$, skal en positiv muliplikation af pivotrækken lægges til. Da de resterende rækker er leksikografisk positive, vil den nulte række vokse lexikografisk. 


(c) Da den nulte række vokser leksikografisk ved hver iteration, vil den aldrig komme tilbage til en tidligere værdi. Den nulte række afhænger af den aktuelle basis. Derfor vil en den samme basis aldrig blive gentaget. Simplex metoden må da slutte efter et endeligt antal iterationer. 
\end{proof}

Det er det nødvendigt at simplex tabellen indeholder Lexi positive rækker. 