\subsection{Fuld tabel metoden}
Fuld tabel metoden, er en metode til Simplex, hvor der ved hjælp af tilføjet slat værdier, for at lave uligheder om til ligheder, laves elementære rækkeoperationer på både bibetingelserne og objektfunktionen.\\
%Indtil $\vec{x}$ får nogle værdier, vil resultatet af objektfunktionen være $0$, derfor når der bliver lavet rækkeaberrationer på objektfunktionen, så vil resultatet også ændre sig. 
\\
%procedure for maksimeringsproblem
\begin{defn}[Simplex Tabel]
Lad $z=\vec{c}^T\vec{x}$ være en objektfunktion, med lineæret uafhængige bibetingelser for et maksimeringsproblem $	\vec{a}_i^T\vec{x} \ \leq \  b_i, \text{for} i \in \{1,2,\cdots, m\}$ så er en \textbf{Simplex Tabel}\\
\begin{center}
\begin{tabular}{| l | c | r |}
  \hline
  $\vec{x}^T$&$\vec{S}^T$& \\ \hline			
  $\vec{a_i}^T$ & $I_m$ & $\vec{b}$ \\ \hline
  $\vec{c}^T$ & $\vec{0}^T$ & $z$ \\
  \hline  
\end{tabular}
\end{center}
\end{defn}
På mere udvidet form, vil Simplex Tabellen se således ud.
\begin{center}
\begin{tabular}{| l  c  c | c  c  c |r |}
  \hline	
  $x_1$&$\dots$&$x_n$&$S_1$&$dots$&$S_m$&\\ \hline		
  $a_{1 1}$ & $\dots$ & $a_{1 n}$ & $1$ & $\dots$ & $0$ & $b_1$\\
  $\vdots$ & $\ddots$ & $\vdots$ & $\vdots$ & $\ddots$ & $\vdots$ & $\vdots$\\
  $a_{m 1}$ & $\dots$ & $a_{m n}$ & $0$ & $\dots$ & $1$ & $b_m$\\ \hline
  $c_1$ & $\dots$ & $c_n$ & $0$ & $\dots$ & $0$ & $z$ \\ \hline  
\end{tabular}
\end{center}
Læg mærke til, at slat værdierne i forvejen danner Pivot-indgange i for af identitetsmatrix, fra tidligere vides det, at hvis der betragtes de søjler, hvor der dannes pivot, så vil alle andre søjlers værder blive $0$. Dette kan anvendes, til at danne Pivot i søjler, hvor $x_1\dots x_n$ får værdier.\\
Når der bliver lavet elementære rækkeoperationer, for at danne Pivot $\vec{A_j}$, så skal der også elementære rækkeoperationer på
$\left[\vec{c}^T | z\right]$ og derfor, ændres maksimummet. Dette kan checkes, ved at løse objektfunktionen, med den bestemte $\vec{x}$.
\begin{pro} [label=pro:simplex,numbers=none,xleftmargin=0em] {Procedure for Fuld tabel metoden}
1. Tilføj en slat værdi $S_1\dots S_m$, til hver lineære bibetingelse og lav uligheder om til ligheder.
2. Opsæt Simplex tabel.
3. Vælg den søjle $\vec{A_j}$, hvor $\vec{c}^T$ har størst ikke negativ værdi.
4. I den j'te søjle, find den værdi $a_{i j}$ der har størst ratio med $b_i$.
5. Vælg denne $a_{i j}$ til at være pivot indganen for søjlen, og brug elementære rækkeoperationer, til at lave $a_{i j}=1$ og resten af $\vec{A_j}=0$.(Dette inkludere $c_j$).
6. Nu vil der i cellen, hvor der tidligere stod $z$ nu stå $z-d$, for $d\in \mathds{R}$, løs $z-d=0$ for $z$ dette er nu den største værdi for $z$.
Gentag fra trin. 3 indtil der ikke er flere ikke negative værdier i $\vec{c}^T$.
\end{pro}

\begin{eks}[Fuld Tabel Metoden]
\begin{center}
\begin{tabular}{ l  c  c  r }
Maksimer $z$ =&$x_1$ & $+2x_2$&\\
I forhold til: &$x_1$ & & $\leq 2$\\
& &$x_2$ & $\leq 2$\\
&$x_1$&$+x_2$&$\leq 3$\\
$x_1,x_2\geq 0$
\end{tabular}
\end{center}
Trin. 1\\
\begin{center}
\begin{tabular}{ l  c  c  c  c  c  r }
Maksimer $z$ =&$x_1$ & $+2x_2$&&&&\\
I forhold til: &$x_1$&&$S_1$&&& $\leq 2$\\
&&$x_2$ &&$S_2$ && $\leq 2$\\
&$x_1$&$+x_2$&&&$S_3$&$\leq 3$\\
$x_1,x_2\geq 0$
\end{tabular}
\end{center}
Trin. 2\\
\begin{center}
\begin{tabular}{| l  c  | c  c  c | r |}
\hline
$1$&$0$&$1$&$0$&$0$&$2$\\
$0$&$1$&$0$&$1$&$0$&$2$\\
$1$&$1$&$0$&$0$&$1$&$3$\\ \hline
$1$&$2$&$0$&$0$&$0$&$z$\\
\hline
\end{tabular}
\end{center}

Bemærk at: $x_1,x_2=0,S_1=2, S_2=2, S_3=3$ er en løsning, men hvor $z=0$.\\
Trin. 3\\
Søjle $\vec{a_2}$ har størst ikke negativ værdig.\\
Trin. 4\\
\begin{center}
$\frac{a_{1 2}}{b_1}=\frac{0}{2}$\\
$\frac{a_{2 2}}{b_2}=\frac{1}{2}$\\
$\frac{a_{3 2}}{b_3}=\frac{1}{3}$\\
\end{center}
Nu har $a_{2 2}$ størst ratio og vil derfor være pivot indgangen for $\vec{a_2}$.\\
Trin. 5
Elementære rækkeoperationer udføres.\\
\begin{center}
\begin{tabular}{| l  c  | c  c  c | r |}
\hline
$1$&$0$&$1$&$0$&$0$&$2$\\
$0$&$1$&$0$&$1$&$0$&$2$\\
$1$&$0$&$0$&$-1$&$1$&$1$\\ \hline
$1$&$0$&$0$&$0$&$0$&$z-4$\\
\hline
\end{tabular}
\end{center}
Trin. 6\\ 
En løsning er $x_2=2, S_1=2, x_1, S_2, S_3=0$, hvor $z=4$.\\
Trin. 3\\
Søjle $\vec{a_1}$ har størst ikke negativ værdig.\\
Trin. 4\\
\begin{center}
$\frac{a_{1 1}}{b_1}=\frac{1}{2}$\\
$\frac{a_{2 1}}{b_2}=\frac{0}{2}$\\
$\frac{a_{3 1}}{b_3}=\frac{1}{1}$\\
\end{center}
$a_{3 1}$ har størst ratio \\
Trin. 5 Elementære rækkeoperationer udføres.\\
\begin{center}
\begin{tabular}{| l  c  | c  c  c | r |}
\hline
$0$&$0$&$1$&$1$&$-1$&$1$\\
$0$&$1$&$0$&$1$&$0$&$2$\\
$1$&$0$&$0$&$-1$&$1$&$1$\\ \hline
$0$&$0$&$0$&$-1$&$-1$&$z-5$\\
\hline
\end{tabular}\\
\end{center}
Trin. 6 \\
En løsning er $x_1=1, x_2=2, S_1, S_2, S_3=0$, hvor $z=5$.\\
Der er nu ikke flere ikke negative værdier i $\vec{c}T$ og den maksimale værdi for $z=5$.
\end{eks}

%noter til selv

%Du har ikke skrevet at alle andre søjler bliver = 0 
%Hele afsnittet skal måske skrives om til et minimeringsproblem




%$B^{-1}\left[\vec{b}\mid A\right]$
