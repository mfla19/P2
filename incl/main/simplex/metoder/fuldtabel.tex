\subsection{Fuld tabel}
Fuld tabel metoden, er en metode til Simplex. For et standard minimeringsproblem $\vec{c}^T\vec{x}$, med bibetingelserne $A\vec{x} \geq \vec{b}$, kan ved brug af slack-variable omskrives til $A\vec{x}+\vec{s}=\vec{b}$. Givet en aktuel basis, $B$, kan ligheden også udtrykkes som  $B^{-1}\vec{b}=B^{-1}A\vec{x}$. Fuld tabel metoden bruger $m \times (n+1)$ matricen $B^{-1}[\vec{b}|A]$.\\

\begin{defn}[Simplex Tabel]
Lad $z=\vec{c}^T\vec{x}$ være en objektfunktion, med lineæret uafhængige bibetingelser for et minimeringsproblem $A\vec{x} \geq \vec{b}$ så er en \textbf{Simplex Tabel}\\
\begin{center}
\begin{tabular}{| c | c |}
  \hline
  $-\vec{c_B}\vec{x}_B$&$\Delta\vec{c}$ \\ \hline			
  $B^{-1}\vec{b}$ & $B^{-1}A$ \\ \hline
\end{tabular}
\end{center}
Hvor $-\vec{c_B}\vec{x}_B$ er det negative af den aktuelle omkostning. 
\end{defn}


På mere udvidet form, vil Simplex Tabellen se således ud.
\begin{center}
\begin{tabular}{| r|r r r|}
  \hline	
  $-\vec{c_B}\vec{x}_B$&$\Delta c_1 $ & $\dots$ &$\Delta c_n$\\ \hline	
  $x_{B(1)}$ &	| & & |\\	
  $\vdots$  & $B^{-1}\vec{A}_1$ & $\hdots$ & $B^{-1}\vec{A}_n$\\
   $x_{B(m)}$ &	| & & |\\
   \hline
\end{tabular}
\end{center}
Søjlen $B^{-1}\vec{b}$ kaldes den nulte søjle, mens den øverste række kaldes en nulte række.

\begin{pro} [label=pro:simplex,numbers=none,xleftmargin=0em] {Procedure for Fuld tabel metoden}
1. Start med en tabel til basismatricen $B$ og den tilhørende mulige basisløsning $\vec{x}$
2. Undersøg den reducerede omkostning i den nulte række. Hvis de alle er ikke negative, er den aktuelle løsning optimal. Elles vælges et $j$, hvor $\Delta c_j <0$
3. Betragt vektoren $\vec{u}=B^{-1}\vec{A}_j$, som er den $j$'te søjle i tabellen. Hvis alle komponenterne i $\vec{u}$ er negative, er den optimale omkostning $-\infty$
4. For hvert $i$, hvor $u_i$ er positiv, beregn forholdet $x_{B(i)}/u_i$. Lad $l$ være indeks for rækken, hvor forholdet er mindst. Søjlen $\vec{A}_{B(l)}$ forlader bases, og $\vec{A}_j$ indgår i basen. 
5.Læg et multiplum af den $l$'te række til alle rækker, således at $u_l$ bliver $1$ og alle andre indgange i pivot søjlen bliver $0$
6.Gå til trin 2. 
\end{pro}
For at starte simplex metoden skal der bruges en mulig basisløsning. Hvis problemet er på formen $A\vec{x} \leq \vec{b}$, og $\vec{b} \geq 0$, vil der altid kunne findes en mulig basisløsning. Denne findes ved om omskrive problemet ved hjælp af slack varialbe $A\vec{x} +\vec{s}= \vec{b}$. Vektoren givet ved $[\vec{x},\vec{s}]$, hvor $\vec{x}=\vec{0}$ og $\vec{s}=\vec{b}$ er en mulig basisløsning. 
\begin{eks}[Fuld Tabel Metoden]
Betragt det lineære progammeringsproblem. 
\begin{center}
\begin{tabular}{ l c c  c  r }
Minimer &$-10x_1$&$-12x_2$ & $-12x_3$&\\
I forhold til: &$x_1$&+$2x_2 $&+$2x_3$ & $\leq 20$\\
&$2x_1$& $+x_2$& $2x_3$ & $\leq 20$\\
&$2x_1$&$+2x_2$&$+x_3$&$\leq 20$\\
$x_1,x_2,x_3\geq 0$
\end{tabular}
\end{center}

Nu indførers slack-variable, så ulighederne bliver til ligheder. 
\begin{center}
\begin{tabular}{ l c c  c c c c r }
Minimer &$-10x_1$&$-12x_2$ & $-12x_3$&&&\\
I forhold til: &$x_1$&+$2x_2 $&+$2x_3$ &$+x_4$&& &$=20$\\
&$2x_1$& $+x_2$& $2x_3$ & & $+x_5$ &&$=20$\\
&$2x_1$&$+2x_2$&$+x_3$&&&$+x_6$&$=20$\\
$x_1,x_2,x_3,x_4,x_5,x_6\geq 0$
\end{tabular}
\end{center}

Trin. 1\\
Nu er $\vec{x}^T=[0,0,0,20,20,20]$ en mulig basisløsning, hvor $B(1)=4,B(2)=5,B(3)=6$. Og den tilhørende basis matrix er identitesmatricen, $I_3$. 
\begin{center}
\begin{tabular}{r| r|r r r r r r|}
  \hline	
  &$0$&$-10$ &$-10$&$-10$&$0$&$0$&$0$\\ \hline	
  $x_4=$&$20$&$1$&$2$&$2$&$1$&$0$&$0$\\	
  $x_5=$&$20$&$2$&$1$&$2$&$0$&$1$&$0$\\
  $x_6=$&$20$&$2$&$2$&$1$&$0$&$0$&$1$\\
   \hline
\end{tabular}
\end{center}

Trin 2\\
I den nulte række findes 3 negative værdier. Den reducerede omkostning for $x_1$ er negativ. Den indgår derfor i basen. \\
\\
Trin 3\\
$\vec{u}^T=[1,2,2]$. Det ses at alle komponenteren er positive. \\
\\
Trin 4\\
For at bestemme hvilken række der skal være pivot række udrenges forholdet $x_{B(i)}/u_i$. 
\begin{align*}
x_{B(1)}/u_1=\frac{20}{1}=20\\
x_{B(2)}/u_2=\frac{20}{2}=10\\
x_{B(3)}/u_3=\frac{20}{2}=10\\
\end{align*}
Det ses at forholdet for $i=2$ og $i=3$ er det samme. $l$ vælges til at være række $2$. \\
\\
Trin 5\\
Den nye tabel er givet ved:
\begin{center}
\begin{tabular}{r| r|r r r r r r|}
  \hline	
  &$100$&$0$ &$-7$&$-2$&$0$&$5$&$0$\\ \hline	
  $x_4=$&$10$&$0$&$1.5$&$1$&$1$&$-0.5$&$0$\\	
  $x_1=$&$10$&$1$&$0.5$&$1$&$0$&$0.5$&$0$\\
  $x_6=$&$0$&$0$&$1$&$-1$&$0$&$-1$&$1$\\
   \hline
\end{tabular}
\end{center}
I denne tabel er den reducerede omkostning for $x_2$ og $x_3$ negativ. Derfor gennemgås procedureren igen, fra trin 2.\\
$x_3$ vælges til at indgå i basen. Efter procedureren er gennemgået er tabellen givet ved. 
\begin{center}
\begin{tabular}{r| r|r r r r r r|}
  \hline	
  &$120$&$0$ &$-4$&$0$&$2$&$4$&$0$\\ \hline	
  $x_3=$&$10$&$0$&$1.5$&$1$&$1$&$-0.5$&$0$\\	
  $x_1=$&$0$&$1$&$-1$&$0$&$-1$&$1$&$0$\\
  $x_6=$&$10$&$0$&$2.5$&$0$&$1$&$-1.5$&$1$\\
   \hline
\end{tabular}
\end{center}
Den reducerede omkostning for $x_2$ er stadig negativ. Derfor gennemgåes procedureren igen fra trin 2. Når $x_2$ indgår i basen er tabellen givet ved. 
\begin{center}
\begin{tabular}{r| r|r r r r r r|}
  \hline	
  &$136$&$0$ &$0$&$0$&$3.6$&$1.6$&$1.6$\\ \hline	
  $x_3=$&$4$&$0$&$0$&$1$&$0.4$&$0.4$&$-0.6$\\	
  $x_1=$&$4$&$1$&$0$&$0$&$-0.6$&$0.4$&$0.4$\\
  $x_2=$&$4$&$0$&$1$&$0$&$0.4$&$-0.6$&$0.4$\\
   \hline
\end{tabular}
\end{center}
Der er nu ingen negative indgange i den nulte række. Derfor er den nuværende basisløsning, $\vec{x}=(4,4,4,0,0,0)$, den optimale løsning. 
\end{eks}

