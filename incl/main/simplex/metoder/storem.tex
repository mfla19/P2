\subsection{Store M-metoden}

Den sidste metode der introduceres er Store M-metoden. 
Hvis problemet der skal løses ikke er på standartform, kan det være svært at finde mulig basisløsning, til at sætte algoritmen igang. Da er Store M-metoden nyttig.
Idéen bag Store M-metoden er at introducere en hjælpefunktion på formen
\begin{align*}
\sum\limits_{j=1}^n c_jx_j + M \sum\limits_{i=1}^m y_i,
\end{align*}
hvor $M$ er en stor positiv konstant, og $y_i$ er kunstige variable. 
Hvis det originale problem har en mulig løsning, og dens optimale kost er endelig, vil de kunstige variable gå mod $0$, når $M$ er tilstrækkelig stor. 
Det betyder, at den originale kostfunktion nu skal minimeres. 
Når den reducerede kost er en funktion af $M$, vil $M$ altid blive behandlet som værende af større værdi, når der skal vurderes, om en reduceret kost er negativ. \\

\begin{eks}
Et lineært programmeringsproblem er givet.

\begin{center}
\begin{tabular}{l >{$}r<{$}	>{$}r<{$} >{$}l<{$} >{$}l<{$} r}
	Minimer 		& 	x_1	 & + \ \ x_2 & + \ \ x_3 \\
	med hensyn til 	&  	x_1	 & +   2 x_2 & +   3 x_3 &  	 & = 3 \\
					&  -x_1	 & +   2 x_2 & +   6 x_3 & 		 & = 2 \\
					&  \ \ 	 & \ \ 4 x_2 & +   9 x_3 & 		 & = 5 \\
					&  \ \ 	 & \ \   	 & \ \ 3 x_3 & + x_4 & = 1 \\
	og $x_1, \dots, x_4 \geq 0$.
\end{tabular}
\end{center}

Der gives nu følgende hjælpeproblem, hvor der er tilføjet kunstige variable.

\begin{center}
\begin{tabular}{l >{$}r<{$}	>{$}r<{$} >{$}l<{$} >{$}r<{$} >{$}r<{$} >{$}r<{$} >{$}r<{$} r}
	Minimer    		&  	x_1	 & + \ \ x_2 & + \ \ x_3 &       & + Mx_5    & + Mx_6    & + Mx_7 \\
	med hensyn til 	&  	x_1	 & +   2 x_2 & +   3 x_3 &       & + \ \ x_5 &           &           & = 3 \\
					&  -x_1	 & +   2 x_2 & +   6 x_3 &       &           & + \ \ x_6 &           & = 2 \\
					&        &     4 x_2 & +   9 x_3 &       &           &           & + \ \ x_7 & = 5 \\
					&   	 &           &     3 x_3 & + x_4 &           &           &           & = 1 \\
	og $x_1, \dots, x_7 \geq 0$.
\end{tabular}
\end{center}

En mulig basisløsning til hjælpeproblemet fås ved at lade $[x_5, x_6, x_7, x_4] = \vec{b}^T = [3,2,5,1]$. 
Den korresponderende basismatrix er da identiteten. 
Derudover er $c_B = (M,M,M,0)$. 
Den reducerede kost for hver variabel $x_i$ evalueres ved $c_i - c'_BA_i$, hvilket giver følgende tabel.

\begin{center}
\begin{tabular}{|c|c|ccccccc|}
\hline
	 &  & $x_1$ & $x_2$ & $x_3$ & $x_4$ & $x_5$ & $x_6$ & $x_7$ \\
\hline
	 & $-10M$ & 1 & $-8M+1$ & $-18M+1$ & 0 & 0 & 0 & 0 \\
\hline
	$x_5=$ & 3 & 1  & 2 & 3   		 & 0 & 1 & 0 & 0 \\
	$x_6=$ & 2 & -1 & 2 & 6			 & 0 & 0 & 1 & 0 \\
	$x_7=$ & 5 & 0  & 4 & 9   		 & 0 & 0 & 0 & 1 \\
	$x_4=$ & 1 & 0  & 0 & \textbf{3} & 1 & 0 & 0 & 0 \\
\hline
\end{tabular}
\end{center}

Den reducerede kost for $x_3$ er negativ når $M$ er stor nok. 
Derfor er $x_3$ nu en del af basen, og $x_4$ er ikke. 
For at sætte den reducerede kost af $x_3$ til nul, skal hele rækken multipliceres med $6M- \frac{1}{3}$, samt lægge det til den nulte række. 
Den nye tabel vil så se ud som følger.

\begin{center}
\begin{tabular}{|c|c|ccccccc|}
\hline
	 &  & $x_1$ & $x_2$ & $x_3$ & $x_4$ & $x_5$ & $x_6$ & $x_7$ \\
\hline
	 & $-4M- \frac{1}{3}$ & 1 & $-8M+1$ & 0 & $6M- \frac{1}{3}$ & 0 & 0 & 0 \\
\hline
	$x_5=$ & 2 	  & 1  & 2		    & 0 & -1 & 1 & 0 & 0 \\
	$x_6=$ & 0 	  & -1 & \textbf{2} & 0 & -2 & 0 & 1 & 0 \\
	$x_7=$ & 2	  & 0  & 4 		 & 0 & -3 & 0 & 0 & 1 \\
	$x_3=$ & $\frac{1}{3}$ & 0 	 & 0 & 1  & $\frac{1}{3}$ & 0 & 0 & 0 \\
\hline
\end{tabular}
\end{center}

Det ses nu, at den reducerede kost af $x_2$ er negativ når $M$ er tilstrækkelig stor, hvorfor $x_2$ nu bringes ind i basen og $x_6$ udgår. 
Tabellen ser nu således ud.

\begin{center}
\begin{tabular}{|c|c|ccccccc|}
\hline
	 &  & $x_1$ & $x_2$ & $x_3$ & $x_4$ & $x_5$ & $x_6$ & $x_7$ \\
\hline
	 & $-4M- \frac{1}{3}$ & $-4M+ \frac{3}{2}$ & 0 & 0 & $-2M+ \frac{2}{3}$ & 0 & $4M- \frac{1}{2}$ & 0 \\
\hline
	$x_5=$ & 2 & \textbf{2} & 0 & 0 & 1 & 1 & -1 & 0 \\
	$x_2=$ & 0 & $- \frac{1}{2}$ & 1 & 0 & -1 & 0 & $\frac{1}{2}$ & 0 \\
	$x_7=$ & 2 & 2 & 0 & 0 & 1 & 0 & -2 & 1 \\
	$x_3=$ & $\frac{1}{3}$ & 0 & 0 & 1 & $\frac{1}{3}$ & 0 & 0 & 0 \\
\hline
\end{tabular}
\end{center}

På samme måde bringes $x_1$ nu ind i basen, samtidig med $x_5$ udgår. 

\begin{center}
\begin{tabular}{|c|c|ccccccc|}
\hline
	 &  & $x_1$ & $x_2$ & $x_3$ & $x_4$ & $x_5$ & $x_6$ & $x_7$ \\
\hline
	 & $- \frac{11}{6}$ & 0 & 0 & 0 & $-\frac{1}{12}$ & $2M-\frac{3}{4}$ & $2M+ \frac{1}{4}$ & 0 \\
\hline
	$x_1=$ & 1 & 1 & 0 & 0 & $\frac{1}{2}$ & $\frac{1}{2}$ & $-\frac{1}{2}$ & 0 \\
	$x_2=$ & $\frac{1}{2}$ & 0 & 1 & 0 & $-\frac{3}{4}$ & $\frac{1}{4}$ & $\frac{1}{4}$ & 0 \\
	$x_7=$ & 0 & 0 & 0 & 0 & 0 & -1 & -1 & 1 \\
	$x_3=$ & $\frac{1}{3}$ & 0 & 0 & 1 & $\mathbf{\frac{1}{3}}$ & 0 & 0 & 0 \\
\hline
\end{tabular}
\end{center}

Nu bringes $x_4$ ind i basen, og $x_3$ udgår. 

\begin{center}
\begin{tabular}{|c|c|ccccccc|}
\hline
	 &  & $x_1$ & $x_2$ & $x_3$ & $x_4$ & $x_5$ & $x_6$ & $x_7$ \\
\hline
	 & $- \frac{7}{4}$ & 0 & 0 & $\frac{1}{4}$ & 0 & $2M-\frac{3}{4}$ & $2M+ \frac{1}{4}$ & 0 \\
\hline
	$x_1=$ & $\frac{1}{2}$ & 1 & 0 & $-\frac{3}{2}$ & 0 & $\frac{1}{2}$ & $-\frac{1}{2}$ & 0 \\
	$x_2=$ & $\frac{5}{4}$ & 0 & 1 & $\frac{9}{4}$ & 0 & $\frac{1}{4}$ & $\frac{1}{4}$ & 0 \\
	$x_7=$ & 0 & 0 & 0 & 0 & 0 & -1 & -1 & 1 \\
	$x_4=$ & 1 & 0 & 0 & 3 & 1 & 0 & 0 & 0 \\
\hline
\end{tabular}
\end{center}

Med en stor nok værdi for M er alle reducerede kostværdier ikke-negative, og der er fundet en optimal løsning til hjælpeproblemet.
Alle de kunstige variable er blevet lig nul, hvorfor der nu ligeledes haves en optimal løsning til det originale problem.  
\end{eks}