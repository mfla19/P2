\chapter{indledning}
\section{Optimering}
Dette projekt omhandler optimering ved hjælp af lineær programering %case hvis case
Når der tales om optimering, så er det at maximere eller minimere et udkom, baseret på nogle lineære uligheder(subjects). \\
Dette kunne være, at en fabrik skal producere bukser og jakker, de har 2000$m^2$ polyester og 100$0m^2$ bomuld som resurse.
Hvert par bukser kan sælges for 30 kroner og kræver 2$m^2$ bomuld og 3$m^3$ polyester, og hver jakke sælges for 50 kroner, men kræver 4$m^2$ bomuld og 3,5$m^2$ polyester. Ud af dette vil den højeste indtjening være det som skal maksimeres …lidt om lineær algebra måske, dvs uddybet om hvordan det kan skrives op \\
\subsection{Historisk sammenhæng}
%kilde: Bertsimas
Lige siden industrialiseringen, har fabriker ønsket at optimere deres arbejdsproduktion og maksimere deres indtjeninger. \\ %mangler lidt
Dette varede helt ind i anden verdenskrig, men her var optimeringsproblemet mere militært omhandlende, som for eksempel hvordan optimere våbenproduktion og transport. Derfor var der mange matematikere rundt omkring i verden, der blev adspurgt om hvordan sådan en optimering kunne foretages. \\
En af disse matematikere var den sovjetiske Leonid Kantorovitj. Kantorovitj formulerede udtrykket lineær programmering og fandt en løsning til optimeringsproblemet, men hans resultater blev ikke kendt på det daværende tidspunkt. \\
Sammme tid som Kantorovitj, var der en økonom ved navn Tjalling Koopmans, som også interesserede sig for løsningen af optimeringsproblemet og vandt i 1975 en nobelpris sammen med Kantorovitj.\\

%kilde Lay
Efter anden verdenskrig, hvor sovjet unionen murede Vest Berlin inde, så der ikke kunne blive transporteret vare ind i byen, blev de allieret nødt til at transportere der vare over muren, ved hjælp af fly. Denne proces skulle optimeres så mest mulige vare transporteret med mindst mulige antal fly. \\
Dette fik matematikeren George Dantzig til at finde en mere praktisk metode til at bruge lineær programmering til at løse optimeringsproblemet, nemlig simplex metoden. %lidt beskrivelse af metoden, måske kun hvis linæer programering bliver beskrevet tidligere