\chapter{indledning}
\section{Optimering}
Dette projekt omhandler optimering ved hjælp af lineær programering %case hvis case
Når der tales om optimering, så er det at maximere eller minimere et udkom, baseret på nogle lineære uligheder(subjects). \\
Et optimeringsproblem kan være, at en fabrik skal producere bukser og jakker, de har $2000m^2$ polyester og $1000m^2$ bomuld som resurse.
Hvert par bukser kan sælges for $30$ kroner og kræver $2m^2$ bomuld og $3m^3$ polyester, og hver jakke sælges for $50$ kroner, men kræver $4m^2$ bomuld og $3,5m^2$ polyester. 
Den højeste indtjening er så det, som i dette eksempel skal maksimeres. 
Lineær programmering kan altså bruges til optimering af alverdens ting rundt omkring i verden.
Lineær programmering bygger på en foregående forståelse af lineær algebra, herunder blandt andet hvad matricer samt lineære rækkeoperationer er.
Løsning af lineære ligningssystemer er også meget vigtigt i forhold til at kunne løse lineære programmeringsproblemer.
En basal forståelse af lineære algebra må derfor gå forud for en forklaring af lineær programmering. 
\\
\subsection{Historisk sammenhæng}
%kilde: Bertsimas
Lige siden industrialiseringen, har fabriker ønsket at optimere deres arbejdsproduktion og maksimere deres indtjeninger. \\ %mangler lidt
I optakten til og starten af anden verdenskrig, blev en sovjetisk matematiker ved navn Leonid Kantorovich interesseret i, den optimale ressourcefordeling i en planøkonomi, for at formindske sovjetunionens omkostninger og maksimere fjendens udgifter. Ud fra dette gav Kantorovich lineær programmering en formulering. Kantorovich gav også en løsningsmetode, men denne løsning blev ikke kendt daværende tidspunkt.\\
Samtidig med Kantorovich's formulering og løsning, kom  mange andre løsninger på banen, og en af dem kom fra den Hollandsk-amerikanske økonom ved navn Tjalling Koopmans. Koopmans formulerede dog problemet i forhold til en klassisk økonomi-model.\\
Koopmans og Kantorovich delte i 1975 en nobelpris i økonomi\\
I 1947 foreslog matematikeren George Dantzig en algoritme, Simplex Metoden, hvilket gjorde løsning af lineære programmeringsproblemer mere praktisk.\\
Dantzigs Simsplex metode er et historisk punkt for udvikling, da den blev lavet i en tid, hvor informationsteknologi var i udvikling.
\\%lay
Under den kolde krig, hvor Sovjetunionen havde indemuret Vestberlin, kunne vesten ikke transportere resurser ind i Vestberlin på andre måder end via luften.
Fly kunne dog ikke transportere et uendeligt antal varer ind i Vestberlin, fordi flyene både var begrænsede af benzin og plads. 
Transporten af disse resurser var derfor optimeringsproblemet.


%Dette varede helt ind i anden verdenskrig, men her var optimeringsproblemet mere militært omhandlende, som for eksempel hvordan optimere våbenproduktion og transport. Derfor var der mange matematikere rundt omkring i verden, der blev adspurgt om hvordan sådan en optimering kunne foretages. \\
%En af disse matematikere var den sovjetiske Leonid Kantorovitj. Kantorovitj formulerede udtrykket lineær programmering og fandt en løsning til optimeringsproblemet, men hans resultater blev ikke kendt på det daværende tidspunkt. \\
%Sammme tid som Kantorovitj, var der en økonom ved navn Tjalling Koopmans, som også interesserede sig for løsningen af optimeringsproblemet og vandt i 1975 en nobelpris sammen med Kantorovitj.\\

%kilde Lay
%Efter anden verdenskrig, hvor sovjet unionen murede Vest Berlin inde, så der ikke kunne blive transporteret vare ind i byen, blev de allieret nødt til at transportere der vare over muren, ved hjælp af fly. Denne proces skulle optimeres så mest mulige vare transporteret med mindst mulige antal fly. \\
%Dette fik matematikeren George Dantzig til at finde en mere praktisk metode til at bruge lineær programmering til at løse optimeringsproblemet, nemlig simplex metoden. %lidt beskrivelse af metoden, måske kun hvis linæer programering bliver beskrevet tidligere