\chapter{Indledning}
\subsection{Historisk sammenhæng}
%kilde: Bertsimas
Lige siden industrialiseringen, har fabrikker ønsket at optimere deres arbejdsproduktion og maksimere deres indtjeninger. \\ %mangler lidt
I optakten til og starten af anden verdenskrig, blev en sovjetiske matematiker ved navn Leonid Kantorovich interesseret i den optimale ressourcefordeling i en planøkonomi for at formindske sovjetunionens omkostninger og maksimere fjendens udgifter. Ud fra dette gav Kantorovich lineær programmering en formulering. Kantorovich gav også en løsningsmetode, men denne løsning blev ikke kendt daværende tidspunkt.\\
Samtidig med Kantorovich's formulering og løsning, kom  mange andre løsninger på banen, og en af dem kom fra den Hollandsk-amerikanske økonom ved navn Tjalling Koopmans. 
Koopmans formulerede dog problemet i forhold til en klassisk økonomi-model.\\
Koopmans og Kantorovich delte i 1975 en nobelpris i økonomi\\
I 1947 foreslog matematikeren George Dantzig en algoritme, Simplex Metoden, hvilket gjorde løsning af lineære programmeringsproblemer mere praktisk.\\
Dantzigs Simsplex metode er et historisk punkt for udvikling, da den blev lavet i en tid, hvor informationsteknologi var i udvikling.
\\%lay
Under den kolde krig, hvor Sovjetunionen havde indemuret Vestberlin, kunne vesten ikke transportere resurser ind i Vestberlin på andre måder end via luften.
Fly kunne dog ikke transportere et uendeligt antal varer ind i Vestberlin, fordi flyene både var begrænsede af benzin og plads. 
Transporten af disse resurser var derfor optimeringsproblemet.


%Dette varede helt ind i anden verdenskrig, men her var optimeringsproblemet mere militært omhandlende, som for eksempel hvordan optimere våbenproduktion og transport. Derfor var der mange matematikere rundt omkring i verden, der blev adspurgt om hvordan sådan en optimering kunne foretages. \\
%En af disse matematikere var den sovjetiske Leonid Kantorovitj. Kantorovitj formulerede udtrykket lineær programmering og fandt en løsning til optimeringsproblemet, men hans resultater blev ikke kendt på det daværende tidspunkt. \\
%Sammme tid som Kantorovitj, var der en økonom ved navn Tjalling Koopmans, som også interesserede sig for løsningen af optimeringsproblemet og vandt i 1975 en nobelpris sammen med Kantorovitj.\\

%kilde Lay
%Efter anden verdenskrig, hvor sovjet unionen murede Vest Berlin inde, så der ikke kunne blive transporteret vare ind i byen, blev de allieret nødt til at transportere der vare over muren, ved hjælp af fly. Denne proces skulle optimeres så mest mulige vare transporteret med mindst mulige antal fly. \\
%Dette fik matematikeren George Dantzig til at finde en mere praktisk metode til at bruge lineær programmering til at løse optimeringsproblemet, nemlig simplex metoden. %lidt beskrivelse af metoden, måske kun hvis linæer programering bliver beskrevet tidligere

\section{Optimering}
Dette projekt omhandler optimering ved hjælp af lineær programering %case hvis case
Når der tales om optimering, så er det at maximere eller minimere et udkom, baseret på nogle lineære uligheder(subjects). \\
Et optimeringsproblem kan være, at en fabrik skal producere bukser og jakker, de har $2000m^2$ polyester og $1000m^2$ bomuld som resurse.
Hvert par bukser kan sælges for $30$ kroner og kræver $2m^2$ bomuld og $3m^3$ polyester, og hver jakke sælges for $50$ kroner, men kræver $4m^2$ bomuld og $3,5m^2$ polyester. 
Den højeste indtjening er så det, som i dette eksempel skal maksimeres. 
Lineær programmering kan altså bruges til optimering af alverdens ting rundt omkring i verden.
Lineær programmering bygger på en foregående forståelse af lineær algebra, herunder blandt andet hvad matricer samt lineære rækkeoperationer er.
Løsning af lineære ligningssystemer er også meget vigtigt i forhold til at kunne løse lineære programmeringsproblemer.
En basal forståelse af lineære algebra må derfor gå forud for en forklaring af lineær programmering. 
\\

\section{Problemanalyse}
%Dette projekt vil starte med en gennemgang af relevandt viden indenfor lineær algebra, som er nødvendig for at kunne forklare lineær programmering. Herefter vil det blive forklaret, hvad lineær programmering er, samt hvad det kan bruges til.Lineær programmering vil så blive forklaret ud fra et geometrisk perspektiv og derefter vil simplex metoden blive gennemgået. Til sidst vil den opnåede viden om lineær programmering blive brugt til, at løse en case om optimering. Denne case handle

I dette projekt vil der blive arbejdet mod løsningen af en specifik case. 
Denne case omhandler en virksomhed, som skal optimere tildelingen af arbejdsopgaver til deres ansatte.
Virksomheden skal have udført et bestemt antal arbejdsopgaver og hver opgave skal udføres et antal gange. 
De ansatte har hver især forskellige lønninger, nogle kan tage flere timer end andre, og de kan udføre specifikke opgaver med forskellig effektivitet.
Der skal altså tages højde for hvilke opgaver de ansatte hver skal udføre, samt hvor mange timer de skal arbejde ud fra hvor høj deres løn er og hvor mange timer de maksimalt må have.
Firmaet skal så afgøre, hvilke ansatte der skal udføre hvilke opgaver, for at firmaet får færrest mulige udgifter til løn. \\
Casen skal løses ved brug af lineær programmering og derfor vil dette blive gennemgået forud for løsningen af casen. 
Herunder vil den geometriske tilgang til lineæer programmering, samt simplex metoden blive gennemgået. 
Inden lineær programmering beskrives, vil der blive gennemgået nogle centrale ting indenfor lineær algebra, som skal forstås for at kunne løse et linæert programmeringsproblem.
Alt dette leder frem til følgende problemformulering.

\subsection{problemformulering}
Hvad er lineæer programmering og hvordan kan det løses geometrisk og ved hjælp af simplex metoden? Hvordan kan virksomheden i casen optimere deres lønningsudgifter ved hjælp af lineær programmering?
