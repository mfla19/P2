\chapter{Problemanalyse}
%Dette projekt vil starte med en gennemgang af relevandt viden indenfor lineær algebra, som er nødvendig for at kunne forklare lineær programmering. Herefter vil det blive forklaret, hvad lineær programmering er, samt hvad det kan bruges til.Lineær programmering vil så blive forklaret ud fra et geometrisk perspektiv og derefter vil simplex metoden blive gennemgået. Til sidst vil den opnåede viden om lineær programmering blive brugt til, at løse en case om optimering. Denne case handle

I dette projekt vil der blive arbejdet mod løsningen af en specifik case. 
Denne case omhandler en virksomhed, som skal optimere tildelingen af arbejdsopgaver til deres ansatte.
Virksomheden skal have udført et bestemt antal arbejdsopgaver og hver opgave skal udføres et antal gange. 
De ansatte har hver især forskellige lønninger, nogle kan tage flere timer end andre, og de kan udføre specifikke opgaver med forskellig effektivitet.
Der skal altså tages højde for hvilke opgaver de ansatte hver skal udføre, samt hvor mange timer de skal arbejde ud fra hvor høj deres løn er og hvor mange timer de maksimalt må have.
Firmaet skal så afgøre, hvilke ansatte der skal udføre hvilke opgaver, for at firmaet får færrest mulige udgifter til løn. \\
Casen skal løses ved brug af lineær programmering og derfor vil dette blive gennemgået forud for løsningen af casen. 
Herunder vil den geometriske tilgang til lineæer programmering, samt simplex metoden blive gennemgået. 
Inden lineær programmering beskrives, vil der blive gennemgået nogle centrale ting indenfor lineær algebra, som skal forstås for at kunne løse et linæert programmeringsproblem.
Alt dette leder frem til følgende problemformulering.

\section{problemformulering}
Hvad er lineæer programmering og hvordan kan det løses geometrisk og ved hjælp af simplex metoden? Hvordan kan virksomheden i casen optimere deres lønningsudgifter ved hjælp af lineær programmering?
